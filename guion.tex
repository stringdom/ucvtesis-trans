\chapter{Guion de entrevista}\label{guion}

Guion para entrevistas a profundidad semi-estructurada a personas trans
para la tesis «Identidad de género y transición en las personas trans de
Caracas»

\section{Introducción y presentación}

Mi nombre es \_\_\_\_\_\_\_\_\_\_\_\_\_\_ y soy estudiante de psicología,
mención social. En este momento me encuentro realizando una investigación
para mi tesis de grado.
El propósito de esta entrevista es conocer y comprender la perspectiva de
las personas trans respecto a un conjunto de temas sobre los que
iré preguntando.
Todo lo que conversemos el día de hoy será confidencial y sólo será utilizado
con fines académicos.

\subsection{Datos personales:}

\begin{itemize}
\item
  Nombre
\item
  Edad
\item
  Trabajo/oficio/profesión/ocupación
\item
  Educación
\end{itemize}

\section{Ejes temáticos}

	\subsection{Identidad}

		\subsubsection{Identidad de género}

\begin{itemize}
\item
  ¿Quién eres?
\item
  ¿Qué aspectos tuyos sientes que te definen?
\item
  ¿Te sientes cómodo con tu cuerpo en este momento?
\item
  ¿Te identificas con el sexo con el cual naciste?
\item
  ¿Con cuál género te identificas?
\item
  ¿Has hecho algo para cambiar tu cuerpo y tu sexo?
\item
  ¿Cómo ha sido tu experiencia de transición?
\end{itemize}

\subsubsection{Ámbito laboral}

\begin{itemize}
\item
  ¿Estas trabajando?, ¿en que estás trabajando?
\item
  ¿Cómo es tu relación con tus compañeros de trabajo?
\item
  ¿Cómo ha sido tu experiencia buscando trabajo?
\end{itemize}

\subsubsection{Devenir, familia}

\begin{itemize}
\item
  Cuéntame acerca de tu infancia
\item
  ¿Cómo te sentías respecto al género durante tu infancia?
\item
  ¿Como te sentías respecto a tu cuerpo durante tu infancia?
\item
  ¿Cuales consideras tú que son los momentos más importantes de tu vida?
\item
  ¿Cómo te diste cuenta que querías cambiar de género?
\item
  ¿Cómo fue ese proceso de descubrimiento?
\item
  ¿Cómo han reaccionado tus padres respecto a tu decisión de cambiar de
  género?
\item
  ¿Cómo son las relaciones con otros miembros de tu familia?
\end{itemize}

\subsubsection{Relaciones de pareja}

\begin{itemize}
\item
  ¿Cómo ha sido tu experiencia con las relaciones de pareja?
\end{itemize}

\subsubsection{Expresión}

\begin{itemize}
\item
  ¿Con qué aspectos de tu género asumido te identificas?
\item
  ¿Cómo es tu rutina diaria en cuanto aspecto?
\item
  ¿Cómo te gusta arreglarte y vestirte?
\item
  ¿En que piensas antes de salir a la calle?
\end{itemize}

\subsection{Estatus legal}

\begin{itemize}
\item
  ¿Cómo ha sido tu experiencia con el aspecto legal?
\item
  ¿Cómo ha sido tu experiencia cuando debes sacarte la cédula?
\item
  ¿Cómo es cuando te piden la cédula?
\item
  ¿Cuál ha sido tu experiencia con autoridades e instituciones que
  requieren una identidad legal?
\end{itemize}

\subsection{Discriminación}

\begin{itemize}
\item
  ¿Has vivido situaciones de discriminación? De ser así, ¿Cómo has
  lidiado con ellas?
\item
  ¿Conoces casos que hayan tenido consecuencias graves?
\end{itemize}

\subsection{Transición}
\subsubsection{Disforia de género}

\begin{itemize}
\item
  ¿Qué conoces acerca de la transexualidad?
\item
  ¿Cómo es tu experiencia cuando necesitas o quieres atención médica?
\end{itemize}

\subsubsection{Concepción del cuerpo}

\begin{itemize}
\item
  ¿Cómo ha sido tu experiencia hasta el momento viviendo con tu género?
\item
  ¿Qué aspectos te agradan más de tu cuerpo?, ¿Qué aspectos te gustaría
  cambiar?
\end{itemize}

\subsubsection{Significación de los procesos de transición}

\begin{itemize}
\item
  ¿Has pensado iniciar o te encuentras en proceso de transición?

Si ya ha iniciado transición:

\item
  ¿Cómo fue tu primera aproximación a la transición?
\item
  ¿Cómo obtienes u obtuviste información sobre como hacer la transición?
\item
  ¿Cómo ha sido el proceso de transición hasta ahora?

Si aún no inicia transición:

\item
  ¿Cómo te imaginas que será la transición?
\item
  ¿Qué ha prevenido que inicies la transición?
\end{itemize}

\subsection{Perspectiva a futuro}

\begin{itemize}
\item
  ¿Cómo te visualizas a futuro?
\end{itemize}
