% 15 de mayo de 2018
\chapter{Presentación y discusión de la información}\label{ch:informacion}

El proceso de entrevistas fue realizado entre los meses de abril y septiembre
del año 2017.
Estas entrevistas fueron transcritas para su análisis en el texto.
Los elementos expresados por los participantes en las entrevistas fueron
codificados y organizados en las categorías y subcategorías que se pueden
observar en la figura~\ref{fig:categorias}.
Las entrevistas fueron codificadas empleando el programa ATLAS.ti en su versión
7.5.7 proceso que facilitó la categorización de la información recolectada a lo
largo de las entrevistas.

\begin{figure}
    \centering
    \includegraphics[width=0.75\textwidth]{categorias}
    \caption{Diagrama de categorías}\label{fig:categorias}
\end{figure}

A continuación realizaremos una descripción de cada una de ellas junto con los
verbatim que les dan origen.
Adicionalmente presentamos el razonamiento e interpretación que damos a cada
categoría y sus implicaciones caso a caso para el cumplimiento de los objetivos
de investigación.
El orden de presentación de las mismas fue elegido
según la frecuencia de manifestación en las entrevistas realizadas.

En la sección~\ref{sec:discusion}, realizamos una discusión y análisis punto a
punto de cada una de las categorías.

\section{Presentación de la información}

Como se puede ver resumido en la figura~\ref{fig:categorias}, hemos elaborado
seis (6) categorías. Dentro de estas categorías ‘La transición’ es uno de los
componentes identificados como constitutivo del proceso de construcción de la
identidad de las personas trans. Procederemos a explorar el contenido de cada
una de las categorías identificadas.

\subsection{Desarrollo}

Esta categoría se encuentra compuesta por elementos que
abarcan desde etapas tempranas de la niñez y del desarrollo. Elementos como la
relación del individuo con su escolaridad y compañeros de clases. También
elementos de la constitución de la identidad de género que se hacen presentes
dentro de la pubertad y los conflictos que estos puedan causar a los
participantes También incluye el papel que juega la familia dentro de la
constitución de la identidad, así como la forma de aproximarse a los problemas.
Todos estos son elementos que pueden marcar la construcción de identidad de una
persona.

\begin{figure}
    \centering
    \includegraphics[width=0.75\textwidth]{desarrollo}
    \caption{Diagrama ‘desarrollo y familia’}\label{fig:desarrollo}
\end{figure}

En esta categoría confluyen elementos de conflicto, así como las experiencias que
permitieron el manejo de los mismos en los participantes y que
consecuentemente se transformaron en estrategias de afrontamiento. Se debe tomar
en cuenta que en esta categoría se presentan los distintos tipos de relación
establecidas tanto en niveles académicos como familiares entre la infancia y
pubertad de los participantes.

Existen dos subcategorías dentro de este factor. Una de ‘infancia y pubertad’,
que reporta sobre las experiencias de desarrollo y formativas en la juventud, y
otra de ‘familia’. Esta habla acerca de la influencia de los lazos y ambiente
familiar durante el desarrollo.

\subsubsection{Infancia y pubertad}
Dentro de esta subcategoría se pueden
encontrar elementos que parecen ser transversales a lo largo de la vida del
individuo. Al ser tanto la infancia como la pubertad etapas tempranas de
desarrollo y constitución se pudo identificar elementos que parecen sentar las
bases para la formación de otros elementos que le permiten al individuo constituirse
como persona. Entre estos elementos cabe resaltar la presencia de conflictos
como lo expresa el caso~1 (verbatim):

\begin{verbatim}
…bueno me identificaba como niño pero me gustaba todo las cosas
de niña. De hecho, siempre pedía al niño Jesús cosas de hembra, por ejemplo,
barbie, oso etcétera. Por su puesto, jamás me traía lo que pedía y era muy
triste para mí, una era inocente y el niño Jesús me dejaba una carta explicando
que eso eran cosas de niña y me traía patineta bicicleta carrito y a mí no me
gustaba…
\end{verbatim}

Este elemento puede estar ligado al rechazo que pueden vivir, según lo expresado
por el caso~1 (verbatim):

\begin{verbatim}
…en la escuela era algo terrible por el bullying, pero yo
siempre imponía carácter y jamás me deje amedrentar por nada ni nadie. De hecho
me agarre a golpes y me expulsaron por 10 días…
\end{verbatim}

O a tener su raíz en conflictos por la constitución de su identidad como lo
expresan con los siguientes verbatims.

Caso~1:

\begin{verbatim}
…yo lloraba porque no me entendía y me sentía mal.
\end{verbatim}

Caso~2:

\begin{verbatim}
  …me criticaban mucho como me vestía, pero es lo que me gusta.
\end{verbatim}

Y caso~3:

\begin{verbatim}
…siempre era el raro del grupo…
\end{verbatim}

La aparición de estos conflictos entra en contacto con las aproximaciones a los
roles de género que suceden por primera vez en la infancia por medio de los
juegos y que perduran a lo largo de la vida de un individuo, como se evidencia
en el relato del caso~1:

\begin{verbatim}
…a mí me decían que tenía que jugar con carritos y no con
muñecas porque eso es de niñas y yo no era una…
\end{verbatim}

Y del caso~3:

\begin{verbatim}
…me gustaba tener el cabello corto y no me arreglaba pero mis
padres me decían que tenía que arreglarme para poder verme linda…
\end{verbatim}

 Un elemento que se ve relacionado con la presencia del conflicto de identidad
 es que la aprehensión del mismo lleva al individuo a buscar o considerar el
 daño que puede causar el no lidiar con esta situación y es por eso que hay
 comentarios como el emitido por el caso~3 quien expresa sobre su
 adolescencia:

 \begin{verbatim}
Descubrí que tenía que hablar de esto con alguien o me iba a volver loco.
 \end{verbatim}

 Este verbatim, trae a la luz el malestar que nace en el individuo por este
 conflicto de identidad, posiblemente el hecho de estar en una situación de la
 cual se tiene poca o ninguna información al alcance del individuo hace que el
 malestar por el conflicto sea mayor y pueda tener consecuencias más peligrosas
 para la identidad de la vida del individuo.

\subsubsection{Familia}

Otro de los componentes de la categoría ‘Desarrollo’ es la subcategoría
denominada \emph{Familia}. Según lo expresado por los participantes existen obstáculos para la discusión intrafamiliar de la identidad, por ejemplo, el
caso~3:

\begin{verbatim}
…en mi casa siempre era un conflicto hablar sobre cómo me sentía.
\end{verbatim}

Caso~2:

\begin{verbatim}
Mi mamá me decía, ‘hija arréglate un poco’ ó ‘te verías muy linda
con vestido’ y yo le decía que no me gustaba eso y venía el regaño.
\end{verbatim}

Esta subcategoría influye en la constitución del individuo y modela su
desarrollo, en aspectos como la forma de establecer relaciones,
tomando lo expresado por el caso~2:

\begin{verbatim}
…la relación con mis padres es distinta, a mi madre le costó más
aceptarme. A mi padre, después de explicarle, lo entendió con más facilidad,
incluso suelo pasar por su trabajo, es moto taxista.
\end{verbatim}

Un elemento que resaltan los participantes es el cambio de dinámicas dentro de
la familia que se da producto de asumirse como persona transgénero. Como se observa en
el relato del caso~1:

\begin{verbatim}
…mi madre al principio le costó mucho aceptarlo y mi papá le
dijo a mis hermano que por qué no me quedaba como yo. Era que él me aceptaba gay
más no vestido de mujer y jamás me permitió aclararle los conceptos que tenía
errado y hasta hoy no me trata ni mi padre ni mi hermano menor…
\end{verbatim}

Estos verbatims asoman que dentro de las familias de los individuos transgéneros
existe una innegable presencia de conflicto generado por la nueva identidad
asumida por el mismo. El cómo esta identidad interactúa con las prenociones de
la familia suma a la conflictividad durante el desarrollo. Es posible que esta
situación esté relacionada con el propio conflicto interno que pueden vivir
personas transgénero como consecuencia de la falta de acceso a información que
permita aclarar sus dudas.

Esto podría ser un indicador de las percepciones hegemónicas sobre la sexualidad
y como estas moldean las diversas interacciones entre los individuos. Se
observa en el verbatim del caso~1 que por parte de su padre existe aún una
confusión entre lo que es la orientación (los gustos del individuo) y su
identidad (como el individuo se observa a sí mismo) sexual.

\subsection{Género y sexualidad}

La siguiente categoría que permite la constitución de identidad en los
participantes es la de \emph{género y sexualidad}. En esta categoría se
presentan elementos como la constitución de la identidad que tienen su raíz en
etapas tempranas de la vida. Con base a lo expresado por los participantes se
generaron las siguientes subcategorías: Identidad de género, Socialización de género
y Orientación Sexual.

En la figura~\ref{fig:genero} se pueden observar los elementos constitutivos de
la categoría.

\begin{figure}
    \centering
    \includegraphics[width=0.75\textwidth]{genero}
    \caption{Diagrama ‘género y sexualidad’}\label{fig:genero}
\end{figure}

\subsubsection{Identidad de género}

Esta subcategoría se mostró como la más importante dentro de lo expresado por
los participantes principalmente porque ha sido el componente central con
el que se enfrentaron en la constitución de su identidad. Esto se puede ver
reflejado en lo expresado por el caso~2:

\begin{verbatim}
…el momento más feliz de mi infancia es cuando me pude vestir
de Ángel Gabriel en un nacimiento viviente que hicieron en mi escuela. Mi mamá
no quería pero yo me sentía en las nubes porque por un momentito pude verme como
quería.
\end{verbatim}

Este conflicto de identidad también se hace visible dentro del relato del caso~3
cuando expresa:

\begin{verbatim}
…no es que solo me gustaba vestirme como hombre, es que pensaba
como uno también.
\end{verbatim}

Esto también lleva a cuestionarse el papel de los roles de género en la
concepción de este ‘pensar como’ y pone en perspectiva crítica el significado de
‘pensar como hombre’ o ‘pensar como mujer’. Dentro del relato del caso~1
comenta:

\begin{verbatim}
Cuando era pequeña no me molestaba, pero cuando empecé a
desarrollarme le rezaba a Dios para que por favor me salieran senos.
\end{verbatim}

Estos verbatims permiten identificar algunos aspectos relacionados con el
conflicto de identidad de los individuos. Elementos como el verbatim expresado
por el caso~2 que considera un recuerdo muy feliz el poder mostrar una
expresión de género acorde a la identidad que, ya para esa etapa de su
desarrollo, sentía era la de él. Además, esto se puede reforzar con el verbatim
del caso~1 cuando expresa que le causaba preocupación y ansiedad el hecho que
sus senos no se desarrollaran junto con el resto de su cuerpo.

Además, se presenta la dicotomía entre lo que las personas sienten que son, las
expectativas de como esperan verse y la imposición de la identidad que se
les asignó por haber nacido con un sexo específico.

\subsubsection{Socialización de género}

Esta subcategoría surge de la relación que los participantes han tenido con
elementos propios de los roles de género, especialmente del rol del género
asignado según su sexo biológico. Es decir, está compuesta por aquellos
elementos que demarcan la diferencia entre lo femenino y lo masculino. Elementos
dentro del relato del caso~1 hacen presentes estas diferencias:

\begin{verbatim}
…el niño Jesús me dejaba una carta explicando que eso eran cosas de niña y me
traía patineta, bicicleta, carritos y a mí no me gustaba…
\end{verbatim}

Así como lo expresado por el caso~3:

\begin{verbatim}
Me decían que tenía que sentarme bien y como niña, y yo me preguntaba ¿cómo es
eso?
\end{verbatim}

Partiendo de estos verbatims se puede observar que existe un modelaje hacia el
individuo sobre el conjunto de expectativas específicas relacionadas con cada
género. Además, se evidencia la insistencia de parte del entorno social por
conformarse a aquellas expresiones de género propios del género que les fue
asignado por su sexo biológico en el momento del nacimiento.

En estos verbatims se pueden observar, además, que existe una atribución, bien
sea masculina o femenina, sobre cosas (juguetes), o actos comportamentales.
Estas atribuciones son arbitrarias, pues en el caso del verbatim del caso~1,
los juguetes que le regalaban podían también ser asociados como juguetes de
niños o en el caso del caso~3, que condiciones específicas son las que
determinan el cómo se debe sentar una persona dependiendo de su sexo o género.
Esto forma parte de los roles y expresiones de género, y son los primeros
elementos a los que los participantes parecen demostrar rechazo.

\subsubsection{Orientación sexual}

El último componente de la categoría de Género y sexualidad es la subcategoría
de ‘Orientación Sexual’. En esta subcategoría se presentan aquellas inquietudes
e incongruencias que surgieron en la vida de los individuos al momento de buscar
pareja sexual y romántica. No solo fundamentado en aspectos físicos sino también
emocionales. El caso~3 relata:

\begin{verbatim}
 Yo solía tener sexo telefónico con una amiga en bachillerato y en una de esas
 fantasías yo era un hombre y ella lo seguía y esas cosas sucedían y el primer
 nombre de mi personaje fue Alex y ahí empezó a asomarse algo que se transformó
 en lo que soy ahora.
\end{verbatim}

Este verbatim asoma un elemento resaltante de la orientación sexual en personas
transgénero, su consolidación se hace presente durante la adolescencia,
permitiendo de esta manera influir en la constitución identitaria del individuo.
Esto parece ser un elemento compartido en común tanto entre personas cisgénero
como transgéneros.

Así se puede interpretar que las personas transgénero dan muestras de su
identidad de género en los juegos sexuales de exploración
tempranos. Reforzando la idea de que estos presentan preferencia por presentarse
según su género sentido desde muy temprano en su vida.

Existe, además, una confusión asociada a la expresión de la sexualidad cuando no
se tienen interiorizados los conceptos propios de lo transexual o transgénero.
El caso~3 afirma al referirse sobre su experiencia sexual temprana:

\begin{verbatim}
Al principio todo hombre trans se cree lesbiana.
\end{verbatim}

El caso~2 refuerza esta confusión:

\begin{verbatim}
…yo no conocía de lo trans ni nada, pensaba que era homosexual y ya…
\end{verbatim}

Esto refuerza la concepción de que la orientación puede ser confundida con la
identidad, sobre todo cuando no existe un concepto claro de la transexualidad o
transgenerismo en la persona.

\subsection{Discriminación}

Esta categoría representa el segundo elemento más importante según lo expresado
por los participantes. La discriminación parece encontrarse inevitablemente
ligada a la condición trans. Los eventos discriminantes a los que ellos se ven
sujetos abarcan desde situaciones de rechazo como los presentes en las vivencias del caso~1:

\begin{verbatim}
La población trans está muy expuesta al rechazo, porque no nos entienden,
piensan que somos unos bichos raros y nos ven y tratan como tales.
\end{verbatim}

Estas situaciones pueden llegar a convertirse en actos de acoso o bullying como
los que comenta haber vivido el caso~2:

\begin{verbatim}
Cuando inicié mi transición los compañeros de trabajo de mi papá se metían
conmigo, hasta que un día les ofrecí unos golpes y todo cambió.
\end{verbatim}

 Puede incluso llegar a generar en el individuo miedo, así como una sensación de
 inseguridad como lo expresa el caso~3:

 \begin{verbatim}
Estando en Ecuador supe de un trans al que violaron y mataron y la verdad me dio
miedo.
 \end{verbatim}

Con base a estos verbatims se puede establecer un punto de referencia que
permite generar subcategorías en lo que se puede considerar una categoría
principal denominada \emph{Discriminación}. Los componentes de esta categoría se
pueden observar en la figura~\ref{fig:discriminacion}. Así mismo partiendo de
estos verbatims se puede pensar que las experiencias discriminantes son
trasversales a la vida de una persona transgénero ya que se pueden presenciar
tanto en etapas tempranas como en etapas más adultas de la vida de los
participantes como se presenta a continuación.

\begin{figure}
    \centering
    \includegraphics[width=0.75\textwidth]{discriminacion}
    \caption{Diagrama ‘discriminación’}\label{fig:discriminacion}
\end{figure}

\subsubsection{Rechazo}

Un elemento común en el discurso de los participantes es la presencia de
vivencias de rechazo, es decir situaciones en las que han sido víctimas de actos
que impiden su participación regular como miembro de la sociedad y que pueden
atentar contra la integridad de un individuo. Esto se hace evidente dentro del
relato del caso~2 cuando expresa:

\begin{verbatim}
Yo solía ir a comprar ropa de hombre y las vendedoras me decían que no. Por qué
compraba eso si yo era mujer y me miraban como con asco y yo les respondía que
lo hacía porque me gusta la ropa y porque tengo la plata para hacerlo.
\end{verbatim}

Así como también dentro del relato del caso~1 cuando comenta que:

\begin{verbatim}
…era horrible cuando me tocaba sacarme la cédula y me obligaban a ir vestida
como hombre, el personal que me atendía se ponía muy hostil conmigo cuando
llegaba maquillada.
\end{verbatim}

Estos verbatims permiten visibilizar elementos discriminantes que afectan la
expresión de identidad de género de los individuos transgénero. El rechazo que
pueden vivir parece tener sus raíces en condiciones socialmente construidas
relacionadas con lo que se espera que sea la expresión de género de una persona
de sexo masculino o femenino. Esta identidad que se le impone a las personas
transgéneros (sin importar su etapa de transición) afecta el proceso
que transitan, generando malestar en el individuo. Debido a su decision de
alinear su expresión de género con la identidad de género con la cual se sienten
identificados se ven criticados y despreciados por comentarios como los
expresados por los participantes.

\subsubsection{Acoso / Bullying}

Otro elemento importante dentro de la vivencia de la discriminación por parte de
los participantes es la relación que ellos desarrollan con el acoso o el bullying.
El caso~1 expresa que:

\begin{verbatim}
…en la escuela era algo terrible por el bullying, pero yo siempre imponía
carácter y jamás me deje amedrentar por nada ni nadie. De hecho, me agarre a
golpes y me expulsaron por 10 días…
\end{verbatim}

Esto demuestra que desde etapas tempranas de su vida se encuentran presentes
elementos de acoso. El caso~2 expresa en su relato también que las situaciones
de acoso se pueden vivir en ambientes académicos:

\begin{verbatim}
Mientras estudiaba para sacar mi título de sexólogo me pasó que una docente era
particularmente agresiva conmigo, porque una vez le respondí feo. La razón de
esto fue porque ella estaba cuestionando mi identidad. Pero nada al final le
saqué un 20 y le callé la boca.
\end{verbatim}

Resalta el hecho de que, por su condición como personas transgénero, hechos como
el acoso o el bullying se hagan presentes con gran peso dentro del relato de los
participantes. Es también importante evaluar las estrategias con las cuales los
participantes lidian con estas situaciones. Parece ser que dependiendo del tipo
de acto que vaya en contra de ellos, ya sea violencia física o simbólica, para
poder ser redimido o valorado positivamente es necesario rebasar o exceder
aquellas expectativas que han sido impuestas por el agresor hacia el agredido.
Esto podría acarrear consecuencias negativas para la persona transgénero pues
vivir con la constante presión de tener que ser valorado solo por sus logros
puede llegar a ser una fuente de estrés negativo en la vida de la persona
transgénero.

\subsubsection{Consecuencias}

Por último, es necesario remarcar que tanto el rechazo como el acoso o bullying
tienen consecuencias en la vida de la persona transgénero. Por ejemplo, el caso~2
relata:

\begin{verbatim}
Hace poco supe de un caso de una chica trans, ella aún no comenzaba con su
tratamiento hormonal pero ya estaba expresando una identidad de género con la
que se sentía cómoda. La cosa es que por ser trans su familia la botó de la casa
y terminó en situación de calle, prostituyéndose para sobrevivir hasta que la
mataron hace una semana, la encontraron en un monte.
\end{verbatim}

Además, se puede tomar en cuenta el comentario del caso~1, quien labora en una
oficina de atención social:

\begin{verbatim}
…aquí nos llegan muchos casos de personas trans que se quedan sin casa o trabajo
por querer ser felices, principalmente pasa con chicas, algunas se terminan
suicidando…
\end{verbatim}

 Esto pone sobre la mesa las consecuencias negativas que tienen situaciones
 aversivas producto de la discriminación. Estas siempre son una sombra de temor
 sobre la vida de las personas trans y es lo que les pone en riesgo de
 situaciones como la prostitución o el suicidio. Estas consecuencias tienen su
 origen en los elementos anteriormente mencionados en las subcategorías previas.
 Tanto vivir el rechazo por parte de otras personas, así como el acoso y la
 necesidad de tener que encontrar estrategias de supervivencia que permitan al
 individuo valorarse ante otros como par hace que la vida de las personas
 transgénero este cargada de dificultades.

\subsection{La Transición}

En esta categoría agrupamos los elementos relacionados con el tránsito entre un
género o sexo a otro. Usualmente del género o sexo asignado al nacer hacia el
género o sexo deseado. Los elementos componentes de esta categoría se pueden
apreciar en la figura~\ref{fig:transicion}. Estos incluyen la elaboración que
realizan los participantes del proceso de transición en sí mismo desde su
subjetividad, la apariencia física y el conocimiento sobre los procedimientos
médicos, quirúrgicos y sociales que permiten llevar a cabo la transición.

\begin{figure}
    \centering
    \includegraphics[width=0.75\textwidth]{transicion}
    \caption{Diagrama ‘la transición’}\label{fig:transicion}
\end{figure}

\subsubsection{Transitar}

Según lo expresado por los participantes, el transitar va más allá de ser un
simple hecho, también es una herramienta. La transición es un medio que
les permite llega a ser quien ellos sienten que realmente son. El transitar
puede abarcar aspectos quirúrgicos y/u hormonales, y como lo expresa el caso~2:

\begin{verbatim}
…las hormonas ayudan pero la operación libera…
\end{verbatim}

El caso~3 coincide al comentar:

\begin{verbatim}
…las hormonas ayudan, cuando dejé de tomarlas perdí todo, era una niña…
\end{verbatim}

Estos comentarios evidencian que existe una valoración de la alteración
quirúrgica como más importante que el tratamiento hormonal. Además, se considera
la transición quirúrgica como un objetivo final o como una acción liberadora.
Cabe preguntarse lo que sucedería si no se puede concretar una transición
quirúrgica.

Sin embargo, lo dicho por el caso~3 hace referencia a como el componente
hormonal tiene una importancia por su cualidad inmediata. Debido a que permite
una visibilización a corto plazo de la identidad, más explícita que una
operación. Los efectos de la transición hormonal son visibles mientras que la
cirugía genital no lo es. De igual manera, el tratamiento de reemplazo hormonal
requiere de una toma constante. Por ello dejar de tomarlo implica una reversión
rápida de sus efectos.

Por otro lado, se tiene que tomar en cuenta lo expresado por el caso~1:

\begin{verbatim}
…por cuanto mi apariencia física y genética me han ayudado en la transición lo
cual ha sido muy fácil, de hecho, no he tomado hormona nunca…
\end{verbatim}

Esto permite indicar que la importancia no está colocada en la toma de hormonas
en sí misma. Sino que la importancia está en los cambios de aspecto físico y de
visibilidad que se producen como resultado de la influencia hormonal. En esta
instancia el caso~1 indica no necesitar de la toma hormonal pues su aspecto ya
es femenino en sí, lo que es su objetivo último.

Este elemento se relaciona con la Apariencia física, pues es el aspecto—
efectivamente la expresión de género—lo que la persona transgénero desea alterar. La
función biológica es secundaria a sus efectos sobre la expresión.

\subsubsection{Apariencia física}

Aquí podemos encontrar la expresión de la relación sexo-género. Es decir, la
relación existente entre el género y las caracterísitcas sexuales secundarias.
Se debe tomar en cuenta que dentro de esta relación hay particularidades que
tienen un mayor peso o que suelen tomarse más en cuenta para su expresión como
lo plantea, por ejemplo, el caso~3 al referirse a un tiempo en el cual dejo de
tomar hormonas:

\begin{verbatim}
  …perdí el torso perdí básicamente eso, la libido cambio…
\end{verbatim}

Esto implica que el poder verse y expresarse en concordancia con los rasgos del
sexo o género al que se transita es fundamental. Además de que también tiene
importancia el cómo se siente. Aquí refleja uno de los efectos de la
testosterona sobre el impulso o deseo sexual. No sólo desea verse, sino sentir
su propio cuerpo como aquel del sexo deseado.

Además de esto se puede tomar en cuenta lo expresado por el caso~2:

\begin{verbatim}
…le tengo terror a la regla porque soy hombre y a los hombres no les debe venir
eso.
\end{verbatim}

Este refleja la importancia del ‘sentirse como’. No se trata, sin embargo, de
una búsqueda funcional. La menstruación como fenómeno biológico tiene un
objetivo reproductivo. No es la función o la menstruación lo que rechaza el
participante sino el desarreglo con la conformación estereotípica de la
dicotomía sexual. No se rechaza el tener la menstruación sino el hecho de tener
la menstruación siendo hombre. Esto no se ajusta con la construcción física
propia del hombre y por ello es rechazado.

Este elemento visibiliza la concepción de concordancia de sexo/género que
prevalece, pues estar alineado no solo puede significar tener características
propias de un sexo o género, sino que también puede significar no tener o evitar
las asociadas con otro.

\subsubsection{Conocimiento}

La subcategoría de conocimiento hace referencia tanto a cómo los participantes
reafirmaron o conocieron su condición, así como a la manera con la cual entraron
en contacto con el proceso que les permitiese su transición. En esta
subcategoría resalta el papel de los medios para visibilizar la
condición trans, el caso~2 expresa que:

\begin{verbatim}
…yo no conocía de lo trans ni nada, pensaba que era homosexual y ya, pero
luego un día viendo televisión con mi novia de ese momento pasaron el primer
capítulo del programa ‘Taboo’, por NatGeo y ahí presentaron a una persona trans
y mi novia me dijo, mira ella dice que se sentía como tú dices que te sientes
y pues eso me dejó pensando.
\end{verbatim}

Podemos entonces afirmar que hay aún un desconocimiento de la diferencia entre
la orientación sexual y la identidad sexual. Además, se evidencia que es
necesario para las personas trans aprender e interiorizar los significados de:
género, sexo, identidad sexual y transición, para poder dar sentido a la propia
experiencia. Sin estos significados no es posible para la persona trans comenzar
a poner en cuestionamiento el género asignado y la propia identidad.

Por otro lado, se puede tomar como referente la experiencia del caso~1 que
expresa:

\begin{verbatim}
Me di cuenta y logré comprender que era ser trans a los 40 años de edad, porque
desconocía que era ser trans, de hecho, en mi ignorancia tampoco sentía que me
identificaba como gay porque me sentía mujer pensaba como mujer y eso no lo
entendía, todo eso lo viví en silencio por 40 años jamás dije nada por el
rechazo que pudiera sentir hasta que me hablaron de personas trans y fue allí
cuando comencé a entender que yo era una mujer transexual.
\end{verbatim}

Esto refuerza la diferencia entre la orientación, la identidad sexual y la
identidad de género. También señala la vivencia privada y subjetiva de la
mayoría de las personas transgénero quienes suelen mantener su confusión en
silencio por miedo al rechazo. Refuerza también la importancia de la
comunicación de los conceptos alrededor del género y la identidad para poder
asumirse a sí mismo como persona trans.

El caso~1 añade:

\begin{verbatim}
Lo descubrí a los 40 años gracias a mi psicólogo que me realizo una terapia y
allí fui abriéndome y diciendo lo que sentía y fue cuando supe que era una mujer
trans.
\end{verbatim}

Esto potencia el rol del psicólogo como voz autorizada y mediador en el proceso
de transición, calmando el malestar del individuo. El psicólogo aquí también
ejerce la función de educador de la persona con un malestar que no ha podido
significar o expresar de una manera satisfactoria. La adquisición de conceptos
psicológicos y los símbolos propios de la identidad de género le permite a la
persona trans construir su propia interpretación de su condición e identidad.

\subsection{Cuerpo y genitalidad}

Esta categoría se centra en el aspecto biológico de la identidad del individuo
pero no abarca componentes hormonales o cromosómicos como tales sino que se
centra únicamente en el aspecto fenotípico genital desde lo estético y lo
sensitivo así como lo que representa para los participantes este aspecto de su
cuerpo.

En la figura~\ref{fig:genitalidad} se pueden observar los componentes de esta
categoría.

\begin{figure}
    \centering
    \includegraphics[width=0.75\textwidth]{genitalidad}
    \caption{Diagrama ‘cuerpo y genitalidad’}\label{fig:genitalidad}
\end{figure}

\subsubsection{Lo esencial}

Esta subcategoría abarca aquello que los participantes consideraron de mayor
importancia sobre su relación con el cuerpo, es interesante observar que el
componente genital no tiene una primacía en la construcción de su identidad. Si
se toma en cuenta lo expresado por el caso~1 cuando se le pregunta por la
transición quirúrgica:

\begin{verbatim}
Solo me decidí operar el pecho y ponerme mamas.
\end{verbatim}

Esto es resaltante porque, aunque el sentirse como perteneciente al género
deseado es importante, aún más importante es ser recibido y aceptado socialmente
como el género con el cual se identifica. Esto permite la validación social de la identidad del individuo. En este sentido la opción quirúrgica
es evaluada en su función para la expresión del género deseado, y por ello se le
da prioridad a la mamoplastia.

También el caso~3 comenta:

\begin{verbatim}
¿Para qué un trans se opera el pecho? para poder quitarse la camisa.
\end{verbatim}

Y el caso~2 agrega:

\begin{verbatim}
Quiero poder operarme para poder quitarme la camisa y hacer cosas normales.
\end{verbatim}

Una vez más se ve reforzado que el cambio corporal esencial es aquel que permite
la inserción social. Parece resaltar la importancia de lo que se considera
caracteres sexuales secundarios sobre los primarios. Es decir, es más importante
lo que ve el otro en público que lo que se ve en privado. Comentarios como el
del caso~2:

\begin{verbatim}
…a mí me encanta ir al gimnasio y tener la espalda ancha, me
siento como un monstruo.
\end{verbatim}

Refuerzan la idea de que lo esencial es mostrarse para todos y ser recibido cómo
el género con el cual hay identificación. Aquí el uso de monstruo es un sentido
positivo. Desde la mirada que asocia a un ‘monstruo’ con fuerza, tamaño físico,
poder y vitalidad. Características estereotípicamente asociadas con la
masculinidad. Además de que la idea que yace detrás es la realización de las
interacciones sociales típicas del rol de género por el que se desea pasar. En
este caso, el hombre que va al gimnasio es para hacerse fuerte, muscular, grande
y poderoso.

\subsubsection{Lo que se tiene}

Además de lo encontrado en la subcategoría anterior los participantes expresaron
que es más importante el genital que se tiene, y como este permite relacionarse
con una pareja, que buscar que los genitales coincidan con el género con el que
se identifican. Es por esto que la presente subcategoría nace, tomando en cuenta
comentarios realizados por los participantes, como por ejemplo, lo expresado por
el caso~3:

\begin{verbatim}
Lo mío es mío y con esto resuelvo.
\end{verbatim}

A lo que le agrega:

 \begin{verbatim}
…un pene falso a nivel sexual es solo un instrumento.
 \end{verbatim}

También comenta el caso~2 sobre la cirugía de reasignación de sexo:

\begin{verbatim}
Eso después hace que pierdas sensibilidad
\end{verbatim}

Se puede interpretar de estas expresiones que existe una preferencia a acercarse
a la relación sexual desde los genitales con los que se nació. La genitalidad
propia del género con el cual se sienten identificados es un aspecto secundario.
A esto se le puede sumar la importancia asignada a la cirugía de mamas, ya sea
para removerlas o crearlas. Esto está relacionado al valor exclusivamente
femenino que tienen los senos en la sociedad como un aspecto visual y evidente
que señaliza la femineidad.

\subsection{El Otro}

Esta categoría fue nombrada debido al peso que tiene la presencia del otro en
la vida de cada individuo dentro de la sociedad. Debido a que el ser humano es
un ser intrínsecamente social existen principios que rigen las interacciones, así
como diversas formas para poder satisfacer las necesidades que surgen de estas
interacciones. Tomando en cuenta que la condición trans puede ser disruptiva
dentro de lo socialmente aceptado se puede comprender que la importancia del
otro cobra un sentido particularmente fuerte en estos casos pues el otro puede
moldear la forma en la que se asume el género, así como patrones de
comportamientos, vestimenta y roles que permitan al individuo expresarse según
el género con el que se sienten identificados.

En la figura~\ref{fig:otro} se puede observar la estructura de los elementos que
componen esta categoría.

\begin{figure}
    \centering
    \includegraphics[width=0.75\textwidth]{otro}
    \caption{Diagrama ‘el otro’}\label{fig:otro}
\end{figure}

\subsubsection{Cómo me ven}

La primera subcategoría es sobre la percepción general que tienen los demás de
la persona transexual, o al menos cómo estos la conciben. Tomando en cuenta lo
expresado por el caso~3:

\begin{verbatim}
  …me solían ver como el raro…
\end{verbatim}

Caso~1:

\begin{verbatim}
Les causa impresión y cambia todo cuando doy mi cedula para pagar maquillaje.
\end{verbatim}

Parece ser que el rol transgresor de la condición transgénero es lo que permea
el trato que estas personas reciben. Es decir, cuando alguien interactúa con una
persona trans el saber o no que se encuentra hablando con una persona trans va a
afectar el trato que reciban. Además, el trato que reciben es percibido como
regular hasta que la otra persona se da cuenta que está interactuando con
alguien trans,

El caso~2 agrega:

\begin{verbatim}
La gente juzga mucho y como que esperan más de ti, que seas más hombre que otro
hombre o más mujer que otra mujer.
\end{verbatim}

Esto demuestra como la persona transgénero se ajusta a la mirada de los otros
por medio de asumir expresiones propias del género con el que se encuentran
identificados. Bien sea mediante comportamientos o alterando su apariencia
física. Resalta el calificativo utilizado por el caso~2 ‘más hombre que otro
hombre o más mujer que otra mujer’. Esto implica que existe una presión y una
expectativa por cumplir con los estereotipos de los roles de género. Incluso más
allá de lo que sería razonable para una persona cisgénero.

Cabe señalar en este aspecto un comentario realizado por el caso~3:

\begin{verbatim}
Yo llevo un aparatico para poder usar el baño de hombres, porque sabes los
hombres se miran a veces y a mí me gusta usar un urinario y orinar de pie.
\end{verbatim}

Este relato entra en contacto con lo expresado por el caso~1:

\begin{verbatim}
Yo soy muy coqueta, me hice así una vez me asumí mujer trans, antes no me
arreglaba mucho.
\end{verbatim}

Parece ser que el adaptar o incluir roles de conducta propios de un género no es
un hecho exclusivo para la aceptación privada, sino es también un elemento
de validación con el otro.

\subsubsection{Aspecto legal}

Complementando a la subcategoría anterior se presenta el ‘Aspecto legal’, que
indica la forma en la cual son reconocidos legalmente las personas transgénero.
Este es un elemento constitutivo de su identidad debido a que al ser ciudadanos
de la República Bolivariana de Venezuela necesitan contar con la seguridad legal
de que sus derechos van a ser respetados y el primer paso es poseer una
identidad legal, como el caso~1 lo expresa:

\begin{verbatim}
…otro momento en el que fui muy feliz fue cuando pude sacarme la foto de la
cédula con maquillaje y expresándome como soy de verdad.
\end{verbatim}

El poder ser identificada legalmente según su expresión de género es algo muy
significativo pues los reafirma como ciudadanos con derechos. Si bien el cambio
de sexo en la identidad no es posible aún, ser reconocido en la fotografía de su
identidad y aparecer cómo realmente se expresan en el día a día representa una
pequeña victoria. El caso~2 refuerza este planteamiento comentando que:

\begin{verbatim} …cuando legalmente se nos pueda identificar sin ningún problema
pero va a ser una victoria muy importante, el siguiente paso sería quitarnos el
prefijo trans, yo soy un hombre y ya no un hombre trans, eso no es inclusivo.
\end{verbatim}

\section{Discusión}\label{sec:discusion}

Desde su concepción, la presente investigación ha tenido como principal foco de
interés proveer una mirada no patologicista a lo que es el transgenerismo, esta
propuesta nace de la necesidad de romper con la noción del transgenerismo como
una enfermedad mental—forma como ha sido catalogada desde aproximadamente la
década de los ochenta según manuales de diagnóstico como el DSM-IV y el CIE-10.
La visión del transgenerismo como un hecho antinatural o que no es normal viene
asociada a la construcción de identidades en la sociedad por medio de dicotomías
rígidas que facilitan la patologización de expresiones humanas que se salgan de
lo \emph{establecido}.

Esto entra en contraste con respecto a lo planteado por la Teoría Queer, que
como lo explican Fonseca y Quintero (2009) asume a la expresión de género como
un continuo; y que  si bien pueden existir extremos no son más que puntos de
referencia, marcadores, entre los cuales se pueden encontrar las diversas
expresiones de género y sexualidad del ser humano. Al tomar esto en cuenta
parece ser posible complejizar la condición transgénero y verla como un hecho
que sufre al ser simplificada por una visión dicotómica de la identidad sexual y
la identidad de género.

Partiendo de la codificación realizada proponemos que la identidad de los
participantes se construye de una manera específica, marcada por las
interacciones que se llevan a cabo en su cotidianidad, desde su infancia hasta
su adultez. Es por esto que se presenta a continuación una discusión detallada
de las categorías elaboradas en base a lo recopilado en las entrevistas
realizadas a los participantes.

\subsection{Desarrollo}

Esta categoría nace de las vivencias de los sujetos desde su infancia, las
cuales se convierten en los cimientos que modelan la forma en la que se
desenvolverán en la sociedad. Es necesario remarcar que la condición transgénero
en los participantes es asumida no como una elección, sino como algo que es, no
es un capricho o moda, es un sentir real y auténtico que en muchos casos los
lleva a un estado de conflicto pues no se logra encontrar una forma de balancear
la identidad sentida y lo socialmente esperado.

Este conflicto se puede tomar como punto de partida para entender la identidad
transgénero. Algunos autores como Hernández, Rodríguez y
García-Valdecasas (2010), afirman que el malestar o conflicto que pueden sufrir
las personas transgéneros viene asociado a la dificultad que encuentran al tener
que identificarse o alinearse con modelos socialmente construidos sobre lo que
significa ser un hombre o una mujer. Esta visión parece también estar influida
por una perspectiva biomédica, asumiendo que el componente genital es el único
indicador determinante al momento de asumir una identidad de género, ya sea
masculina o femenina. Como muestra de esto se puede tomar en cuenta la
prevalencia de instrumentos de diagnóstico como el Manual diagnóstico y
estadístico de los trastornos mentales para identificar y clasificar a
una persona como transgénero y que relegan otros elementos propios de la
identidad humana como lo son elementos sociales, culturales,
históricos-personales que pueden influir en la constitución de la identidad del
individuo. Sin embargo, es necesario resaltar que la edición actual del DSM
(en este momento la quinta) ha ajustado su enfoque al fenómeno transgénero y ha
cambiado la nomenclatura por la cual se identificaba a esta condición.

Entonces esto lleva a cuestionar cual es la comprensión que se tiene sobre el
sexo, sexualidad e identidad desde una visión no científica y patologizadora.
Podría plantearse entonces que debido a una predominancia del discurso médico
esta visión se ve sesgada, limitando estos aspectos de la vida humana a aquello
que ha sido validado por una visión biomédica. Esta visión puede que sea la
que se reproduce en distintos núcleos familiares y que se cristaliza en los
miembros de la familia y que puede llegar a causar conflictos cuando uno de
estos miembros interactúa con, por ejemplo, la identidad transgénero.

En las entrevistas realizadas a los participantes se presentó un elemento en
común y es que tanto la infancia como la adolescencia fueron etapas críticas de
su desarrollo en las cuales entraron en contacto con su identidad como persona
transgénero. En estas etapas es cuando que surge la duda en ellos sobre la
incongruencia entre el género con el cual se identifican y las expresiones
(juegos, ropa, roles) que se les pedía, o se esperaba, que expresaran. A raíz de
este conflicto presentaban situaciones que afectan la dinámica familiar de los
participantes. Al existir una visión ligada a la relación que debe existir
entre un determinado sexo biológico y su expresión de género por parte de
miembros de la familia al entrar en contacto con la identidad transgénero de los
participantes podía llegar a causar confusión, así como conflictos.

Esto refuerza planteado anteriormente y es que la visión de lo que
significa ser hombre o mujer encuentra sus cimientos en estas etapas de la vida
de un individuo. Es en estas etapas cuando se entra en contacto con lo que se
esperaría fuera la expresión e identidad de género de una persona según los
genitales con los que nacieron, hecho que se aleja de la realidad pues como lo
pueden confirmar las personas transgénero el nacer con un determinado genital no
influye directamente en la identidad sentida por el individuo.

Es importante cuestionar que elementos validan la construcción de lo que
significa hombre o mujer, juegos, ropa, roles en general. Según lo expresado por
los participantes, cuando se identificaban con elementos relacionados con el
género sentido, recibían una respuesta que daba a entender que esa expresión,
con la que se identificaban, no estaba en concordancia con lo que ellos debían
ser. Pero esto es un elemento cuestionable pues, tomando en cuenta el verbatim
del caso 1:

\begin{verbatim}
…el niño Jesús me dejaba una carta explicando que eso eran cosas de niña y me
traía patineta bicicleta carrito y a mí no me gustaba.
\end{verbatim}

Cabe cuestionarse que es lo que determina que una patineta, bicicletas o carros
de juguete sean juguetes expresamente para niños y no niñas. Especialmente
cuando en la realidad social existen hombres y mujeres que son deportistas
profesionales y usan patinetas, bicicletas y carros para competir. Estos
elementos que parecen estar rígidamente asociados a una concepción de hombre o
mujer parecen perder esta rigidez cuando son confrontados con ejemplos más
cercanos a la realidad social. Esto también se puede presenciar al explorar el
verbatim expresado en el caso 3:

\begin{verbatim}
…me gustaba tener el cabello corto y no me arreglaba pero mis padres me decían
que tenía que arreglarme para poder verme linda…
\end{verbatim}

Este verbatim permite acercarse a la construcción que se tiene de lo que es la
feminidad y la concepción de una mujer. Cabe cuestionar que es lo que se
entiende por “arreglarse para verse linda” que se expresa en este verbatim. Pero
es resaltante pues conforma un elemento asociado a la concepción de mujer,
además de permitir cuestionar que si el arreglarse es un elemento asociado a la
expresión de feminidad ¿si un hombre lo hace estaría atentando contra su
masculinidad? Esto podría ser cierto si se considera al género como un hecho
binario y no como un continuo a través del cual los individuos en una sociedad
van moviendo su expresión según aquello que les permita vivir una vida plena.

Esta visión del género como un continuo se hace muy presente en la construcción
identitaria que viven las personas transgéneros en su infancia y adolescencia
como se pudo presenciar en los verbatims recopilados de las entrevistas
realizadas a los participantes. Se puede afirmar esto por la presencia de un
hecho en común en la vida de los tres casos. Ellos se ven en la necesidad de
tener una expresión de género que no es acorde al género con el que se
identifican y, sin embargo, progresivamente su expresión de género se alinea con
la identidad de género con la que se identifican.

Otro elemento común en los tres casos es el conflicto que surge
con sus familiares cuando se aproximan al transgenerismo. Conflicto que
puede tener consecuencias a largo plazo en las relaciones interpersonales con
miembros de la familia. En determinados casos la interacción con un
miembro de la familia, que no comprende la condición transgénero del individuo,
puede llegar a interrumpirse indefinidamente.

Es de resaltar que parece existir una prevalencia de estos conflictos con los
padres de los individuos transgénero y que, dependiendo de las estrategias
empleadas para acercar a los padres al transgenerismo, los conflictos pueden ser
solventados. Estos conflictos de no ser solucionados adecuadamente pueden llegar
a tener consecuencias como la mencionada en el caso~1:

\begin{verbatim}
…mi madre al principio le costó mucho aceptarlo y mi papa le dijo a mis hermano
que porque no me quedaba como yo era que él me aceptaba gay más no vestido de
mujer y jamás me permitió aclararle los conceptos que tenía errado y hasta hoy
no me trata ni mi padre ni mi hermano menor.
\end{verbatim}

En este verbatim se puede observar que, la falta de estrategias para encarar la
transición del individuo transgénero, puede llegar a romper las relaciones
familiares. Esto representa un reto para las personas al momento de informar a
su familia sobre la transición que planean emprender, pues puede ser tomado de
mala manera y la persona terminaría perdiendo apoyo emocional importante por
parte de sus familiares. La pérdida de este apoyo podría no solo
romper relaciones dentro de la familia, la persona transgénero podría verse
expulsada de su hogar por parte de sus padres o ser víctima de violencia física
o emocional.

Es por esto que se puede considerar a la familia de la persona transgénero como
un elemento vital en la constitución de su identidad. No solo la familia puede
afectar al individuo durante etapas críticas del desarrollo, también puede
influir sobre la decisión de informar su transición, tomando en cuenta que esto
podría fracturar a la familia.

\subsection{Género y sexualidad}

Esta categoría fue compuesta en base a lo expresado por los participantes
cuando se les cuestiono sobre el género y la sexualidad. Un primer elemento que
surgió de estas entrevistas es precisamente el conflicto que nace en los
individuos al tener que vivir una identidad de género con la que no se
identifican. Como ha sido mencionado anteriormente este conflicto se hace
presente en etapas tempranas del desarrollo de los individuos como resalta en el
relato del caso~2:

\begin{verbatim}
…el momento más feliz de mi infancia es cuando me pude vestir de Ángel Gabriel
en un nacimiento viviente que hicieron en mi escuela, mi mamá no quería pero yo
me sentía en las nubes porque por un momentito pude verme como quería.
\end{verbatim}

Ya se hacía presente en su infancia la discordancia con el género que se le
esperaba expresara. Esta discordancia, entre lo que se siente ser y las
expectativas de como esperan que se exprese causa un gran conflicto en los
individuos pues dependiendo del género con el que se identifiquen tendrán
expectativas distintas como se ve dentro del relato del caso~1:

\begin{verbatim}
Cuando era pequeña no me molestaba, pero cuando empecé a desarrollarme le rezaba
a Dios para que por favor me salieran senos
\end{verbatim}

El hecho de que estos conflictos se presenten en etapas tempranas del desarrollo
del individuo sumado a que sean tan persistentes y tan fuertes permiten
confirmar que la identidad de género es una realidad que viven los individuos y
que el deseo de transición se encuentra presente desde etapas tempranas del
desarrollo. Esto ayuda a visibilizar que casos como el del hijo de la cantante
Karina referenciado por el caso~2 son realidades cristalizadas en el individuo
desde edades tempranas de la vida y que deben ser respetadas por ser un elemento
muy importante para la constitución de la identidad del individuo y que estos
viven como fuera de su control voluntario.

Además de esto otro elemento que surge de la codificación de las entrevistas
realizadas a los participantes y que en contacto con una realidad presente en la
sociedad es la concepción de la superioridad del hombre sobre la mujer desde el
patriarcado. Autores como Lerner (1990) expresan que el patriarcado es un
sistema en el que los roles de género se encuentran claramente marcados y
segmentados en una relación de opresor y oprimidas.

Es por esto que el relato del caso~1 contrasta contra los del caso~2 y el
caso~3. En el caso~1, para ella el asumirse y expresarse como mujer fue un
proceso más arduo que para el caso~2 y 3 el asumirse y expresarse como hombres.
Principalmente porque el adaptarse a los roles de género es más complicado
debido a las exigencias impuestas a las mujeres socialmente. Es decir, el tener
que verse femenina puede ser un reto después de casi toda una vida donde no era
un factor determinante para sus interacciones sociales. Desde lo expresado por
el caso~1 se reproduce una visión que impone a la mujer valores de belleza,
atractivo y sensualidad como condición para poder ser valoradas y encima demanda
la maternidad para validarles como \emph{verdaderas} mujeres.

Por otro lado, el caso~2 y 3 expresan que para ellos el asumir roles masculinos
va ligado a una adquisición de estatus. Con la presencia de musculatura, un
cuerpo visualmente masculino, les permite desarrollarse con normalidad en la
sociedad, validados por defecto por la condición adquirida de masculinidad.
Asociado a esto viene la conducta de cortejar a varias parejas
potenciales.

Partiendo de lo expresado por los participantes podría plantearse que desde la
mirada transgénero existen dos direcciones de cambio que tienen subyacentes una
diferencia. Un hombre que transita a mujer pierde el privilegio masculino y debe
validarse socialmente con otras cualidades, mientras que una mujer que transita
a hombre adquiere el privilegio masculino y su valor será dado por sentado
siempre que no se ponga en duda su masculinidad.

Parece entonces ser posible afirmar que la construcción de la identidad de
género de los participantes se encuentra muy influenciada por los estereotipos
de género. Desde esta perspectiva se podría plantear que para una persona
transgénero pueda sentirse en concordancia con el género con el que se
identifica tiene que ser más hombre o mujer que aquellas personas que no son
transgénero, como si existiera una validación por parte del otro basado en la
capacidad de expresión del género.

Lo expresado por los participantes permite confirmar que existe una construcción
social sobre lo que significa ser hombre o ser mujer donde se le atribuye a lo
masculino elementos más relacionados con expresiones de fuerza y agresión
mientras que a lo femenino se le atribuyen expresiones más delicadas,
relacionadas principalmente con el autocuidado. Estas concepciones pueden ser
las que permiten una dominación hegemónica patriarcal dentro de la sociedad.

Un último elemento que resalta dentro de esta categoría es la relación que
establecieron los individuos con su orientación sexual. En los relatos de los
participantes se encontraron verbatims como el del caso~3:

\begin{verbatim}
Yo solía tener sexo telefónico con una amiga en bachillerato y en una de esas
fantasías yo era un hombre y ella lo seguía y esas cosas sucedían y el primer
nombre de mi personaje fue Alex y ahí empezó a asomarse algo que se transformó
en lo que soy ahora.
\end{verbatim}

Este verbatim permite confirmar que la identidad de género no condiciona la
orientación sexual de un individuo. Es decir, el interés o atracción que puede
sentirse hacia una persona de un sexo y/o género determinado no depende del sexo
o género de aquel que siente dicha atracción.

Sin embargo, si el individuo desconoce la existencia de la identidad transgénero,
pero vive el conflicto de identidad pueden surgir situaciones como las
planteadas en el caso~2:

\begin{verbatim}
 …yo no conocía de lo trans ni nada, pensaba que era homosexual y ya…
\end{verbatim}

Que se ven reforzadas por la siguiente afirmación expresada por el caso~3:

\begin{verbatim}
Al principio todo hombre trans se creen lesbiana.
\end{verbatim}

Lo expresado por los participantes permite afirmar que un individuo transgénero
que desconoce la existencia de esta identidad podría considerarse a sí mismo como
homosexual. Esto es un elemento que trae a colación lo complejo que resulta la
constitución de identidad si no se cuenta con las herramientas o una guía
adecuada. Además se visibiliza que se puede llegar a ignorar elementos del
conflicto que vive el individuo para poder alinearse o dar sentido a la
identidad de género con la que se identifica el individuo, pero que aún no ha
sido expresada.

Sumado a esto se puede interpretar que puede cambiar la etiqueta bajo la cual se
describe la atracción sexual y romántica del individuo, sin embargo, la fuente
de esta etiqueta descriptiva, la persona de un sexo o género especifico, se
puede mantener estable. En otras palabras, si una persona transgénero MaH no ha
iniciado su proceso de transición y siente atracción por personas cisgénero
femeninas esta condición no va a cambiar una vez comience su transición o
a lo largo de la misma, pues este es un elemento separado dentro de su
identidad.

Entonces es posible considerar que, si bien la identidad de género no condiciona
la orientación sexual de un individuo, si pueden presentarse situaciones en las
que la orientación sexual se confunda con identidad de género. Esta visión
podría estar ligada a la dicotomía que domina el discurso de identidad de género
y sexual. Al un individuo desconocer la existencia de identidades que
transcienden lo comúnmente visibilizado buscará alinearse con aquello que le
permita dar explicación a sus gustos pues de esta manera se permite la
constitución identitaria que evita el malestar de no alinearse a lo socialmente
esperado. Sin embargo, el realizar esto es una solución incompleta, que no
alivia la angustia y la inconformidad, y que deja de lado la expresión propia
del género del individuo por asumir que la orientación sexual puede dar
explicación a la identidad de la persona.

\subsection{Discriminación}\label{ssec:discriminacion}

Esta categoría se presenta como el segundo elemento más relevante dentro de los
relatos expresados por los participantes, esto podría deberse a que el romper
con lo socialmente aceptado o esperado genera una disrupción con otros
individuos de la sociedad con los cuales se interactúa. A causa de esto es
posible que sean vistos como \emph{anormales} o con una identidad falsa y los
lleva a asumir posiciones de defensa de su integridad como individuo,
defendiéndose de comentarios emitidos por miembros de sus familias, compañeros
de clases e instituciones. Estos actores no solo pueden constituir en algunas
ocasiones el principal obstáculo con el que tienen que lidiar las personas
transgénero, también actúan como perjuradores dentro de las actitudes y
prejuicios dirigidos a la población trans.

Como fue explorado anteriormente, la identidad transgénero suele presentarse en
etapas tempranas de la vida expresándose en el malestar que nace al no
identificarse con el género que socialmente se espera se identifique una persona
de un sexo determinado. En los relatos de los participantes se pueden recolectar
distintos elementos que visibilizan las distintas situaciones de rechazo que
vivieron, un ejemplo de esto es el relato del caso~1 que expresa el siguiente
verbatim:

\begin{verbatim}
…en la escuela era algo terrible por el bullying, pero yo siempre imponía
carácter y jamás me deje amedrentar por nada ni nadie.  De hecho, me agarre a
golpes y me expulsaron por 10 días…
\end{verbatim}

Además, dentro del relato del caso~1 se puede recuperar el siguiente verbatim
que muestra las formas en las que puede afectar la discriminación y el rechazo
a la relación familiar y el trato intrafamiliar:

\begin{verbatim}
…mi madre al principio le costó mucho aceptarlo y mi papa le dijo a mis hermano
que porque no me quedaba como yo era que él me aceptaba gay más no vestido de
mujer y jamás me permitió aclararle los conceptos que tenía errado y hasta hoy
no me trata ni mi padre ni mi hermano menor…
\end{verbatim}

Otros verbatims que permiten visibilizar estos hechos de discriminación se
presentan dentro del relato del caso~3 cuando expresa, por ejemplo:

\begin{verbatim}
Siempre era el raro del grupo.
\end{verbatim}

O cuando hace referencia al conflicto que se vivía en su familia cuando hablaba
sobre el malestar que le podía causar tener una identidad discordante:

\begin{verbatim}
En mi casa siempre era un conflicto hablar sobre cómo me sentía.
\end{verbatim}

Partiendo desde estos verbatims se hace posible confirmar que la población
transgénero es muy vulnerable a experimentar situaciones de rechazo, tanto en
ambientes familiares así como en interacciones cotidianas con otros miembros de
la sociedad. Este rechazo parece estar ligado a que la persona transgénero rompe
con las nociones cristalizadas sobre lo que significa desde una mirada social
ser hombre o mujer y cómo se debería ver cada uno. Cuando una persona no
sensibilizada tiene un contacto con el transgenerismo se hace visible la
dominación patriarcal así como la perspectiva hegemónica que domina las
relaciones entre hombres y mujeres.

Esta vivencia del rechazo y sus raíces en la mirada patriarcal del género se
puede identificar dentro del relato de los participantes como se presenta en el
siguiente verbatim del caso~2:

\begin{verbatim}
…yo solía ir a comprar ropa de hombre y las vendedoras me decían que no porque
compraba eso si yo era mujer y me miraban como con asco y yo les respondía que
lo hacía porque me gusta la ropa y porque tengo la plata para hacerlo.
\end{verbatim}

También se hace presente dentro del relato del caso~1:

\begin{verbatim}
…era horrible cuando me tocaba sacarme la cedula y me obligaban a ir vestida
como hombre, el personal que me atendía se ponía muy hostil conmigo cuando
llegaba maquillada.
\end{verbatim}

Estos verbatims refuerzan lo planteado anteriormente, la visión de género y su
expresión que se maneja comúnmente dentro de la sociedad está muy permeada por
la visión dicotómica de la misma. Esta visión es resultado de la influencia
del patriarcado sobre la interacción y desarrollo de los miembros de la
sociedad. Consecuentemente esta visión influye en el desarrollo como persona de
aquellos miembros de la sociedad que son transgénero y les afecta pues tienen
que enfrentar estas situaciones de rechazo que los lleva a desarrollar
estrategias para poder enfrentarse a estos problemas.

Sumado a esto la vivencia de la discriminación se expresa en las experiencias de
acoso o bullying las cuales aparecen dentro de los verbatims de los
participantes, un ejemplo de esto es lo expresado por el caso~1:

\begin{verbatim}
En la escuela era algo terrible por el bullying, pero yo siempre imponía
carácter y jamás me deje amedrentar por nada ni nadie.  De hecho, me agarre a
golpes y me expulsaron por 10 días.
\end{verbatim}

También se hace visible en relato del caso~2:

\begin{verbatim}
Cuando inicié mi transición los compañeros de trabajo de mi papá se metían
conmigo, hasta que un día les ofrecí unos golpes y todo cambió.
\end{verbatim}

Parece que las situaciones de acoso o bullying que viven las personas
transgénero pueden surgir a consecuencia de los procesos discriminantes que
viven, es el rechazo llevado a un acto en concreto, no necesariamente ligado a
violencia física, también puede darse en casos de violencia simbólica como se
presenta dentro del relato del caso~2:

\begin{verbatim}
…mientras estudiaba para sacar mi título de sexólogo me pasó que una docente era
particularmente agresiva conmigo, porque una vez le respondí feo, la razón de
esto fue porque ella estaba cuestionando mi identidad. Pero nada al final le
saqué un 20 y le callé la boca.
\end{verbatim}

Esto le da más peso a la susceptibilidad de la población transgénero a vivir
situaciones aversivas y que atentan contra su integridad física y mental. Los
participantes demuestran de sus relatos el uso de estrategias para poder
combatir estas situaciones, las cuales parecen generarse de acuerdo al acto que
los enfrenta. Es decir, si la situación de acoso o bullying se presenta como
dirigida a atentar contra la integridad física del individuo la respuesta suele
ser responder de una manera igualmente agresiva, por otro lado, si el acto
violento se da en ambientes académicos, la respuesta que da la persona afectada
va a ser a nivel simbólico.

El tener que recurrir a estas herramientas de defensa puede llegar a pesar en la
constitución de identidad del individuo transgénero. La defensa continua de la
integridad como individuo, así como la exigencia de tener que validarse en los
campos donde se ve inmersa la persona transgénero afecta su desenvolvimiento
como miembro de una comunidad. Puede también llegar a tener consecuencias
graves, no solo por tener que estar en un estado constante de validación y
defensa, sino porque un agresor podría atentar contra la vida de la persona
transgénero. También puede esto causar estrés, ansiedad y angustia a niveles que
pueden colocar a la persona transgénero en riesgo de autolesión o suicidio.

Esto se presenta en el relato de los casos de distintas formas, en el caso del
relato del caso~3 se presenta de la siguiente manera:

\begin{verbatim}
Estando en Ecuador supe de un trans al que violaron y mataron y la verdad me dio
miedo.
\end{verbatim}

En el relato del Caso 2 se puede recuperar el siguiente verbatim:

\begin{verbatim}
Hace poco supe de un caso de una chica trans, ella aun no comenzaba con su
tratamiento hormonal pero ya estaba expresando una identidad de género con la
que se sentía cómoda. La cosa es que por ser trans su familia la botó de la casa
y terminó en situación de calle, prostituyéndose para sobrevivir hasta que la
mataron hace una semana, la encontraron en un monte.
\end{verbatim}

Y por último, en el relato del caso~1, se visibiliza por medio del siguiente
verbatim:

\begin{verbatim}
…aquí nos llegan muchos casos de personas trans que se quedan sin casa o
trabajo por querer ser felices, principalmente pasa con chicas, algunas se
terminan suicidando.
\end{verbatim}

Partiendo de estos verbatims se puede reforzar la idea de que la población
transgénero se encuentra expuesta a actos que atentan contra su integridad
física, así como con su derecho a la vida. Como fue planteado anteriormente los
participantes desarrollan herramientas que les permiten implementar estrategias
de defensa de su identidad y validar ante los otros, pero cuando una persona
transgénero no cuenta con estas herramientas y se enfrenta a situaciones
aversivas es posible que el resultado de este enfrentamiento tenga graves
consecuencias.

En definitiva, podría afirmarse que los obstáculos que tienen que ser sorteados
por las personas transgénero pueden influir en la constitución de sus
identidades, pues deben lograr superar las valoraciones negativas y actos contra
su integridad para poder expresarse de acuerdo al género con el que se
identifican. La obstrucción de esta posibilidad podría traer como consecuencias
dudas dentro del individuo, así como un malestar que va a encontrarse siempre
presente hasta que esta incongruencia de género sea solucionada de alguna
manera, esto se puede visibilizar dentro del relato del caso~3 con el siguiente
verbatim:

\begin{verbatim}
Descubrí que tenía que hablar de esto con alguien o me iba a volver loco
\end{verbatim}

Es por esto que es de vital importancia poder visibilizar más la identidad
transgénero y ofrecer apoyo no solo a aquellos que la viven, sino a la población
en general. Para poder reducir estas situaciones aversivas que pueden vivir las
personas transgénero.



\subsection{La Transición}\label{ssec:transicion}

El concepto de la transición para los participantes ocupa un papel central pero
utilitario dentro de su conformación personal. La transición parece ser
clasificada en función del nivel de intervención que requiere de parte del
participante. Así, la transición comienza para los participantes cuando asumen
una conciencia de ser transgénero o transexual y luego deciden presentarse
socialmente en función de su identidad de género y no al género que les fue
asignado.

Estos usos de la transición varían en función de la dirección de la transición,
ya sea de hombre a mujer (HaM) o de mujer a hombre (MaH). Pero, aunque los
detalles de implementación sean distintos, se pueden agrupar en las mismas
etapas o estadios.

El primer paso utilizado es el cambio de la estética y la ropa. En esta etapa la
persona transgénero hace uso y aprovecha cualquier ventaja que pueda permitirle
señalar socialmente su identidad de género. Por ejemplo, en el caso~1 (HaM)
utilizar un tono de voz más agudo y resaltar los rasgos faciales más femeninos
aplicando maquillaje. Incluye esto también el uso de toallas sanitarias. Si bien
una mujer transgénero no tiene menstruación ni sangrado vaginal, incluir la
rutina mensual de la adquisición y uso de productos sanitarios femeninos
potencia la percepción subjetiva de pertenecer al género deseado.

En los casos~2~y~3 (MaH) utilizar ropa más holgada en el torso para ocultar más
fácilmente los senos y utilizar cortes de cabello masculinos. Hablar con voz más
gruesa y realizar entrenamiento con pesos en el gimnasio para obtener una mayor
masa corporal, una espalda más ancha, además de participar de los rituales deportivos masculinos.

Sin embargo, todos los casos reportan el interés y uso eventual de la terapia de
sustitución hormonal. Esta se puede considerar otra etapa de la transición en la
cual se toman las hormonas responsables de los rasgos sexuales secundarios del
sexo con el cual está asociado el género deseado. Testosterona en el caso de los
hombres trans y estrógeno en el caso de las mujeres trans, en ambos casos son
componentes sintéticos elaborados artificialmente. Sin embargo, como
consecuencia de la dificultad para su acceso, ya que sólo se pueden obtener
mediante récipe médico de un endocrinólogo o en el mercado negro. El segundo
caso está expuesto a fraudes, estafas, abusos y complicaciones médicas por
fallas en la composición. Esto debido a que las hormonas que se obtienen en los
mercados negros suelen estar destinadas al uso veterinario.

El objetivo de esta terapia hormonal es potenciar las características
secundarias de la expresión de género deseada y disminuir aquellas del género
asignado al nacer. El valor asignado a la terapia hormonal está en la percepción
de efectos inmediatos. Estos incluyen el crecimiento de vello facial o su
inhibición, engrosamiento o afinamiento de la voz y, en el caso de los hombres
trans, la interrupción del ciclo menstrual.

Estos cambios son de mucha importancia para los participantes pues permiten
pasar desapercibidos en espacios públicos siendo reconocidos como su género
deseado. Al coincidir las expresiones con las características de su identidad de
género se disminuye la angustia y el miedo a ser señalado como inadecuado, raro
o fuera de lugar.

Esto se puede extender mediante los efectos a largo plazo de la terapia
hormonal. Esos cambios a largo plazo son la redistribución de la grasa corporal,
alteración del grosor de ciertas estructuras óseas, como caderas y mandíbula, y
estimulación del crecimiento muscular en el caso de los hombres trans.

En conjunto, todos los cambios son altamente valorados por los participantes, al
punto de que tener que dejar de tomar hormonas es lamentado como una pérdida de
un elemento fundamental. Como en el caso~2, para quien la toma de hormonas es
tan importante que se convirtió en una causa para su toma de decisiones respecto
a emigrar del país. Prefiere tener un acceso regular y continuo de su
tratamiento hormonal que permanecer en su país de nacimiento.

La intervención quirúrgica es una etapa concebida como ubicada al final en el
proceso de transición por los participantes. Sin embargo, las posibles cirugías
son variadas y no se limitan a la cirugía de reasignación de sexo. La
mastectomía, las mamoplastias de aumento y de reducción, las cirugías plásticas
de feminización o de masculinización del rostro, lipoescultura, entre otras
opciones de cirugía plástica habilitan a la persona transgénero para alterar su
aspecto físico.

De todas ellas, los participantes valoran como la más importante aquellas que
modifican los senos incluso por encima de la reconstrucción genital. Podemos
interpretar que se debe a la fuerte asociación que existe entre la feminidad y
los senos como una de las características sexuales secundarias más visible
socialmente. El seno tiene un significado que está ligado a cualidades
estereotípicamente femeninas. La sensualidad, la tentación, la maternidad y la
nutrición son todas expresiones simbolizadas por el seno en la cultura
occidental, entre otras, debido a su asociación tradicional con las
características asignadas a las mujeres. Roles hegemónicos de género como el
cuidado, la maternidad, la tentación del sexo reproductivo, la protección
maternal, son representados en las funciones biológicas, así como las
asociaciones culturales en las que el seno hace de mediador.

Los participantes también consideran esta cirugía con mayor importancia y
frecuencia que la de reasignación genital debido a que es percibida como más
sencilla en su procedimiento y más accesible dentro de los servicios de salud.
Además, generalmente se le considera en la medicina como un procedimiento
cosmético\footnote{Sólo la mastectomía tiene una función terapéutica en el
tratamiento de tumores en el cáncer de mama. Y en estos casos se trata de un
procedimiento distinto a la mastectomía cosmética o de reducción que se realizan
los hombres trans.}, las cirugías de mamoplastia tienen menos requisitos en
cuanto a evaluación psicológica. Esto hace más fácil acceder a la operación en
el contexto de un cambio de género.

Todo esto hace a los participantes considerar la transición como un proceso
continuo. Con un inicio pero que realmente no se detiene, aunque la cirugía
pueda ser un paso \emph{final}. Ninguno de los casos plantea un final o meta
dentro de la transición. Para ellos, vivir con los procesos de transición, como
la toma constante de hormonas, es un hecho intrínseco y natural de su vida. No
se percibe como una etapa transitoria ni incidental. Sino que es constitutiva de
su identidad, su experiencia y su rutina diaria, al punto que perder o abandonar
las rutinas que les permiten mantener las características biológicas se percibe
como impensable o intolerable.

Los tres casos coinciden al considerar que desde el primer momento en que
deciden presentarse como alguien del género opuesto al sexo con el cual nacieron
han ingresado a la transición. A partir de ese momento su meta constante será
vivir lo más cercano posible a su género deseado y ser reconocidos como tal.
Esta búsqueda nunca se va a detener y por ello nunca habrá un cese en la
búsqueda de estrategias, métodos, alternativas que permitan esta vivencia del
género sentido.

Para acceder a estos métodos primero debe haber una consciencia de la identidad
de género. Este suele ser el primer obstáculo pues, según lo observado en los tres casos, un
desconocimiento de la transexualidad y el transgenerismo durante su
adolescencia. Es el contacto, muchas veces accidental, con los conceptos lo que
suele disparar su proceso de autoconcientizar su propia identidad. Sin este
primer momento de ‘¿Qué tal si…?’ no suele desarrollarse la necesidad de
comenzar una transición para expresarse con su género deseado.

En este proceso pueden intervenir los medios de comunicación, personas
significativas del entorno social y familiar, e incluso profesionales de salud y
de la psicología. Los participantes coinciden en que esto es lo que les permite
las herramientas para transformar un malestar emocional en un deseo de cambio
racionalmente expresado y en una identidad clara de pertenecer a un género.

Al respecto el caso~2 expresa que la disforia de género será un tema a debatir
como diagnóstico de la transexualidad y transgenerismo pues por primera vez en
la historia hay muchos adolescentes trans que crecen con una identidad clara en
cuanto a estos contenidos. Él comenta sobre el caso del hijo de la cantante
Karina, quien toma supresores hormonales:

\begin{verbatim}
Imagínate el hijo de Karina. Ese muchacho nunca va saber lo que es una
menstruación, porque lo que él toma algo para no desarrollarse. Él nunca va a
saber lo que es todos los meses desear que te trague la tierra porque eso no es
lo que te debería estar pasando, a los hombres no les viene la regla.
\end{verbatim}

Esta idea apunta a que un conocimiento temprano de los conceptos de identidad de
género puede empoderar a los jóvenes trans y disminuir significativamente el
malestar iniciando más temprano la transición y evitando experiencias
angustiantes. También nos hace visible la posibilidad real de que la
construcción identitaria de una joven trans empoderada en su transición desde
una temprana edad será diferente a la de aquellas personas trans que crecen en
completo desconocimiento de los conceptos de identidad de género y que no
cristalizan una identidad de género clara hasta un momento posterior de la vida.
Como por ejemplo, en el caso~1 quien no decidió iniciar su transición hasta la
edad de los 40 años por diversos miedos y angustias propios de la discriminación
en contra de la condición trans.

\subsection{Cuerpo y genitalidad}

El cuerpo y la genitalidad han sido el centro desde la visión biologicista
hegemónica de la identidad de género. Especialmente la configuración fenotípica
de los genitales. Hasta el punto que ha sido la configuración de los genitales,
y el deseo de su transformación, lo que ha pasado a definir desde ciertos
discursos la diferencia entre una persona transgénero y una persona transexual
(Noseda, 2012). Sin embargo, lo reportado por los participantes da a entender
que existe una mayor complejidad en el rol del cuerpo en la identidad de lo que
estos discursos podrían sugerir.

En estrecha relación con lo planteado en la categoría ‘La Transición’, lo que es
considerado como esencial por los tres casos es la capacidad comunicativa del
cuerpo por encima de lo funcional. Como se explicó en la sección anterior se le
asigna un valor importante a las mamas y a todas las cualidades simbólicas que
representa socialmente. Esta característica habla del valor que tiene el cuerpo
en la constitución identitaria al ser el principal vehículo de expresión social.

Además, la constitución corporal señaliza pertenencia de grupo. Para poder
pertenecer al grupo del género deseado hay que poseer las señas y marcas que
están socialmente sancionadas como las que le pertenecen a ese grupo social. Por
ello la insistencia de parte de los participantes en el uso de señales
estereotípicas del género que desean expresar. En particular aquellas que
facilitan la participación en los rituales de socialización. Por ejemplo, el uso
de maquillaje en el caso~1, el ejercicio en el caso~2, y las salidas sociales
que involucran la estética, el cuerpo y la interacción social.

Si bien existe una expresión de lo genital como objetivo final u objeto de deseo
definitivo. Por ejemplo, el caso~3 expresa:

\begin{verbatim}
Si me pusieran en una bandeja a elegir, así mágicamente, entre poseer una vagina
y un pene, seleccionaría tener pene. Pero la realidad no funciona así.
\end{verbatim}

Existe una aceptación y adaptación a la realidad. Desde allí se vive el ser
transgénero como una condición que se tiene. Por ello los tres casos hacen una
fuerte insistencia en complementar o suplementar su expresión de género mediante
los aspectos sexuales secundarios.

Este énfasis en los caracteres sexuales secundarios advierte, además, de la forma
en la que se relacionan la identidad de género, la expresión y la orientación
sexual. En un sentido estricto, ninguno de los tres casos se deja de considerar
como miembro del género con el cual se identifica por el hecho de no tener a la
perfección la totalidad de las características estereotípicas hegemónicas. De
hecho, el haber nacido con un cuerpo que es asignado a un sexo con el cual no se
identifican no es una limitación para nuestros casos para sentirse participes
del género con el cual se identifican.

El deseo de ajustar el cuerpo a la identidad tiene que ver con alinear la
expresión corporal a la expresión deseada. La identidad ya está clara para los
participantes, la intención es ajustar la expresión del cuerpo a lo que se desea
proyectar. En este sentido, la persona transgénero no es diferente a una persona
cisgénero que explora su expresión para ajustarla a su propio deseo.

 Cualquier insatisfacción que queda posterior a alcanzar algún nivel de
 transición está asociado a la autopercepción y no a la participación social.
 Esta autopercepción se concentra en la asociación genitales-sexo-género propia
 de la concepción tradicional del sexo y del género. Sin embargo, cuando son
 consultados, los tres casos coinciden en ser capaces de adaptarse a la
 genitalidad con la que nacieron. Tanto en la vivencia de la intimidad sexual
 como en el día a día.

 Producto de la concepción tradicional existe la inconformidad y la
 disatisfacción, pero en la vida diaria se vive y se construye individualmente
 en función de la genitalidad que se posee. En la genitalidad el énfasis que
 realizan los participantes es en la ausencia de una cualidad del género propio
 y no en la cualidad del género asignado al nacer que sobra. Por ejemplo, para
 los casos MaH, el deseo es la posesión de un pene y testículos como genitales
 propios. La insatisfacción se vive como producto de una falta. La presencia de
 una vagina es incidental, una condición con la cual se nació y con la que se ha
 de lidiar.

 En este sentido la persona transgénero podría tener cualquier otra
 configuración genital intermedia o ambigua. Aún existiría la percepción de
 falta del componente genital deseado. La realidad es que el deseo de los
 participante sería poseer un cuerpo cisgénero en todos los sentidos, aunque
 perciban que por su condición no podrán alcanzar este objetivo.

\subsection{El Otro}

Una buena parte de la identidad de una persona es determinada por las relaciones
interpersonales que se establecen con las otras personas con quienes se
interactúa. La identidad personal es condicionada en buena parte en función del
lugar que se ocupa en una sociedad y la forma en la que somos percibidos por los
otros. Además, es la sociedad la que hace entrega a los individuos de las
concepciones sobre el género que pueden posteriormente moldear la forma en la
que el individuo construye su propia noción del género. Los roles de género y
las formas en las que la identidad de género se puede expresar o no en cada
cultura son transmitidas por la relación con otras personas.

En este sentido, los tres casos coinciden en una percepción inicialmente
negativa de la relación con el otro, marcada principalmente por la
discriminación y el rechazo. Como ya se exploró en la sección
~\ref{ssec:discriminacion} sobre la categoría ‘discriminación’, estas
experiencias negativas tienen un impacto importante en el proceso de
construcción de la identidad y también en la forma en la que la persona trans se
relaciona con los demás.

Los participantes conciben su relación con el otro anónimo como aprehensiva en
principio. Esto debido a que la condición trans es percibida como transgresora
del orden tradicional del género. La cautela y la aprehensión inicial se deben a
que, ser identificados como trans, es potencialmente ser expuesto bajo una
mirada negativa. Esto les expone a una diferenciación inmediata que puede ser
acompañada de un cambio en el trato, o incluso una completa negación de la
interacción.

La estrategia que asumen para combatir esta aprehensión es una expresión
hiper-estereotipada del género con el cual se identifican. Esto los empuja a
tipos de expresión que sobrepasan incluso las expectativas impuestas sobre las
personas cisgénero. Se hace obvio entonces el énfasis puesto por la persona
transgénero en los caracteres sexuales secundarios en su cuerpo y a su vez la
forma en la que adaptan conductas y actitudes corporales propias del estereotipo
hegemónico de género. Es un esfuerzo consciente y deliberado de prevenir ser
expuesto y rechazado como miembro del género con el que se identifican.

Esto incluye también todo un repertorio conductual y actitudinal, las formas de
caminar, hablar y gesticular son adaptadas a la expresión del genero
identificado. Este repertorio es desplegado como forma de autoafirmación frente
al otro que puede conocer de su condición trans o no.

En el ámbito personal y la intimidad el repertorio se mantiene pues es además
una herramienta de relación que reafirma la identidad personal incluso cuando el
otro acepta la condición trans. De manera que no sólo se trata de una estrategia
de evitación de la discriminación, sino que es una forma de reafirmación
subjetiva. En este caso se convierte en una expresión óntica del ser ellos
mismos, en este caso sin la aprehensión y cautela sino como una expresión de
\emph{quien realmente soy}. Este es un cambio en el sentido de la conducta pero
que mantiene y refuerza la forma de relación.

En todos los ámbitos los participantes buscan activamente la forma en la cual
pueden realizar estas reafirmaciones de la identidad frente al otro. Esto va
desde micro interacciones con extraños y puede llegar hasta la esfera del
reconocimiento legal. El poder expresarse y ser reconocidos como su género
deseado en tantos espacios como sea posible es un deseo que motiva a los
participantes.

Es importante también que la reafirmación de la identidad frente a los otros
tenga impacto real sobre la realidad. Que no sea sólo un gesto simbólico, aun
cuando estos se aprecian fuertemente. Es parte de la motivación a la lucha por
una identidad legal. La misma que, además, fungiría posteriormente como
herramienta para la demanda de derechos que les permitiría acceder a otras
instancias de relacionamiento social con un renovado apoyo, como las instancias
laborales, educativas y de servicios.
