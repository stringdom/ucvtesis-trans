% 15 de mayo de 2018
\chapter{Presentación y discusión de la información}\label{ch:informacion}

El proceso de entrevistas fue realizado entre los meses de abril y septiembre
del año 2017.
Estas entrevistas fueron transcritas para su análisis en el texto.
Los elementos expresados por los participantes en las entrevistas fueron
codificados y organizados en las categorías y subcategorías que se pueden
observar en la figura \ref{fig:categorias}.
Las entrevistas fueron codificadas empleando el programa ATLAS.TI en su versión
7.5.7 proceso que facilitó la categorización de la información recolectada a lo
largo de las entrevistas.

\begin{figure}
    \centering
    \includegraphics[width=0.75\textwidth]{categorias}
    \caption{Diagrama de categorías}
    \label{fig:categorias}
\end{figure}

A continuación realizaremos una descripción de cada una de ellas junto con los
verbatim que les dan origen.
Adicionalmente presentamos el razonamiento e interpretación que damos a cada
categoría y sus implicaciones caso a caso para el cumplimiento de los objetivos
de investigación.
El orden de presentación de las mismas fue elegido
según la frecuencia de manifestación en las entrevistas realizadas.

En la sección \ref{sec:discusion}, realizamos una discusión y análisis punto a
punto de cada una de las categorías.

\section{Presentación de la información}

\subsection{Desarrollo}

\begin{figure}
    \centering
    \includegraphics[width=0.75\textwidth]{desarrollo}
    \caption{Diagrama desarrollo y familia}
    \label{fig:desarrollo}
\end{figure}

\subsubsection{Infancia y pubertad}

\subsubsection{Familia}

\subsection{Género}


\section{Discusión}\label{sec:discusion}

\subsection{Desarrollo}

\subsubsection{Infancia y pubertad}

\subsubsection{Familia}
