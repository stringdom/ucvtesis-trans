% 15 de mayo de 2018
\chapter{Presentación y discusión de la información}\label{ch:informacion}

El proceso de entrevistas fue realizado entre los meses de abril y septiembre
del año 2017.
Estas entrevistas fueron transcritas para su análisis en el texto.
Los elementos expresados por los participantes en las entrevistas fueron
codificados y organizados en las categorías y subcategorías que se pueden
observar en la figura~\ref{fig:categorias}.
Las entrevistas fueron codificadas empleando el programa ATLAS.TI en su versión
7.5.7 proceso que facilitó la categorización de la información recolectada a lo
largo de las entrevistas.

\begin{figure}
    \centering
    \includegraphics[width=0.75\textwidth]{categorias}
    \caption{Diagrama de categorías}\label{fig:categorias}
\end{figure}

A continuación realizaremos una descripción de cada una de ellas junto con los
verbatim que les dan origen.
Adicionalmente presentamos el razonamiento e interpretación que damos a cada
categoría y sus implicaciones caso a caso para el cumplimiento de los objetivos
de investigación.
El orden de presentación de las mismas fue elegido
según la frecuencia de manifestación en las entrevistas realizadas.

En la sección~\ref{sec:discusion}, realizamos una discusión y análisis punto a
punto de cada una de las categorías.

\section{Presentación de la información}

Como se puede ver resumido en la figura~\ref{fig:categorias}, hemos elaborado
seis (6) categorías. Dentro de estas categorías ‘La transición’ es uno de los
componentes identificados como constitutivo del proceso de construcción de la
identidad de las personas trans. Procederemos a explorar el contenido de cada
una de las categorías identificadas.

\subsection{Desarrollo}
Esta categoría se encuentra compuesta por elementos que abarcan desde etapas
tempranas de la niñez y del desarrollo.
Elementos como la relación del individuo con su escolaridad y compañeros de
clases.
También elementos de la constitución de los géneros que se hacen presentes
dentro de la pubertad y los conflictos que estos puedan causar a los
participantes
También incluye el papel que juega la familia dentro de la constitución de la
identidad así como la forma de aproximarse a los problemas.
Todos estos son elementos que pueden marcar la construcción de identidad de una
persona.

\begin{figure}
    \centering
    \includegraphics[width=0.75\textwidth]{desarrollo}
    \caption{Diagrama desarrollo y familia}\label{fig:desarrollo}
\end{figure}

En esta categoría confluyen elementos de conflicto así como las experiencias que
permitieron el manejo de los mismos en los participantes y que
consecuentemente se transformaron en estrategias de afrontamiento. Se debe tomar
en cuenta que en esta categoría se presentan los distintos tipos de relación
establecidas tanto en niveles académicos como familiares entre la infancia y
pubertad de los participantes.

Existen dos subcategorías dentro de este factor. Una de ‘infancia y pubertad’,
que reporta sobre las experiencias de desarrollo y formativas en la juventud, y
otra de ‘familia’. Esta habla acerca de la influencia de los lazos y ambiente
familiar durante el desarrollo.

\subsubsection{Infancia y pubertad}
Dentro de esta subcategoría se pueden
encontrar elementos que parecen ser transversales a lo largo de la vida del
individuo. Al ser tanto la infancia como la pubertad etapas tempranas de
desarrollo y constitución se pudo identificar elementos que parecen sentar las
bases para la forma otros elementos que le permiten al individuo constituirse
como persona. Entre estos elementos cabe resaltar la presencia de conflictos
como lo expresa el Sujeto 1 (verbatim):

\begin{verbatim}
…bueno me identificaba como niño pero me gustaba todo las cosas
de niña. De hecho siempre pedía al niño Jesús cosas de hembra, por ejemplo,
barbie, oso etcétera,  por su puesto jamás me traía lo que pedía y era muy
triste para mí, una era inocente y el niño Jesús me dejaba una carta explicando
que eso eran cosas de niña y me traía patineta bicicleta carrito y a mí no me
gustaba…
\end{verbatim}

Este elemento puede estar ligado al rechazo que pueden vivir, según lo expresado
por Sujeto 1 (verbatim):

\begin{verbatim}
…en la escuela era algo terrible por el bullying, pero yo
siempre imponía carácter y jamás me deje amedrentar por nada ni nadie.  De hecho
me agarre a golpes y me expulsaron por 10 días…
\end{verbatim}

O a tener su raíz en conflictos por la constitución de su identidad como lo
expresan con los siguientes verbatims.

Sujeto 1:
\begin{verbatim}
…yo lloraba porque no me entendía y me sentía mal.
\end{verbatim}

Sujeto 2:

\begin{verbatim}
  …me criticaban mucho como me vestía, pero es lo que me gusta.
\end{verbatim}

Y sujeto 3:

\begin{verbatim}
…siempre era el raro del grupo…
\end{verbatim}

La aparición de estos conflictos entra en contacto con las aproximaciones a los
roles de género que suceden por primera vez en la infancia por medio de los
juegos y que perduran a lo largo de la vida de un individuo, como se evidencia
en el relato de Sujeto 1:

\begin{verbatim}
…a mí me decían que tenía que jugar con carritos y no con
muñecas porque eso es de niñas y yo no era una…
\end{verbatim}

Y del Sujeto 3:

\begin{verbatim}
…me gustaba tener el cabello corto y no me arreglaba pero mis
padres me decían que tenía que arreglarme para poder verme linda…
\end{verbatim}

 Un elemento que se ve relacionado con la presencia del conflicto de identidad
 es que la aprehensión del mismo lleva al individuo a buscar o considerar el
 daño que puede causar el no lidiar con esta situación y es por eso que hay
 comentarios como el emitido por el Sujeto 3 quien expresa:

 \begin{verbatim}
Descubrí que tenía que hablar de esto con alguien o me iba a volver loco.
 \end{verbatim}

 Este verbatim, trae a la luz el malestar que nace en el individuo por este
 conflicto de identidad, posiblemente el hecho de estar en una situación de la
 cual se tiene poca o ninguna información al alcance del individuo hace que el
 malestar por el conflicto sea mayor y pueda tener consecuencias mas peligrosas
 para la identidad de la vida del individuo.

\subsubsection{Familia}

Otro de los componentes de la categoría ‘Desarrollo’ es la subcategoría
denominada \emph{Familia}, pues según  lo expresado por los individuos como
Sujeto 3:

\begin{verbatim}
…en mi casa siempre era un conflicto hablar sobre cómo me sentía.
\end{verbatim}

Sujeto 2:

\begin{verbatim}
Mi mamá me decía, ‘hija arréglate un poco’ ó ‘te verías muy linda
con vestido’ y yo le decía que no me gustaba eso y venia el regaño.
\end{verbatim}

Esta subcategoría influye en la constitución del individuo y modela su
desarrollo, en aspectos como la forma de establecer relaciones,
tomando lo expresado por Sujeto 2:

\begin{verbatim}
…la relación con mis padres es distinta, a mi madre le costó más
aceptarme, a mi padre después de explicarle lo entendió con más facilidad,
incluso suelo pasar por su trabajo, es moto taxista.
\end{verbatim}

Un elemento que resaltan los participantes es el cambio de dinámicas dentro de
la familia que se da producto de asumirse como persona trans. Como se observa en
el relato de Sujeto 1:

\begin{verbatim}
…mi madre al principio le costó mucho aceptarlo y mi papa le
dijo a mis hermano que porque no me quedaba como yo. Era que él me aceptaba gay
mas no vestido de mujer y jamás me permitió aclararle los conceptos que tenía
errado y hasta hoy no me trata ni mi padre ni mi hermano menor…
\end{verbatim}

Estos verbatims asoman que dentro de las familias de los individuos transgéneros
existe una innegable presencia de conflicto generado por la nueva identidad
asumida por el mismo. El cómo esta identidad interactúa con las prenociones de
la familia suma a la conflictividad durante el desarrollo. Es posible que esta
situación este relacionada con el propio conflicto interno que pueden vivir
personas transgénero como consecuencia de la falta de acceso a información que
permita aclarar sus dudas.

Esto podría ser un indicador de las percepciones hegemónicas sobre la sexualidad
y como estas moldean las diversas interacciones entre los individuos. Se
observa en el verbatim del Sujeto 1 que por parte de su padre existe aún una
confusion entre lo que es la orientación (los gustos del individuo) y su
identidad (como el individuo se observa a sí mismo) sexual.

\subsection{Género y sexualidad}

La siguiente categoría que permite la constitución de identidad en los
participantes es la de \emph{género y sexualidad}. En esta categoría se
presentan elementos como la constitución de la identidad que tienen su raíz en
etapas tempranas de la vida, con base a lo expresado por los participantes se
generaron las siguientes subategorias: Identidad de género, Hegemonía de género
y Orientación Sexual.

En la figura~\ref{fig:genero} se puede observar los elementos constitutivos de
la categoría.

\begin{figure}
    \centering
    \includegraphics[width=0.75\textwidth]{genero}
    \caption{Diagrama de categorías}\label{fig:genero}
\end{figure}

\subsubsection{Identidad de género}

Esta subcategoría se mostró como la más importante dentro de lo expresado por
los participantes principalmente porque ha sido el componente central con
el que se enfrentaron en la constitución de su identidad. Esto se puede ver
reflejado en lo expresado por Sujeto 2:

\begin{verbatim}
…el momento más feliz de mi infancia es cuando me pude vestir
de Ángel Gabriel en un nacimiento viviente que hicieron en mi escuela. Mi mamá
no quería pero yo me sentía en las nubes porque por un momentito pude verme como
quería.
\end{verbatim}

Este conflicto de identidad también se hace visible dentro del relato de Sujeto
3 cuando expresa:

\begin{verbatim}
…no es que solo me gustaba vestirme como hombre, es que pensaba
como uno también.
\end{verbatim}

Así como dentro del relato de Sujeto 1 cuando comenta:

\begin{verbatim}
Cuando era pequeña no me molestaba, pero cuando empecé a
desarrollarme le rezaba a Dios para que por favor me salieran senos.
\end{verbatim}

Estos verbatims permiten identificar algunos aspectos relacionados con el
conflicto de identidad de los individuos. Elementos como el verbatim expresado
por el Sujeto 2 que considera un recuerdo muy feliz el poder mostrar una
expresión de género acorde a la identidad que, ya para esa etapa de su
desarrollo, sentía era la de él. Además esto se puede reforzar con el verbatim
del Sujeto 1 cuando expresa que le causaba preocupación y ansiedad el hecho que
sus senos no se desarrollaran junto con el resto de su cuerpo.

Además se presenta la dicotomía entre lo que las personas sienten que son, las
expectativas de como esperan verse y la imposición de la identidad que se
les asigno por haber nacido con un sexo especifico.

\subsubsection{Hegemonía de género}

\subsubsection{Orientación sexual}

\subsection{Discriminación}

\subsubsection{Rechazo}

\subsubsection{Acoso / Bullying}

\subsubsection{Consecuencias}

\subsection{La transición}

\subsubsection{Transitar}

\subsubsection{Apariencia física}

\subsubsection{Conocimiento}

\subsection{Genitalidad}

\subsubsection{Lo esencial}

\subsubsection{Lo que se tiene}

\subsection{El Otro}

\subsubsection{Cómo me ven}

\subsubsection{Aspecto legal}

\section{Discusión}\label{sec:discusion}

\subsection{Desarrollo}

\subsubsection{Infancia y pubertad}

\subsubsection{Familia}
