\chapter{Marco metodológico}
	Para cumplir con los objetivos planteados para esta investigación se propone
aplicar los principios metodológicos de la teoría fundamentada en el contexto
de un paradigma de investigación cualitativa. Para ello se utilizará como
principal actividad de construcción de información una serie de entrevistas a
profundidad. Estas entrevistas semi-estructuradas serán realizadas a personas
transexuales o transgénero, de ambos géneros, que vivan en la ciudad de
Caracas. Como parte de los propios principios y lineamientos de la teoría
fundamentada se mantendrá abierta la posibilidad al uso de otras fuentes de
información como estadísticas, encuestas y artículos documentales. Esto
dependiendo del transcurso de la investigación pero manteniendo la entrevista
como la principal fuente de información.

	Se plantea el uso de la investigación cualitativa en concordancia con la
alineación entre las ventajas que brinda la perspectiva paradigmática con los
objetivos del problema de investigación. Siendo el propósito de esta propuesta
de investigación llegar a una comprensión del fenómeno y vivencia de la
transexualidad, subyace al mismo una perspectiva ontológica que plantea al ser
humano como un ser cognoscente, autor de su propia realidad y constructor del
sentido que esta adquiere para su experiencia. Esta experiencia, concebida como
individual y propia de aquellos que la viven subjetivamente, es por lo tanto
única en su expresión y no conlleva una pretensión de universalidad
en las conclusiones que se originarán de su análisis. Debido a esto parto de
las concepciones de la reflexión epistemológica como se puede apreciar en los
escritos de \Textcite{Gialdino2009}.

	En su obra, la epistemología del sujeto cognoscente plantea al individuo como
un agente activo de sus propios procesos de vida y además consciente y poseedor
de recursos para la elaboración de reflexiones acerca de sí mismo y su
ubicación en el espacio social. De allí que los métodos y técnicas propuestos
aquí se fundamenten en la capacidad de los informantes para actuar como
co-constructores del conocimiento y comprensión que deriven de esta
investigación.

	Además expone que la investigación cualitativa se interesa particularmente por
las formas en las que el mundo es comprendido y producido, así como por el
contexto y el cambio social \parencite[Mason 1996, c.p.]{Gialdino2009}. Agrega
que, como método, la investigación cualitativa es interpretativa y reflexiva.
De manera que los objetivos y fines de la investigación cualitativa están
orientados a la obtención de una comprensión reflexiva de la realidad objetivo.

\todo{Incluir algunos elementos onto-epistemo-metodologico de guba y lincoln}

	En este sentido se justifica el uso de la metodología de la teoría fundamentada.
Esta es una tradición de investigación cualitativa que permite realizar un
análisis de los elementos simbólicos y de comprensión que subyacen a la
información que se construye mediante las técnicas seleccionadas por el
investigador \parencite{Glaser1967}.

\section{Informantes clave}

	Como informantes clave para la construcción de la
información de interés para esta investigación se han seleccionado a las
personas transexuales de Caracas, independientemente de la dirección de la
identidad asumida.

	Como se expresó en el marco referencial, se comprende una diferencia entre la
identidad transgénero y transexual. Además existe una diferenciación binaria
entre la expresión y vivencia de los géneros masculino y femenino. Es
pertinente, entonces, considerar que las vivencias entre la transexualidad de
femenino a masculino y la transexualidad de masculino a femenino podrían
acarrear experiencias, sentidos y expresiones diferenciadas en su forma de
comprensión de la identidad y de la transición de un sexo a otro. Como plantea
\textcite[][p. 218]{BergeroMiguel2008}, se ha encontrado que las ideas del
género de las
personas transexuales están enmarcadas en una concepción rígida y excluyente.
Por ello se espera que las mismas se vean expresadas en una diferencia de algún
tipo entre las vivencias de una y otra dirección de la transición de sexo.

	Respondiendo a estas expectativa se plantea la aproximación a personas que
pertenezcan a ambos grupos de la identidad trans, tanto transexual como
transgénero. Y se abordarán,
no en un interés comparativo sino, más bien en un interés comprensivo que
considere la diferencia como una oportunidad para enriquecer el proceso de
análisis y la calidad de las conclusiones de la investigación.

	El acceso a estos informantes está mediado, sin embargo, por las concepciones
venezolanas de la sexualidad como tabú y de la identidad trans como una
desviación. Tradicionalmente, en la sociedad, la identidad trans está asociada
fuertemente con el fenómeno de la prostitución de trans femeninos. Además, el
mayor  interés de las personas trans está fijado en su capacidad para
\emph{pasar} efectivamente como el género con el cual se identifican. Esto puede
representar un obstáculo para el acceso a los informantes pues pueden sentir que
su vivencia de la identidad está en evaluación.

	Afortunadamente se ha tenido la oportunidad durante el último año de establecer
relaciones de amistad y trato con varias personas trans. Estas, además, tienen
como costumbre la búsqueda del contacto con otros individuos de su misma
condición para establecer redes y grupos de apoyo. Esto me permite el acceso a
estas personas como informantes, pues el contacto con estas amistades actúa como
fe de confianza del respeto de su privacidad y confidencialidad. Aprovechando
estas condiciones puedo facilitar el acceso a estos informantes en calidad de
interlocutor válido para vocear sus vivencias en esta investigación ya que
algunos han manifestado ya su interés para participar de la misma.

	Por otro lado, respondiendo al contexto social actual en el cual vivimos en un
país en crisis económica, se dificulta en muchas ocasiones el traslado
frecuente a través de los espacios geográficos. Adicionalmente, los procesos
mismos de transculturización que caracterizan a las sociedades post-modernas
establecen una diferenciación entre las vivencias cotidianas de las urbes y los
espacios rurales. Esta diferencia está mediada por el acceso a tecnologías de
comunicación digital y el establecimiento de prácticas culturales. En este
momento puedo sugerir, en base a las experiencias personales, que la
experiencia y construcción de la identidad sexual entre quienes habitan la
ciudad de Caracas y quienes habitan poblaciones rurales es potencialmente
distinta. Por ello, respondiendo a las limitaciones de tiempo, capacidad de
movilización y construcción simbólica del sentido, considero apropiado limitar
a los informantes claves a aquellos que actualmente habitan la ciudad de
Caracas, lugar donde resido. Esto sin hacer una distinción o discriminación en
cuanto al origen geográfico de los participantes, sino como una mera
constricción práctica. En concordancia con esto último, tampoco se descarta el
uso de las herramientas de comunicación digital propias de nuestra para la
realización de entrevistas en línea. Pero considerando siempre las ventajas y
potencialidades de la comunicación presencial, tales como los elementos
kinésicos y proxémicos de la comunicación interpersonal, por sobre las
limitaciones propias de la tecnología.

\section{Dispositivo de construcción de la información}

La entrevista a profundidad es una técnica de entrevista semi-estructurada en
la cual se construye, junto con el entrevistado o informante, el tema central
de la entrevista. Se trata de una conversación entre dos personas con un
objetivo definido al que se aproxima mediante un proceso natural de
comunicación interpersonal. Sin embargo esta se distingue de una conversación
cotidiana en el sentido que uno de los interlocutores, en este caso el
investigador, posee un conocimiento de los factores psicológicos y sociales
para la obtención de información dirigida a la consecución de un objetivo
particular y predeterminado \parencite{Hidalgo2005}.

Como es expresado por \textcite{Colin1999}, la entrevista es la herramienta
principal de la psicología para la construcción de información y para entrar en
contacto con el universo simbólico de los otros. Y el mismo autor describe a la
entrevista semidirigida como una entrevista que invita al entrevistado a tratar
aspectos que no tiene claro el entrevistador y le da libertad de llenar lagunas
de información, sin dejar de cumplir los objetivos y los tiempos requeridos por
el contexto en el cual la entrevista se está realizando (p. 93).

Se ha seleccionado esta forma de construcción de la información ya que la misma
permite el acceso a profundidad de las vivencias y significados que poseen las
personas sobre su propia realidad. Bajo esta concepción de la entrevista
semidirigida y utilizando la guía de los autores,
\textcites{Colin1999}{Hidalgo2005}{Perpina2014}, planteo un uso de esta técnica
para el tratamiento a profundidad del tema de la vivencia de la trasexualidad y
la transición de sexo. Esto en conformidad con la búsqueda del cumplimiento de
los objetivos de investigación planteados.

Siguiendo las propuestas metodológicas propias de este tipo de entrevista se
utilizará como instrumento de orientación un guión de entrevista que contemple
los ejes temáticos de interés y preguntas abiertas sobre los mismo. Con el fin
de dar la mayor libertad de expresión y elaboración a los informantes sin perder
de vista el foco investigativo. Este guión, que lejos de ser una camisa de
fuerza rígida, estará elaborado en base a la revisión de material referencial
expuesto en el capítulo \ref{marcoreferencial} y consultado en su contenido con
los tutores y demás expertos en el área. Una versión inicial de este guión puede observarse en el anexo \ref{guion}.

\section{Plan de análisis de la información}

	Para analizar la información construida y recopilada durante el curso de esta
investigación se propone el uso de la metodología de análisis de la teoría
fundamentada. En particular la estrategia de comparación constante.

	Esta estrategia o método de análisis como es propuesto por \textcite[][p.
102]{Glaser1967} combina el proceso analítico de la información recopilada y la
construcción de categoría y códigos. En el mismo la información que se va
recabando de los informantes u otras fuentes documentales se codifica y compara
con la información analizada y recopilada anteriormente. Los resultados de este
proceso se utilizan para guiar las indagaciones posteriores. Dando pie a la
modificación de los instrumentos de recolección de información, como las
preguntas en los guiones de entrevista y las búsquedas documentales. Esto con el
propósito de llenar vacíos en las exploraciones previas y de enriquecer los
productos del análisis.
	
	En esta forma de proceder, y en concordancia con el método de la teoría
fundamentada, se van elaborando hipótesis explicativas acerca del fenómeno de la
investigación que luego se contrastan mediante la recopilación de nueva
información a través de esas modificaciones a los instrumentos que se menciona
en el párrafo anterior. Así se puede ir elaborando un modelo de categorías,
hipótesis y propiedades que subyacen a la información recopilada que es flexible
en su estructura. En última instancia creando un modelo de comprensión del
problema de investigación.

%\clearpage
	Por todas estas características considero a la teoría fundamentada y el método
de comparación constante como el adecuado para la persecución de los objetivos
de investigación. Además, en acuerdo con la postura ontológica y
epistemológica desde la cual parto para la aproximación al fenómeno de la
transexualidad y su comprensión fenomenológica.