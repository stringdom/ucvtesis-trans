\chapter{Marco metodológico}

\todo{Aquí va un breve resumen del capítulo y su presentación}

\section{Características de la Investigación}
Toda investigación está fundamentada en alguna postura teórica y metodológica.
Para Strauss y Corbin (2002) la metodología en las ciencias sociales es en sí
una forma de “pensar la realidad social y de estudiarla” (p. 11).
Es decir, es una manera de abordar lo que se quiere conocer.
Partiendo de esta concepción es necesario entonces caracterizar los postulados
de los investigadores para que se entienda en mayor profundidad el contexto de
los resultados obtenidos.
Para hacer esta descripción utilizaremos dos ejes principales: el enfoque y el
paradigma.
Y a partir de allí esclareceremos el resto de los elementos que dan sentido a
esta investigación.

\subsection{Enfoque cualitativo}
El enfoque de la investigación refiere a una forma de pensar y clasificar a las
investigaciones a partir de las características de sus datos, su propósito
general y la elección de métodos que realizan sus investigadores.
En el caso de esta investigación, el enfoque seleccionado ha sido el denominado
cualitativo.
Strauss y Corbin (2002) definen la investigación cualitativa como “cualquier
tipo de investigación que produce hallazgos a los que no se llega por medio de
procedimientos estadísticos u otros medios de cuantificación” (p. 19).
A partir de allí podemos indicar la característica de los datos a coleccionar y
construir para dar sentido a la investigación.

Para Hernández, Fernández y Baptista (2014) el propósito del enfoque cualitativo
“consiste en ‘reconstruir’ la realidad, tal como la observan los actores de un
sistema social definido previamente” (p. 9).
Este propósito condiciona luego la elección de los métodos y procedimientos de
los investigadores para aproximarse al fenómeno en estudio.

Yin (2011) propone que para considerar a una investigación como cualitativa,
esta ha de cumplir con cinco características:

\begin{enumerate}
    \item Estudia el significado en la vida de las personas, en condiciones
    reales.
    \item Representa las perspectivas y puntos de vista de los participantes.
    \item Cubre las condiciones contextuales en las que viven las personas.
    \item Contribuye una mirada a conceptos existentes o emergentes que puede
    ayudar a comprender el comportamiento social humano.
    \item Busca múltiples fuentes de evidencia.
\end{enumerate}

A lo largo de este capítulo abordaremos las distintas formas a través de las
cuales se da cumplimiento a estas características y que permiten considerar a
esta una investigación cualitativa.

Finalmente, la forma en la que se suelen garantizar estas características es a
través de los métodos y procedimientos utilizados.
Estas estrategias, métodos, procedimientos para la aproximación al conocimiento
son concebidos en principio como en oposición, o al menos como una alternativa,
al enfoque cuantitativo predominante en las ciencias sociales durante la primera
parte de su tradición investigativa (Guba y Lincoln, 2002, p. 113).

En la tradición cuantitativa, los pasos se suelen estructurar alrededor del
propósito de falsear el valor de verdad de una hipótesis.
No es este el proceder en la tradición cualitativa.
“Los estudios cualitativos pueden desarrollar preguntas e hipótesis antes,
durante o después de la recolección y el análisis de los datos” afirman
Hernández et.al. (2014, p.7).
Esta naturaleza flexible y adaptable es la que diferencia al enfoque
cualitativo.
Para Glasser y Strauss se trata de la capacidad para “descubrir los puntos de
vista emic” (c.p. Guba y Lincoln, 2002, p. 116).
En otras palabras, la capacidad de comprender los puntos de vista internos de
los individuos, grupos y sociedades, datos que en una cuantificación se
perderían de vista y serían invisibilizados.

Sin embargo, Guba y Lincoln (2002) opinan que “las cuestiones de método son
secundarias frente a las de paradigma” (p. 113).
Es por ello que a continuación presentamos un encuadre paradigmático de esta
investigación.

\subsection{Postura paradigmática}
Para caracterizar y describir esta investigación nos basaremos en las propuestas
expuestas por Guba y Lincoln (2002).
Para estos autores, un paradigma es “el sistema básico de creencias o visión del
mundo que guía al investigador” (p. 113).
También conceptualizado por Martínez Miguélez (2006) como “las relaciones
primordiales que constituyen los supuestos básicos, determinan los conceptos
fundamentales y rigen los discursos y las teorías” (p. 38).

Para Guba y Lincoln (2002) la ubicación paradigmática de una investigación se
concibe a partir de tres ejes: ontología, epistemología y metodología.
Es decir, la forma en la que se concibe la realidad humana, el conocimiento
sobre esa realidad y la forma de acceder a este.
Estos tres ejes pueden definirse en función de la respuesta a tres preguntas (p.
120):

\begin{itemize}
    \item Pregunta ontológica: ¿Cuál es la forma y la naturaleza de la realidad
    y qué es lo que podemos conocer de ella?
    \item Pregunta epistemológica: ¿Cuál es la naturaleza de la relación entre
    quien conoce y lo que puede ser conocido?
    \item Pregunta metodológica: ¿Cómo puede el investigador averiguar si lo que
    él cree puede ser conocido?
\end{itemize}

Intentaremos dar respuesta a cada una.
En su planteamiento, Guba y Lincoln (2002), comparan tres paradigmas que fungen
de categorías generales para agrupar a un amplio conjunto de metodologías y
posturas que surgen como alternativas ante la crítica al paradigma positivista
heredado de la tradición científica.
Las metodologías son agrupadas en función de las características que comparten
respecto a los tres ejes paradigmáticos.
Estos tres paradigmas son: el post-positivismo, la teoría crítica, y el
socio-construccionismo.

Esta investigación parte en principio desde una mirada socio-construccionista.
Cabe destacar que Guba y Lincoln (2002) refieren para esta postura paradigmática
la denominación de ‘teoría constructivista’.
Nos haremos eco en su lugar de la crítica elaborada por Martínez Miguélez (2006)
de que el uso de este termino denota una postura radicalmente opuesta al
positivismo.
Lo que es igual a decir que todo el mundo externo al individuo es un simple
“material de construcción, informe y desarticulado” (p. 43).
Por ello preferimos hacer uso del término ‘socio-construccionismo’ que refleja
varias de las cualidades que preferimos como investigadores.
En resumen, un enfoque cualitativo que desde la lógica dialéctica aborda la
realidad de los participantes para captar su sentido en la forma en la que estos
lo interpretan.
Esto coincide con cierta cercanía con los planteamientos del ‘nuevo paradigma
emergente’ según Martínez Miguélez (2006).

\subsubsection{Ontología relacional}
Para la visión del constructivismo o construccionismo social, no existe una
realidad \emph{verdadera}.
Por el contrario, cada grupo, cultura, sociedad, o conjunto de personas elabora
de forma dinámica construcciones que dan cuenta y sentido a la realidad.
Por ello las construcciones sobre la realidad son múltiples e intangibles pero
comprensibles que dependen de los individuos y grupos que las sostienen, y son
mutables, susceptibles de cambiar (Guba y Lincoln, 2002).

Entonces, “las construcciones no son más o menos ‘verdaderas’ en ningún sentido
absoluto” (Guba y Lincoln, 2002, p. 128).
Las relaciones son relativas y relacionales, su sentido cambiará en función del
contexto y de quienes se encuentren participando en su construcción y
reconstrucción.

En resumen, el paradigma socio-construccionista propone que la realidad es un
producto de las relaciones sociales en un contexto local y temporal específico,
y que lo que se considera verdadero no tiene que ver con el grado de objetividad
sino con la aceptación obtenida de la comunidad donde se genera.

Desde esta mirada, esta investigación concibe la realidad de las personas trans
como una realidad propia y única que surge de la interacción con su entorno y
contexto social.
Esta realidad no es una verdad absoluta, sino su interpretación y construcción
relativa.
En consecuencia, lo que podemos llegar a conocer es la interpretación de los
investigadores de la experiencia, sentido y significado que reportan, sobre su
propia realidad, las personas trans.

\subsubsection{Epistemología transaccional}
Desde la epistemología, el socio-construccionismo supone que existe un vínculo
entre el investigador y el objeto de estudio y que esta relación va
transformando el fenómeno a estudiar.
En consecuencia, no es posible distinguir el fenómeno a estudiar del mismo
proceso de investigación pues ambos se van construyendo simultáneamente.
“Los ‘hallazgos’ son literalmente creados al avanzar la investigación” (Guba y
Lincoln, 2002, p. 128).
Este vinculo no es objetivo, o ajeno a la experiencia individual, sino que es
subjetivo.
“El investigador y el \emph{objeto} de investigación están vinculados
interactivamente”
(p. 128).
Por ello se habla de una epistemología transaccional y subjetiva.

Entonces, esta investigación aborda la realidad de las personas trans desde la
dimensión experiencial.
Recopilando el testimonio y el reporte de los participantes junto con el
investigador.
Así, se encuentra una sección de este capítulo denominada participantes, en
lugar del clásico muestreo en la tradición del positivismo lógico.
El origen de los datos a interpretar no está en una acumulación estadística de
muestras o cuantificaciones.
Sino en la relación que los investigadores establecieron con los participantes
durante los encuentros de entrevista y las construcciones que de allí se
derivan.

\subsubsection{Metodología hermenéutica y dialéctica}
En la definición comparativa del socio-construccionismo, Guba y Lincoln (2002)
plantean, respecto a la metodología, que “el objetivo final es destilar una
construcción consensuada que sea más informada y sofisticada que cualquiera de
las construcciones precedentes” (p. 128).

Al respecto Martínez Miguélez (2006) resalta que una epistemología emergente se
encuentra en contra de la existencia de un “punto arquimédico del conocimiento”
(p. 45).
Expresado de otra manera, lo que se está intentando conocer no es una ‘verdad
pura’, sino una verdad entretejida con nuestras “relaciones y compromisos con el
mundo” (Heidegger c.p. Martínez Miguélez, 2006, p. 107).
Explica luego que el modo de conocer dialéctico es similar al modo espontáneo de
conocer de la mente humana.
“Los seres humanos conocemos a través de la interacción” (p. 107).
Sin interacción no puede haber comprensión.

\begin{quote}
    Ser humano es ser ‘interpretativo’, porque la verdadera naturaleza de la
    realidad humana es ‘interpretativa’; por tanto, la interpretación no es un
    ‘instrumento’ para adquirir conocimientos, es el modo natural de ser de los
    seres humanos (Heidegger c.p. Martínez Miguélez, 2006, p. 107)
\end{quote}

Siguiendo entonces a Heidegger, la forma en la cual se puede uno aproximar a la
construcción informada y más sofisticada que mencionan Guba y Lincoln como
objetivo del conocer socio-construccionista, hemos de voltear la atención a la
interpretación de la realidad y las experiencias de los participantes.
Esta interpretación es caracterizada por Martínez Miguélez (2006) como una
‘fusión de horizontes’ o “una interacción dialéctica entre las expectativas del
intérprete y el significado del texto o acto humano” (p. 108).
La misma toma la forma de un \emph{circulo virtuoso}, en oposición al círculo
vicioso,
o ‘circulo hermenéutico.
En el mismo se va del todo a las partes y de nuevo de las partes al todo para
elaborar el sentido de lo que se intenta interpretar.

Retomando lo anteriormente expuesto, si la realidad es relativa, relacional y el
conocimiento que se puede tener de esta se encuentra mediado por las
construcciones múltiples y mutables que las personas hacen de ella, entonces ha
de ser posible recopilar, registrar e interpretar estas construcciones para
darles sentido, reconstruirlas, interpretarlas de la forma más completa posible,
a través de establecer una relación interactiva con los sujetos a quienes
investigamos.

En un sentido pragmático, las consecuencias de esta postura paradigmática se
expresan en una elección de método particular para guiar los aspectos logísticos
y pasos instruccionales a seguir.
Esta elección dentro del enfoque cualitativo está, sin embargo, sujeta a
cambios.
El método más apropiado para una investigación será aquel que permita establecer
la relación más fructífera en el contexto del fenómeno investigado.

En nuestro caso, este devenir nos ha llevado a seleccionar como método de
investigación al estudio de casos.
Entenderemos el estudio de caso como “un abordaje empírico que investiga un
fenómeno dentro de su contexto de la vida real, especialmente cuando las
fronteras entre fenómeno y contexto no son claramente evidentes” (traducción de
los autores: “A case study is an empirical inquiry that investigates a
contemporary phenomenon within its real-life context, especially when the
boundaries between phenomenon and context are not clearly evident”, p. 13).
Esta definición de Robert Yin (2011) ilustra la intencionalidad del estudio de
casos y la manera en la que este se ajusta a la visión epistemológica y
metodológica propuesta por el paradigma socio-construccionista y expuesta en los
párrafor anteriores.

Además, permite una forma particular de construcción del conocimiento a través
de la interacción investigador y el caso investigado pues a través del método de
estudio de casos “el individuo es la unidad primaria de análisis” (traducción:
“an individual person is the case being studied and the individual is the
primary unit of analysis”, Yin, 2011, p. 21).
Esto es de particular utilidad en el estudio de la experiencia y vivencia de las
personas trans pues serán ellas mismas la unidad de análisis utilizada para
teorizar.

Yin (2011) propone que para poder hablar de un estudio de caso, y considerarle
como útil para una investigación, se deben tomar en cuenta tres elementos: las
características de la pregunta de investigación, el nivel de control que tienen
los investigadores sobre el fenómeno y el énfasis en fenómenos contemporáneos en
contraposición a fenómenos históricos.

A partir de allí podemos afirmar que nuestra pregunta de investigación: ¿Cómo
influye la transición de sexo y género en la construcción de la identidad en un
grupo de personas transgénero que residen en el Área Metropolitana de  Caracas?,
se asemeja a la sugerencia de Yin (2011).
Este tipo de pregunta está tipificado por este autor como el tipo de pregunta
para el cuál el estudio de casos es más apropiado.
Se trata de una pregunta explicativa que intenta ubicar las asociaciones
operacionales entre los elementos presentes en el fenómeno.

En segundo lugar, el ser trans, la identidad de género, y la transición, son
fenómenos espontáneos de la vida individual y social de aquellas personas que se
definen como trans.
Sobre estas vivencias, experiencias y subjetividades no existen manipulaciones
experimentales posibles dentro del universo de lo éticamente correcto.
Es por ello que como investigadores no nos encontramos en una posición donde nos
sea posible controlar o manipular el fenómeno.

Finalmente, el interés investigativo será centrado en un fenómeno contemporáneo.
No es parte de nuestro propósito investigativo elaborar respecto al devenir
socio-histórico de la transexualidad y transgenerismo en nuestra sociedad.
Sino establecer una comprensión de su vivencia hoy día, en personas vivas que
están experimentando el fenómeno en este momento.
Con estos tres elementos, consideramos que podemos hacer uso del estudio de caso
pues la investigación cumple con las consideraciones sugeridas para que el mismo
sea útil y apropiado.

Sobre el tipo de estudio que estaremos realizando, lo denominaremos un estudio
de caso colectivo por tratarse de una investigación en la cual se seleccionaron
los participantes por una “condición general” compartida entre todos ellos y que
“se ha de estudiar intensivamente” (Stake c.p. Jimenez, 2016, p. 7).

\section{Participantes}
En el proceso cualitativo, la muestra hace referencia a un grupo de personas,
eventos, sucesos, comunidades, etc., sobre el cual se habrán de recolectar los
datos, sin que necesariamente sea representativo del universo o población que se
estudia (Hernández, Fernández y Baptista, 2014).
No estamos intentando entonces, desde este enfoque, establecer una
representatividad estadística.
Adicionalmente, la construcción de la información será interactiva y subjetiva.
Por estos motivos utilizaremos en esta investigación el termino de
participantes, individuos o, a veces, unidad de análisis, con preferencia por el
primero.

Con estas consideraciones en mente realizamos la búsqueda y selección de
nuestros participantes utilizando las estrategias sugeridas por Hernández et
.al. (2014).
Esta es el denominado muestreo propositivo en investigación cualitativa en el
cual “nos preguntamos qué casos nos interesan inicialmente y dónde podemos
encontrarlos” (p. 384).

Seleccionamos el criterio de elección a partir del muestreo teórico para
asegurarnos de que participarían sólo aquellas personas con el atributo que
contribuiría más a informar y generar teoría (Draucker, Martsolf, Ross y Rusk
c.p. Hernández et.al. 2014 p. 389).
Esto coincide con las sugerencias de Martínez Miguélez (2006) quien describe la
elección de la muestra en investigación cualitativa como intencional y “donde se
prioriza la profundidad sobre la amplitud” (p. 83).
Esta elección también destaca por carecer de criterios probabilísticos o
estadísticos pues su intención no es una generalización estadística de los pocos
participante al universo de poblacional.

El criterio final que determinó la elección en esta investigación fue el
siguiente: pueden participar aquellas personas mayores de 18 años de edad de la
ciudad de Caracas que se identifiquen a sí mismas como transgénero o
transexuales.
Esto responde a tres condiciones: deben ser consideraros en un sentido amplio
adultos capaces de dar su consentimiento informado, estar geográficamente
cercanos a los investigadores para facilitar la realización de entrevistas en
persona y además considerarse a sí mismo como personas trans.

Nos apoyamos, además, por un muestreo en cadena.
A través del contacto con un participante inicial se le preguntó si conocía a
otras personas de características similares que pudieran participar (Morgan c.p.
Hernández et. al. , 2014 p. 388).

Sobre la cantidad de participantes, Hernández et.al. (2014) propone tres
factores para determinar su número: La capacidad operativa de recolección y
análisis; el entendimiento del fenómeno, o saturación de categorías; y la
naturaleza del fenómeno en cuanto a su frecuencia y accesibilidad (p. 384).
El número final de participantes en esta investigación es de 3 personas, 2
hombres transgénero en distintas etapas de transición hormonal y una mujer
transgénero sólo con operación de mamás y hormonación.

Consideramos este número como suficiente para obtener información sobre la
experiencia en la transición de su constitución de identidad y poder hacer un
análisis completo.
Durante el análisis preliminar los testimonios informaron
suficientes categorías en común como para considerar una saturación de
información.
Además, la baja cantidad de personas trans en la ciudad de Caracas
que los mismos participantes reportan hace difícil asegurar un número mayor de
unidades de análisis, atenuado también por el estigma social que les hace buscar
ser menos visibles y más reservados con su privacidad.

A continuación, se presenta una tabla donde se indican algunas características
que definieron los criterios de selección de los participantes y la numeración
asignadas para identificarles en el capítulo correspondiente al análisis y
discusión de los resultados.

\begin{table}[]
    \caption{Descripción de los participantes}
    \label{tab:participantes}
    \centering
    \renewcommand{\tabcolsep}{1pt}
    \begin{tabularx}{\textwidth}{@{}cXXcXX@{}}
    \toprule
    Id & Sexo & Género         & Edad & Denominación & Ocupación
    \\ \midrule
    1            & Masculino      & Hombre a Mujer & 45   & Transgénero      & Abogada            \\
    2            & Femenino       & Mujer a hombre & 32   & Transgénero      & Lic. en Psicología \\
    3            & Femenino       & Mujer a hombre & 24   & Transgénero      & TSU Informático    \\ \bottomrule
    \end{tabularx}
\end{table}

\section{Estrategia de construcción de la información}
La estrategia seleccionada como principal fuente y forma de construcción de la
información para la investigación fue la entrevista.
Esta es definida por Janesick (c.p. Hernández et.al, 2014) como “una reunión
para conversar e intercambiar información entre una persona (el entrevistador) y
otra (el entrevistado) […] en la entrevista se logra una comunicación y la
construcción conjunta de significados respecto a un tema” (p. 403).

Su propósito es una recolección de parte del entrevistador de lo que se construye
durante el encuentro. Hernández et.al. (2014) caracteriza esta recolección como
“obtener las perspectivas y puntos de vista de los participantes (sus emociones,
prioridades, experiencias, significados y otros aspectos más bien subjetivos)”
(p. 8).
Para Martínez Miguélez (2006) se trata de un “diálogo coloquial” (p. 93) y por
ende de una estrategia mayormente verbal y experiencial.

Esta reunión y comunicación puede adquirir muchas formas y características.
En función del nivel de predeterminación de las preguntas, el nivel de libertad
que se le permite al entrevistado y la forma en la que se conduce la relación
entre los interlocutores.
En la clasificación de Hernández et.al. (2014), la estrategia utilizada por
esta investigación es la entrevista semi-estructurada.
Esta se define como aquella entrevista en la que “el entrevistador tiene la
libertad de introducir preguntas adicionales para precisar conceptos u obtener
mayor información” (p. 403).

Estas entrevistas se llevaron a cabo siguiendo las sugerencias expresadas por
Martínez Miguélez (2006) y ajustadas a la caracterización que elabora Yin (2011)
sobre como se deben realizar entrevistas en investigación de estudio de casos
con enfoque cualitativo.
Más específicamente, la actitud del entrevistador fue la de una conversación
natural, manteniendo un ‘rapport’ con el entrevistado, dejando al participante
vocalizar sus propias prioridades conversacionales, evitando una expresión
excesivamente obvia de los juicios de valor del investigador y siguiendo un
protocolo de entrevista.

Este protocolo fue un guion elaborado antes de las entrevistas en base al
interés exploratorio de los investigadores, algunas experiencias previas con
personas trans y el conocimiento de algunas propuestas teóricas ya existentes
respecto a la vivencia de las personas trans.
Este guion inicial consistía de 41 preguntas de longitud variada clasificadas en
once (11) diferentes temáticas agrupadas a su vez en dos (2) ejes temáticos:
Identidad y transición.
En la tabla 2 se puede observar cuales fueron estas categorías y sub-categorías.
En los anexos se puede leer el texto de cada pregunta.
El protocolo no era otra cosa que una guía para el entrevistador pues tanto
entrevistado como entrevistador disponían de la libertad de seguir cualquier
línea de cuestionamiento relevante que surgiera en la conversación, dejando
espacio abierto para agregar nuevos temas, nuevas preguntas o profundizar en
algunos puntos novedosos.
En general, las entrevistas duraron entre una (1) y dos (2) horas cada una.

\begin{table}[]
\centering
\caption{Estructura del protocolo de entrevista semiestructurada}
\label{tab:protocolo}
\begin{tabularx}{\textwidth}{@{}XX@{}}
\toprule
Eje temático                & Tema                                        \\ \midrule
\multirow{7}{*}{Identidad}  & Identidad de género                         \\
                            & Ámbito laboral                              \\
                            & Devenir, familia                            \\
                            & Relaciones de pareja                        \\
                            & Expresión                                   \\
                            & Estatus legal                               \\
                            & Discriminación                              \\
\midrule
\multirow{4}{*}{Transición} & Disforia de género                          \\
                            & Concepción del cuerpo                       \\
                            & Significación de los procesos de transición \\
                            & Perspectiva a futuro                        \\
\bottomrule
\end{tabularx}
\end{table}

\section{Metodología de Análisis}
Desde el enfoque cualitativo la orientación de la investigación está dirigida a
la creación de teoría.
Martínez Miguélez (2006) propone que el producto final de la investigación
cualitativa es producir una teoría que cuenta del fenómeno investigado.
En palabras de Jones: “lo principal es generar una comprensión del problema de
investigación, en lugar de forzar los datos dentro de una lógica deductiva
derivada de categorías o suposiciones” (c.p. Strauss y Corbin, 2002, p. 23).

Lo que esto significa es que durante el análisis no intentaremos comprobar
hipótesis explicativas construidas a priori.
En su lugar, buscaremos la teoría explicativa contenida en la información
construida con los participantes.
El procedimiento que aplicaremos para ello es denominado por Yin (2011) como una
“generalización analítica, en la que una teoría previamente desarrollada es
usada como base con la cual comparar los resultados empíricos del caso de
estudio” (p. 31).
Este será el rol del marco referencial encontrado en el capítulo 3 de este
texto.
Es decir, el rol de la teoría previa es el de informar, proveer insumos, para el
análisis de los casos.
En contraste directo con las posturas deductivas propias del paradigma
neopositivistas que proponen una generalización estadística.

Este proceso de análisis está fundamentado en las propuestas del interaccionismo
interpretativo como es descrito por Denzin.
De esta tradición tomamos el concepto de triangulación múltiple como “la
combinación de múltiples métodos, múltiples tipos de datos, múltiples observadores y múltiples teorías”
(c.p. Martínez Miguélez, 2006, p. 128).
Es esta combinación de teoría anteriores y varias fuentes de información las que
permiten alcanzar la ‘construcción mejor informada y más completa posible’ a la
que hacen referencia Guba y Lincoln (2002) como producto final de la
investigación desde el paradigma socio-construccionista.

Para analizar la información producida a través de las entrevistas utilizamos
las estrategias de codificación expuestas por Strauss y Corbin (2002).
Para la fase inicial de aproximación a los datos utilizamos la codificación
abierto que como "el proceso analítico por medio del cual se identifican los
conceptos y se descubren en los datos sus propiedades y dimensiones" (p.110).
Esta clasificación inicial explora y recopila los elementos principales sentido
y de comprensión presentes en las entrevistas con cada caso.

