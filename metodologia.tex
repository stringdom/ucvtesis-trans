\chapter{Marco metodológico}\label{ch:metodologia}
Toda investigación está fundamentada en alguna postura teórica y metodológica.
La elección de esta no es un acto aleatorio o azaroso.
Por el contrario, se estima en el discurso científico que la elección
metodológica sea el resultado de un análisis que permita utilizar los
herramientas que mejor se adapten al problema a investigar.
En este capítulo presentaremos las decisiones que hemos tomado alrededor de
esta investigación así como la lógica y el razonamiento al que responde cada
una.

Al final del capítulo presentamos una breve evaluacion ética de la
metodología de investigación.

\section{Características de la Investigación}
Para Strauss y Corbin (2002) la metodología en las ciencias sociales es en sí
una forma de “pensar la realidad social y de estudiarla” (p. 11).
Es decir, es una manera de abordar lo que se quiere conocer.
Partiendo de esta concepción es necesario entonces caracterizar los postulados
de los investigadores para que se entienda en mayor profundidad el contexto de
los resultados obtenidos.
Para hacer esta descripción utilizaremos dos ejes principales: el enfoque y el
paradigma.
Y a partir de allí esclareceremos el resto de los elementos que dan sentido a
esta investigación.

\subsection{Enfoque cualitativo}
El enfoque de la investigación refiere a una forma de pensar y clasificar a las
investigaciones a partir de las características de sus datos, su propósito
general y la elección de métodos que realizan sus investigadores.
En el caso de esta investigación, el enfoque seleccionado ha sido el denominado
cualitativo.
Strauss y Corbin (2002) definen la investigación cualitativa como “cualquier
tipo de investigación que produce hallazgos a los que no se llega por medio de
procedimientos estadísticos u otros medios de cuantificación” (p. 19).
A partir de allí podemos indicar la característica de los datos a coleccionar y
construir para dar sentido a la investigación.

Para Hernández, Fernández y Baptista (2014) el propósito del enfoque cualitativo
“consiste en ‘reconstruir’ la realidad, tal como la observan los actores de un
sistema social definido previamente” (p. 9).
Este propósito condiciona luego la elección de los métodos y procedimientos de
los investigadores para aproximarse al fenómeno en estudio.

Yin (2011) propone que para considerar a una investigación como cualitativa,
esta ha de cumplir con cinco características:

\begin{enumerate}
    \item Estudia el significado en la vida de las personas, en condiciones
    reales.
    \item Representa las perspectivas y puntos de vista de los participantes.
    \item Cubre las condiciones contextuales en las que viven las personas.
    \item Contribuye una mirada a conceptos existentes o emergentes que puede
    ayudar a comprender el comportamiento social humano.
    \item Busca múltiples fuentes de evidencia.
\end{enumerate}

A lo largo de este capítulo abordaremos las distintas formas a través de las
cuales se da cumplimiento a estas características y que permiten considerar a
esta una investigación cualitativa.

Finalmente, la forma en la que se suelen garantizar estas características es a
través de los métodos y procedimientos utilizados.
Estas estrategias, métodos, procedimientos para la aproximación al conocimiento
son concebidos en principio como en oposición, o al menos como una alternativa,
al enfoque cuantitativo predominante en las ciencias sociales durante la primera
parte de su tradición investigativa (Guba y Lincoln, 2002, p. 113).

En la tradición cuantitativa, los pasos se suelen estructurar alrededor del
propósito de falsear el valor de verdad de una hipótesis.
No es este el proceder en la tradición cualitativa.
“Los estudios cualitativos pueden desarrollar preguntas e hipótesis antes,
durante o después de la recolección y el análisis de los datos” afirman
Hernández et al. (2014, p.7).
Esta naturaleza flexible y adaptable es la que diferencia al enfoque
cualitativo.
Para Glasser y Strauss se trata de la capacidad para “descubrir los puntos de
vista emic” (c.p. Guba y Lincoln, 2002, p. 116).
En otras palabras, la capacidad de comprender los puntos de vista internos de
los individuos, grupos y sociedades, datos que en una cuantificación se
perderían de vista y serían invisibilizados.

Sin embargo, Guba y Lincoln (2002) opinan que “las cuestiones de método son
secundarias frente a las de paradigma” (p. 113).
Es por ello que a continuación presentamos un encuadre paradigmático de esta
investigación.

\subsection{Postura paradigmática}
Para caracterizar y describir esta investigación nos basaremos en las propuestas
expuestas por Guba y Lincoln (2002).
Para estos autores, un paradigma es “el sistema básico de creencias o visión del
mundo que guía al investigador” (p. 113).
También conceptualizado por Martínez Miguélez (2006) como “las relaciones
primordiales que constituyen los supuestos básicos, determinan los conceptos
fundamentales y rigen los discursos y las teorías” (p. 38).

Para Guba y Lincoln (2002) la ubicación paradigmática de una investigación se
concibe a partir de tres ejes: ontología, epistemología y metodología.
Es decir, la forma en la que se concibe la realidad humana, el conocimiento
sobre esa realidad y la forma de acceder a este.
Estos tres ejes pueden definirse en función de la respuesta a tres preguntas (p.
120):

\begin{itemize}
    \item Pregunta ontológica: ¿Cuál es la forma y la naturaleza de la realidad
    y qué es lo que podemos conocer de ella?
    \item Pregunta epistemológica: ¿Cuál es la naturaleza de la relación entre
    quien conoce y lo que puede ser conocido?
    \item Pregunta metodológica: ¿Cómo puede el investigador averiguar si lo que
    él cree puede ser conocido?
\end{itemize}

Intentaremos dar respuesta a cada una.
En su planteamiento, Guba y Lincoln (2002), comparan tres paradigmas que fungen
de categorías generales para agrupar a un amplio conjunto de metodologías y
posturas que surgen como alternativas ante la crítica al paradigma positivista
heredado de la tradición científica.
Las metodologías son agrupadas en función de las características que comparten
respecto a los tres ejes paradigmáticos.
Estos tres paradigmas son: el post-positivismo, la teoría crítica, y el
socio-construccionismo.

Esta investigación parte en principio desde una mirada socio-construccionista.
Cabe destacar que Guba y Lincoln (2002) refieren para esta postura paradigmática
la denominación de ‘teoría constructivista’.
Nos haremos eco en su lugar de la crítica elaborada por Martínez Miguélez (2006)
de que el uso de este termino denota una postura radicalmente opuesta al
positivismo.
Lo que es igual a decir que todo el mundo externo al individuo es un simple
“material de construcción, informe y desarticulado” (p. 43).
Por ello preferimos hacer uso del término ‘socio-construccionismo’ que refleja
varias de las cualidades que preferimos como investigadores.
En resumen, un enfoque cualitativo que desde la lógica dialéctica aborda la
realidad de los participantes para captar su sentido en la forma en la que estos
lo interpretan.
Esto coincide con cierta cercanía con los planteamientos del ‘nuevo paradigma
emergente’ según Martínez Miguélez (2006).

\subsubsection{Ontología relacional}
Para la visión del constructivismo o construccionismo social, no existe una
realidad \emph{verdadera}.
Por el contrario, cada grupo, cultura, sociedad, o conjunto de personas elabora
de forma dinámica construcciones que dan cuenta y sentido a la realidad.
Por ello las construcciones sobre la realidad son múltiples e intangibles pero
comprensibles que dependen de los individuos y grupos que las sostienen, y son
mutables, susceptibles de cambiar (Guba y Lincoln, 2002).

Entonces, “las construcciones no son más o menos ‘verdaderas’ en ningún sentido
absoluto” (Guba y Lincoln, 2002, p. 128).
Las relaciones son relativas y relacionales, su sentido cambiará en función del
contexto y de quienes se encuentren participando en su construcción y
reconstrucción.

En resumen, el paradigma socio-construccionista propone que la realidad es un
producto de las relaciones sociales en un contexto local y temporal específico,
y que lo que se considera verdadero no tiene que ver con el grado de objetividad
sino con la aceptación obtenida de la comunidad donde se genera.

Desde esta mirada, esta investigación concibe la realidad de las personas trans
como una realidad propia y única que surge de la interacción con su entorno y
contexto social.
Esta realidad no es una verdad absoluta, sino su interpretación y construcción
relativa.
En consecuencia, lo que podemos llegar a conocer es la interpretación de los
investigadores de la experiencia, sentido y significado que reportan, sobre su
propia realidad, las personas trans.

\subsubsection{Epistemología transaccional}
Desde la epistemología, el socio-construccionismo supone que existe un vínculo
entre el investigador y el objeto de estudio y que esta relación va
transformando el fenómeno a estudiar.
En consecuencia, no es posible distinguir el fenómeno a estudiar del mismo
proceso de investigación pues ambos se van construyendo simultáneamente.
“Los ‘hallazgos’ son literalmente creados al avanzar la investigación” (Guba y
Lincoln, 2002, p. 128).
Este vinculo no es objetivo, o ajeno a la experiencia individual, sino que es
subjetivo.
“El investigador y el \emph{objeto} de investigación están vinculados
interactivamente”
(p. 128).
Por ello se habla de una epistemología transaccional y subjetiva.

Entonces, esta investigación aborda la realidad de las personas trans desde la
dimensión experiencial.
Recopilando el testimonio y el reporte de los participantes junto con el
investigador.
Así, se encuentra una sección de este capítulo denominada participantes, en
lugar del clásico muestreo en la tradición del positivismo lógico.
El origen de los datos a interpretar no está en una acumulación estadística de
muestras o cuantificaciones.
Sino en la relación que los investigadores establecieron con los participantes
durante los encuentros de entrevista y las construcciones que de allí se
derivan.

\subsubsection{Metodología hermenéutica y dialéctica}
En la definición comparativa del socio-construccionismo, Guba y Lincoln (2002)
plantean, respecto a la metodología, que “el objetivo final es destilar una
construcción consensuada que sea más informada y sofisticada que cualquiera de
las construcciones precedentes” (p. 128).

Al respecto Martínez Miguélez (2006) resalta que una epistemología emergente se
encuentra en contra de la existencia de un “punto arquimédico del conocimiento”
(p. 45).
Expresado de otra manera, lo que se está intentando conocer no es una ‘verdad
pura’, sino una verdad entretejida con nuestras “relaciones y compromisos con el
mundo” (Heidegger c.p. Martínez Miguélez, 2006, p. 107).
Explica luego que el modo de conocer dialéctico es similar al modo espontáneo de
conocer de la mente humana.
“Los seres humanos conocemos a través de la interacción” (p. 107).
Sin interacción no puede haber comprensión.

\begin{quote}
    Ser humano es ser ‘interpretativo’, porque la verdadera naturaleza de la
    realidad humana es ‘interpretativa’; por tanto, la interpretación no es un
    ‘instrumento’ para adquirir conocimientos, es el modo natural de ser de los
    seres humanos (Heidegger c.p. Martínez Miguélez, 2006, p. 107)
\end{quote}

Siguiendo entonces a Heidegger, la forma en la cual se puede uno aproximar a la
construcción informada y más sofisticada que mencionan Guba y Lincoln como
objetivo del conocer socio-construccionista, hemos de voltear la atención a la
interpretación de la realidad y las experiencias de los participantes.
Esta interpretación es caracterizada por Martínez Miguélez (2006) como una
‘fusión de horizontes’ o “una interacción dialéctica entre las expectativas del
intérprete y el significado del texto o acto humano” (p. 108).
La misma toma la forma de un \emph{circulo virtuoso}, en oposición al círculo
vicioso,
o ‘circulo hermenéutico.
En el mismo se va del todo a las partes y de nuevo de las partes al todo para
elaborar el sentido de lo que se intenta interpretar.

Retomando lo anteriormente expuesto, si la realidad es relativa, relacional y el
conocimiento que se puede tener de esta se encuentra mediado por las
construcciones múltiples y mutables que las personas hacen de ella, entonces ha
de ser posible recopilar, registrar e interpretar estas construcciones para
darles sentido, reconstruirlas, interpretarlas de la forma más completa posible,
a través de establecer una relación interactiva con los sujetos a quienes
investigamos.

En un sentido pragmático, las consecuencias de esta postura paradigmática se
expresan en una elección de método particular para guiar los aspectos logísticos
y pasos instruccionales a seguir.
Esta elección dentro del enfoque cualitativo está, sin embargo, sujeta a
cambios.
El método más apropiado para una investigación será aquel que permita establecer
la relación más fructífera en el contexto del fenómeno investigado.

En nuestro caso, este devenir nos ha llevado a seleccionar como método de
investigación al estudio de casos.
Entenderemos el estudio de caso como un abordaje empírico que investiga un
fenómeno dentro de su contexto de la vida real, especialmente cuando las
fronteras entre fenómeno y contexto no son claramente evidentes (Yin, 2011, p.
13).
Esta definición de Robert Yin (2011) ilustra la intencionalidad del estudio de
casos y la manera en la que este se ajusta a la visión epistemológica y
metodológica propuesta por el paradigma socio-construccionista y expuesta en los
párrafor anteriores.

Además, permite una forma particular de construcción del conocimiento a través
de la interacción investigador y el caso investigado pues a través del método de
estudio de casos el individuo es la unidad primaria de análisis (Yin, 2011, p.
21).
Esto es de particular utilidad en el estudio de la experiencia y vivencia de las
personas trans pues serán ellas mismas la unidad de análisis utilizada para
teorizar.

Yin (2011) propone que para poder hablar de un estudio de caso, y considerarle
como útil para una investigación, se deben tomar en cuenta tres elementos: las
características de la pregunta de investigación, el nivel de control que tienen
los investigadores sobre el fenómeno y el énfasis en fenómenos contemporáneos en
contraposición a fenómenos históricos.

A partir de allí podemos afirmar que nuestra pregunta de investigación: ¿Cómo
influye la transición de sexo y género en la construcción de la identidad en un
grupo de personas transgénero que residen en el Área Metropolitana de  Caracas?,
se asemeja a la sugerencia de Yin (2011).
Este tipo de pregunta está tipificado por este autor como el tipo de pregunta
para el cuál el estudio de casos es más apropiado.
Se trata de una pregunta explicativa que intenta ubicar las asociaciones
operacionales entre los elementos presentes en el fenómeno.

En segundo lugar, el ser trans, la identidad de género, y la transición, son
fenómenos espontáneos de la vida individual y social de aquellas personas que se
definen como trans.
Sobre estas vivencias, experiencias y subjetividades no existen manipulaciones
experimentales posibles dentro del universo de lo éticamente correcto.
Es por ello que como investigadores no nos encontramos en una posición donde nos
sea posible controlar o manipular el fenómeno.

Finalmente, el interés investigativo será centrado en un fenómeno contemporáneo.
No es parte de nuestro propósito investigativo elaborar respecto al devenir
socio-histórico de la transexualidad y transgenerismo en nuestra sociedad.
Sino establecer una comprensión de su vivencia hoy día, en personas vivas que
están experimentando el fenómeno en este momento.
Con estos tres elementos, consideramos que podemos hacer uso del estudio de caso
pues la investigación cumple con las consideraciones sugeridas para que el mismo
sea útil y apropiado.

Sobre el tipo de estudio que estaremos realizando, lo denominaremos un estudio
de caso colectivo por tratarse de una investigación en la cual se seleccionaron
los participantes por una “condición general” compartida entre todos ellos y que
“se ha de estudiar intensivamente” (Stake c.p. Jimenez, 2016, p. 7).

\section{Participantes}
En el proceso cualitativo, la muestra hace referencia a un grupo de personas,
eventos, sucesos, comunidades, etc., sobre el cual se habrán de recolectar los
datos, sin que necesariamente sea representativo del universo o población que se
estudia (Hernández, Fernández y Baptista, 2014).
No estamos intentando entonces, desde este enfoque, establecer una
representatividad estadística.
Adicionalmente, la construcción de la información será interactiva y subjetiva.
Por estos motivos utilizaremos en esta investigación el termino de
participantes, individuos o, a veces, unidad de análisis, con preferencia por el
primero.

Con estas consideraciones en mente realizamos la búsqueda y selección de
nuestros participantes utilizando las estrategias sugeridas por Hernández et al.
(2014).
Esta es el denominado muestreo propositivo en investigación cualitativa en el
cual “nos preguntamos qué casos nos interesan inicialmente y dónde podemos
encontrarlos” (p. 384).

Seleccionamos el criterio de elección a partir del muestreo teórico para
asegurarnos de que participarían sólo aquellas personas con el atributo que
contribuiría más a informar y generar teoría (Draucker, Martsolf, Ross y Rusk
c.p. Hernández et al. 2014 p. 389).
Esto coincide con las sugerencias de Martínez Miguélez (2006) quien describe la
elección de la muestra en investigación cualitativa como intencional y “donde se
prioriza la profundidad sobre la amplitud” (p. 83).
Esta elección también destaca por carecer de criterios probabilísticos o
estadísticos pues su intención no es una generalización estadística de los pocos
participante al universo de poblacional.

El criterio final que determinó la elección en esta investigación fue el
siguiente: pueden participar aquellas personas mayores de 18 años de edad de la
ciudad de Caracas que se identifiquen a sí mismas como transgénero o
transexuales.
Esto responde a tres condiciones: deben ser consideraros en un sentido amplio
adultos capaces de dar su consentimiento informado, estar geográficamente
cercanos a los investigadores para facilitar la realización de entrevistas en
persona y además considerarse a sí mismo como personas trans.

Nos apoyamos, además, por un muestreo en cadena.
A través del contacto con un participante inicial se le preguntó si conocía a
otras personas de características similares que pudieran participar (Morgan c.p.
Hernández et  al. , 2014 p. 388).

Sobre la cantidad de participantes, Hernández et al. (2014) propone tres
factores para determinar su número: La capacidad operativa de recolección y
análisis; el entendimiento del fenómeno, o saturación de categorías; y la
naturaleza del fenómeno en cuanto a su frecuencia y accesibilidad (p. 384).
El número final de participantes en esta investigación es de 3 personas, 2
hombres transgénero en distintas etapas de transición hormonal y una mujer
transgénero sólo con operación de mamás y hormonación.

Consideramos este número como suficiente para obtener información sobre la
experiencia en la transición de su constitución de identidad y poder hacer un
análisis completo.
Durante el análisis preliminar los testimonios informaron
suficientes categorías en común como para considerar una saturación de
información.
Además, la baja cantidad de personas trans en la ciudad de Caracas
que los mismos participantes reportan hace difícil asegurar un número mayor de
unidades de análisis, atenuado también por el estigma social que les hace buscar
ser menos visibles y más reservados con su privacidad.

A continuación, se presenta en la tabla \ref{tab:participantes} algunas
características que definieron los criterios de selección de los
participantes y la numeración asignadas para identificarles en el capítulo
correspondiente al análisis y discusión de los resultados.

\begin{table}[]
    \caption{Descripción de los participantes}
    \label{tab:participantes}
    \centering
    \renewcommand{\tabcolsep}{1pt}
    \begin{tabularx}{\textwidth}{@{}ccXcXX@{}}
    \toprule
    Id & Sexo & Género         & Edad & Denominación & Ocupación
    \\ \midrule
    1            & Masculino      & Hombre a Mujer & 45   & Transgénero      & Abogada            \\
    2            & Femenino       & Mujer a hombre & 32   & Transgénero      & Lic. en Psicología \\
    3            & Femenino       & Mujer a hombre & 24   & Transgénero      & TSU Informático    \\ \bottomrule
    \end{tabularx}
\end{table}

\section{Estrategia de construcción de la información}
La estrategia seleccionada como principal fuente y forma de construcción de la
información para la investigación fue la entrevista.
Esta es definida por Janesick (c.p. Hernández et al, 2014) como “una reunión
para conversar e intercambiar información entre una persona (el entrevistador) y
otra (el entrevistado) […] en la entrevista se logra una comunicación y la
construcción conjunta de significados respecto a un tema” (p. 403).

Su propósito es una recolección de parte del entrevistador de lo que se construye
durante el encuentro. Hernández et al. (2014) caracteriza esta recolección como
“obtener las perspectivas y puntos de vista de los participantes (sus emociones,
prioridades, experiencias, significados y otros aspectos más bien subjetivos)”
(p. 8).
Para Martínez Miguélez (2006) se trata de un “diálogo coloquial” (p. 93) y por
ende de una estrategia mayormente verbal y experiencial.

Esta reunión y comunicación puede adquirir muchas formas y características.
En función del nivel de predeterminación de las preguntas, el nivel de libertad
que se le permite al entrevistado y la forma en la que se conduce la relación
entre los interlocutores.
En la clasificación de Hernández et al. (2014), la estrategia utilizada por
esta investigación es la entrevista semi-estructurada.
Esta se define como aquella entrevista en la que “el entrevistador tiene la
libertad de introducir preguntas adicionales para precisar conceptos u obtener
mayor información” (p. 403).

Estas entrevistas se llevaron a cabo siguiendo las sugerencias expresadas por
Martínez Miguélez (2006) y ajustadas a la caracterización que elabora Yin (2011)
sobre como se deben realizar entrevistas en investigación de estudio de casos
con enfoque cualitativo.
Más específicamente, la actitud del entrevistador fue la de una conversación
natural, manteniendo un ‘rapport’ con el entrevistado, dejando al participante
vocalizar sus propias prioridades conversacionales, evitando una expresión
excesivamente obvia de los juicios de valor del investigador y siguiendo un
protocolo de entrevista.

Este protocolo fue un guion elaborado antes de las entrevistas en base al
interés exploratorio de los investigadores, algunas experiencias previas con
personas trans y el conocimiento de algunas propuestas teóricas ya existentes
respecto a la vivencia de las personas trans.
Este guion inicial consistía de 41 preguntas de longitud variada clasificadas en
once (11) diferentes temáticas agrupadas a su vez en dos (2) ejes temáticos:
Identidad y transición.
En la tabla \ref{tab:protocolo} se puede observar cuales fueron estas categorías y
sub-categorías.
En los anexos se puede leer el texto de cada pregunta.
El protocolo no era otra cosa que una guía para el entrevistador pues tanto
entrevistado como entrevistador disponían de la libertad de seguir cualquier
línea de cuestionamiento relevante que surgiera en la conversación, dejando
espacio abierto para agregar nuevos temas, nuevas preguntas o profundizar en
algunos puntos novedosos.
En general, las entrevistas duraron entre una (1) y dos (2) horas cada una.

\begin{table}[]
\centering
\caption{Estructura del protocolo de entrevista semiestructurada}
\label{tab:protocolo}
\begin{tabularx}{\textwidth}{@{}rX@{}}
\toprule
Eje temático                & Tema                                        \\ \midrule
\multirow{7}{*}{Identidad}  & Identidad de género                         \\
                            & Ámbito laboral                              \\
                            & Devenir, familia                            \\
                            & Relaciones de pareja                        \\
                            & Expresión                                   \\
                            & Estatus legal                               \\
                            & Discriminación                              \\
\midrule
\multirow{4}{*}{Transición} & Disforia de género                          \\
                            & Concepción del cuerpo                       \\
                            & Significación de los procesos de transición \\
                            & Perspectiva a futuro                        \\
\bottomrule
\end{tabularx}
\end{table}

\section{Metodología de Análisis}
Desde el enfoque cualitativo la orientación de la investigación está dirigida a
la creación de teoría.
Martínez Miguélez (2006) propone que el producto final de la investigación
cualitativa es producir una teoría que cuenta del fenómeno investigado.
En palabras de Jones: “lo principal es generar una comprensión del problema de
investigación, en lugar de forzar los datos dentro de una lógica deductiva
derivada de categorías o suposiciones” (c.p. Strauss y Corbin, 2002, p. 23).

Lo que esto significa es que durante el análisis no intentaremos comprobar
hipótesis explicativas construidas a priori.
En su lugar, buscaremos la teoría explicativa contenida en la información
construida con los participantes.
El procedimiento que aplicaremos para ello es denominado por Yin (2011) como una
“generalización analítica, en la que una teoría previamente desarrollada es
usada como base con la cual comparar los resultados empíricos del caso de
estudio” (p. 31).
Este será el rol del marco referencial encontrado en el capítulo 3 de este
texto.
Es decir, el rol de la teoría previa es el de informar, proveer insumos, para el
análisis de los casos.
En contraste directo con las posturas deductivas propias del paradigma
neopositivistas que proponen una generalización estadística.

Este proceso de análisis está fundamentado en las propuestas del interaccionismo
interpretativo como es descrito por Denzin.
De esta tradición tomamos el concepto de triangulación múltiple como “la
combinación de múltiples métodos, múltiples tipos de datos, múltiples
observadores y múltiples teorías” (c.p. Martínez Miguélez, 2006, p. 128).
Es esta combinación de teorías anteriores y varias fuentes de información las
que permiten alcanzar la ‘construcción mejor informada y más completa
posible’ a la que hacen referencia Guba y Lincoln (2002) como producto final de
la investigación desde el paradigma socio-construccionista.

Para analizar la información producida a través de las entrevistas utilizamos
las estrategias de codificación expuestas por Strauss y Corbin (2002).
Para la fase inicial de aproximación a los datos utilizamos la codificación
abierta que es definidia como "el proceso analítico por medio del cual se
identifican los conceptos y se descubren en los datos sus propiedades y
dimensiones" (p.110).
Esta categorización inicial explora y recopila los elementos principales
de sentido y de comprensión presentes en las entrevistas con cada caso.

La forma en la que se procede, posteriormente, es clasificar los códigos
generados. Strauss y Corbin (2002) lo describen como un proceso de
comparación en el que se identifican las propiedades y dimensiones a las que
refieren los códigos para entender donde ubicar las categorías construídas.
Estas propiedades y dimensiones son seleccionadas a partir de la teoría previa.
En sus palabras: “La gente no inventa un mundo nuevo cada día sino que se
basa en lo que ya conoce para tratar de comprender lo desconocido” (p. 87).
Entonces al comparar se puede dar con una clasificación inicial de las
propiedades que permiten examinar el tema de investigación (p. 88).

El proceso descrito anteriormente es conceptualizado por los autores en la
siguiente cita:

\begin{quote}
    El caso específico proporciona guías (en cuanto a propiedades y
    dimensiones) para observar todos los casos, y permite a los
    investigadores pasar de la descripción a la conceptualización y de lo más
    específico a lo general o abstracto. (Strauss y Corbin, 2002, p. 97)
\end{quote}

A este paso es denominado por Strauss y Corbin (2002) como codificación axial.
El paso final del mismo es una comparación caso a caso de las categorías y
sub-categorías elaboradas y de la comparación entre los casos y la teoría.
Lo que estamos intentando entonces no es sólo analizar el incidente
específico sino el significado que subyace a este para dar cuenta del
fenómeno (p. 89).

Otros autores han tipificado este proceso de análisis utilizando una
terminología distinta pero refiriendo a los mismos pasos o similares.
Por ejemplo, Yin (2011) lo identifica como una deconstrucción y
reconstrucción de los datos.
En este se descomponen los datos construídos en sus elementos y propiedades
constitutivas para luego re-contextualizarlos en una nueva construcción teórica.
Esto es similar al círculo hermenéutico descrito por Gadamer, citado por
Martínez Miguélez (2006), quien luego lo utiliza para fundamentar el proceso
de Categorización y contrastación que propone como base para el análisis en
el enfoque cualitativo.

El análisis de los datos que componen esta investigacion se podrá leer en el
capítulo \ref{analisis}.
En ese capítulo realizamos un análisis de las entrevistas realizadas
etiquetando los distintos fragmentos comunicacionales en función de la
información con la que estos se relacionan y ubicando los temas recurrentes
en todas las entrevistas para agruparles.
De este análsis inicial se derivaron seis (6) categorías que agrupan quince
(15) sub-categorías.
En la tabla \ref{tab:categorias} se puede observar un resumen de estas
categorías.

\begin{table}[h]
\centering
\caption{Categorías y sub-categorías}
\label{tab:categorias}
\begin{tabularx}{\textwidth}{@{}XX@{}}
\toprule
Categorias                           & Sub-categorias      \\ \midrule
\multirow{2}{*}{Desarrollo}          & Infancia y Pubertad \\
                                     & Familia             \\ \midrule
\multirow{3}{*}{Género y sexualidad} & Identidad           \\
                                     & Hegemonía de género \\
                                     & Orientación sexual  \\ \midrule
\multirow{3}{*}{Discriminación}      & Rechazo             \\
                                     & Acoso- Bullying     \\
                                     & Consecuencias       \\ \midrule
\multirow{3}{*}{Transición}          & Transitar           \\
                                     & Apariencia física   \\
                                     & Conocimiento        \\ \midrule
\multirow{2}{*}{Genitalidad}         & Lo esencial         \\
                                     & Lo que se tiene     \\ \midrule
\multirow{2}{*}{El otro}             & Como me ven         \\
                                     & Aspecto legal       \\ \bottomrule
\end{tabularx}
\end{table}

Esta clasificación y etiquetado fue realizado utilizando como herramienta el
programa informático ATLAS.ti en su Version 7.5.7 (2012).

El siguiente paso fue la vinculación de los códigos elaborados con los
planteamientos teóricos para intentar aproximarnos a las explicaciones mejor
relacionadas con el contenido de las entrevistas.

\section{Ética de la investigación}
% TODO lectura crítica y redacción de sección de ética

En su obra, \emph{Ética}, Adolfo Sánchez (1984) caracteriza este
concepto como el pensamiento filosófico y reflexión en torno a los aspectos
teóricos de la moral.
Es decir, al análisis y pensamiento alrededor de las normas sociales que son
asumidas y seguidas íntimamente por los grupos humanos.
Por su definición se trata de un análisis del sentido general de un
\emph{deber ser}, en contraposición del análisis particular de los dilemas
morales.
De esta manera lo que ubica la ética es la naturaleza de lo que se considera
bueno, valioso y que por lo tanto vale la pena intentar alcanzar.
Ya sea como individuos, comunidad, sociedad o grupo humano.

Dentro de esta misma línea lógica, Sara Fuentes (2006) diferencia entre dos
formas relacionadas pero distintas de la ética.
Por un lado la ética general, y por el otro la ética profesional.
Esta última como una expresión o parte de la ética general que atiende
específicamente lo que se considera bueno, de valor y justo en la práctica de
los profesionales.
A esta también se le denomina a veces como deontología, o ideal de la
práctica profesional.

La práctica de la psicología en el contexto Venezolano es regído por un marco
legal sencillo.
La \emph{Ley de ejercicio de la psicología} promulgada en el año 1978, hace
énfasis en la obligación de la práctica de la psicología en función de un
código ético o deontológico que es establecido por la Federación de
psicólogos de Venezuela.
Este código sancionado en el año 1981 dedica su primer capítulo a las
responsabilidades de la investigación en el ámbito de la psicología.

Sin embargo, hace falta contextualizar lo estipulado por este código y por
otros principios de la ética profesional en el marco de la presente
investigación.
Chavarría (2001) caracteriza el análisis de la ética de las ciencias humanas
como responder a las preguntas “¿Para qué y para quién hacemos ciencia?” (p.
33).
Este autor parte de Foucault para poner en duda un sentido tradicional de la
ética científica como una búsqueda de la verdad absoluta para el bienestar de
la humanidad en su totalidad.
Visión que comparte el código de ética venezolano.
Propone Chavarría como contrapeso una visión crítica y compleja de la
realidad.

Esta postura se ajusta filosófica y epistemológicamente de mejor manera con las
posturas del enfoque cualitativo y el paradigma socio-construccionista que
plantea esta investigación.
Para llevar adelante la exposición del lugar desde el cuál planteamos nuestra
postura ética analizamos a continuación, punto por punto, los componentes de
una ética profesional en investigación social.
De la misma manera como planteamos nuestra postura paradigmática.
El modelo que seguiremos está adaptado de la interpretación realizada por
González Ávila (2002) de las propuestas de la bioética en su deconstrucción de
una ética para la investigación cualitativa.

El modelo de la bioética del cual partimos fue planteado originalmente por
Ezekiel Emanuel (c.p. González, 2002).
Este contiene siete puntos ha tener en cuenta cuando se hace investigación
con seres humanos.
Si bien este modelo es planteado en principio para el contexto de las
investigaciones médicas, su propuesta teórica-ética contiene principios
aplicables para la investigación cualitativa.

En el análisis realizado por González Ávila (2002),
estos principios son extendidos, adaptándolos para la investigación
cualitativa, e incluyendo un punto adicional sobre la cualidad del diálogo entre
investigador y participantes.
Esto a partir del razonamiento de que “los aspectos éticos que son aplicables
a la ciencia en general lo son también a la investigación cualitativa”
(González, 2002, p. 94).

Hemos agrupado estos puntos en tres aspectos centrales: el valor social de la
investigación, la validez científica y la relación de los investigadores con
los participantes.
Sobre cada uno de estos aspectos, además, comentaremos las precisiones a las
que haya lugar en torno a posturas y perspectivas de intepretación alternativa.

\subsection{Valor social}
Como valor social entendemos la relevancia e importancia que puede tener el
aporte de una investigación para solventar un problemas social.
El conocimiento científico debe ser entendido, en este contexto, no como
una busqueda objetiva de verdad, sino como una búsqueda de mejoras para la
sociedad.
Sobre este punto González Ávila (2002) plantea que: “La ciencia es importante
porque, entre otras cosas, cambia la forma en la que la gente ve y vive en el
mundo, aunque sea mediante imágenes e inspiraciones.” (p. 90).
Esto es a lo que refiere Chavarría (2001) cuando deconstruye el discurso
Foucaultiano alrededor de los regímenes de poder.

La ciencia y los científicos se han constituído en un regímen de poder que es
capaz de instaurar lo que socialmente se considera como verdad.
En este caso, la verdad científica.

Por ello compartimos la crítica que articula Chavarría (2001) al decir que ha
existido hasta ahora una noción de la \emph{ciencia al servicio del hombre}
que ha privilegiado a unos grupos por encima de otros.
Algunos grupos sociales, entre ellos la diversidad sexual, han sido
invisibilizados y segregados históricamente del discurso científico.
Si bien la ciencia había perseguido un beneficio para \emph{toda la humanidad}.
Esa misma \emph{humanidad} había sido compuesta de grupos privilegiados
dejando de lado minorias.

Por ello rescatamos la propuesta:

\begin{quote}
    Si se ha homogenizado y promovido los intereses que retrasan la construcción de
    una sociedad solidaria, corresponde hoy, en contrapartida, investigar para
    liberar las potencialidades sociales que tienden hacia la construcción de la
    solidaridad. (Chavarría, 2001, p. 36)
\end{quote}

Mediante la realización de una investigacion que busca la mirada,
la voz y la perspectiva de las personas trans, tal y como estas la
manifiestan y reconociendo la participación interpretativa de nosotros como
investigadores, esperamos aportar a la introducción al discurso científico de
un aporte que reivindique el valor humano de la población trans.
Es así que respondemos a una necesidad de visibilizar la experiencia trans.
No solamente por una búsqueda de avance del conocimiento científico.
Sino como un acto crítico de reivindicación.

En ese sentido consideramos esta investigación posee relevancia social al ser
coherente con la necesidad de normalizar socialmente la experiencia de la
transición en los términos de las mismas personas trans.

También reforzamos la importancia del enfoque cualitativo y
socio-construccionista en esta tarea.
Chavarría (2001) articula la noción de que los aspectos epistemológicos
tienen consecuencias éticas.
La forma en la que realizamos investigación científica tiene una relevancia
ética.

Existen tres tareas morales fundamentales para el investigador según
Chavarría (2001).
Estas son: facilitar la emergencia de la subjetividad compleja;
superar la falsa diada tradicional subjetividad-objetividad;
y propiciar en sí mismo y en los informantes la crítica y reconstrucción de
la subjetividad.

Creemos que la propuesta del enfoque cualitativo desde el paradigma
socio-construccionista es la apuesta metodológica más apropiada para dar
cumplimiento a estas tareas morales y éticas.

\subsection{Validez científica}
Bajo el termino de validez científica agruparemos cuatro de los elementos del
modelo bioético adaptado por González Ávila (2002).
La validez de la metodología de trabajo a utilizar.
En esta investigación vendría a ser el estudio de caso;
El marco referencial suficiente;
La calidad del informe;
Y la evaluación independiente de la investigación.

    \subsubsection{Validez en estudios de casos}
La validez y su hermana cercana, la confiabilidad, se refieren a la capacidad
de la metodología seleccionada para alcanzar sus propios objetivos y de
hacerlo de manera consistente.
Desde la mirada de Yin (1994), la validez del estudio de casos puede ser
planteada en tres aspectos.

Primero, la validez de constructo.
Esta requiere del establecimiento y uso apropiado de las operaciones que
correspondan para capturar el concepto que está siendo estudiado (Yin, 1994, p.
33).
En nuestro caso, capturar la vivencia de la transición en las personas trans.
Para ello la entrevista aporta de una profundidad y complejidad que otros
métodos no pueden aportar.
Especialmente desde el enfoque cualitativo.

En segundo lugar se encuentra la validez interna.
En este aspecto la importancia está en el poder de la metodología para
establecer relaciones causales y diferenciarla de relaciones espurias
(\emph(ibid.)).
Estando el objetivo relacionado con una experiencia subjetiva, una de las
estrategias de comprobación a utilizar será el repreguntar a los participantes.
En esta estrategia, dentro del enfoque cualitativo, se consultan las
conclusiones de los investigadores con los participantes, cuando estos están
disponibles, para que comenten acerca de la integridad de las conclusiones de
la investigación.
De esta manera se asegura no estar representando erróneamente a la comunidad
científica en general las vivencias de los participantes.

El tercer aspecto es la validez externa.
Esta se refiere al establecimiento de el dominio preciso al cual aplican las
generalizaciones de la investigación (\emph(ibid.)).
Como hemos comentado en este mismo capítulo, desde el enfoque metodológico
utilizado las generalizaciones no son de naturaleza estadística.
Se trata por el contrario de generalizaciones teóricas.
Es decir, la aplicabilidad del conocimiento construído no tiene como fin
explicar a la totalidad de una población sino de complejizar el conocimiento
acerca de los fenómenos que interceptan la vivencia de los participantes.
En nuestro caso son conceptos como el género, la transición, la identidad y el
cuerpo.

En la misma línea de razonamiento Yin (1994) define la confiabilidad como la
demostración de que las operaciones del estudio pueden ser repetidos
brindando los mismos resultados.
Este aspecto tiene sus propios retos particulares para la investigación con
las experiencias subjetivas humanas pues estas en su definición no se repiten
\emph{exactamente de la misma manera} siempre.
Adicionalmente, las interpretaciones, expresiones y narrativas expresadas
durante la entrevista son producto de una reconstrucción puntual que cambia
con el tiempo a medida que los participantes viven nuevas experiencias.

En este aspecto Martínez Miguélez (2006, p. 283) sugiere que existen dos
elementos que proveen de solidez al enfoque cualitativo.
El primero es el uso intensivo de la triangulación.
Mediante la presentación exhaustiva de los contenidos teóricos que guían la
interpretación de los investigadores y el origen de los datos con los que se
extrapolan.
Por ejemplo, protocolos de entrevista, grabaciones, transcripciones y
referencias bibliográficas.
Además, una observación de parte de entes idenpendientes que velen por el uso
apropiado de la metodología de investigación.

Sobre estos dos puntos elaboraremos en las siguiente secciones sobre el marco
referencial suficiente y la evaluación independiente.

    \subsubsection{Marco referencial suficiente}
Desde el modelo de investigación de la bioética, el propósito de tener un
marco referencial suficiente está en asegurar la calidad de la comprensión
del estado actual del conocimiento científico respecto al tema a investigar.
En el contexto de la investigación cualitativa esto también asegura los
mejores insumos posibles para complementar el proceso de triangulación
(González Ávila, 2002; Martínez Miguélez, 2006).
De esta manera, se fortalece la búsqueda de aquella construcción mejor
informada y más completa a la que hace referencia Guba y Lincoln (2002).

En el capítulo \ref{ch:marcoreferencial}, esperamos haber llevado a cabo esta
tarea de manera satisfactoria.
Los elementos contenidos en este marco referencial incluyen aquellos
conceptos que utilizaremos para el análisis e interpretación de las
construcciones elaboradas junto con los participantes.
Esperamos que la variadad de conceptos y variedad de puntos de vista
presentados sea suficiente para dar cuenta del estado actual de la
comprensión científica desde las ciencias sociales de la transexualidad,
transgenerismo y género.

Este esfuerzo forma parte de las operaciones que garantizan una validez a las
interpretaciones y conclusiones alcanzadas en la investigación

    \subsubsection{La calidad del informe}
El informe de investigación desde la mirada de González Ávila debe ser
presentado con un lenguaje seleccionado cuidadosamente, reflejando la mayor
cualidad de los valores de la tradición de comunicación científica en cuanto
estructura y estilo.

Una vez mas esperamos que este informe cumpla con estas características.
Hemos seleccionado deliberadamente el lenguaje y el estilo de  presentación en
función de la tradición de nuestra \emph{alma mater}, la Universidad Cental
de Venzeuela, nuestra escuela de psicología, y el estilo de presentación de
informes de la APA en su 6ª edición.

    \subsubsection{Evaluación independiente de la investigación}
La evaluación independiente, es decir, la observación de la investigación por
parte de un agente no afiliado a la misma, es una de las garantías éticas.
Un observador externo en las etapas previas, durante y después de una
investigacion, permite a los investigadores mantener la vigilancia ética
sobre las acciones e interacciones que tienen en relación con sus
participantes.

Esta investigación ha tenido la obsevación de tres agentes independientes:
Primero de parte del departamento de psicología social, quienes siguieron la
formulación de la investigacion en su etapa de anteproyecto y aprobaron su
realización;
Segundo de parte del tutor académico, quien aporta además su experiencia y
trayectoria profesional a la nuestra como investigadores relativamente
noveles durante todas las etapas de desarrollo de la investigación;
Y en tercer lugar la evaluacion final realizada por un jurado de
profesionales de la psicología quienes determinarán en su momento cuan
apropiada es esta investigacion en cuanto a relevancia, calidad, validez,
ética, etc.

\subsection{Relación con los participantes}
Los otras cuatro elementos de la bioética los agruparemos en esta sección
sobre la relación con los participantes.
En primer lugar consideraremos la selección de los participantes.

En este aspecto los elementos que se deben tomar en cuenta son: la selección
de participantes relevantes y una proporción favorable riesgo-beneficio
dentre de la investigación.
En este sentido hemos hecho una selección basada en la relevancia de los
participantes para el tema que estamos investigando.
Nuestros participantes son personas transgénero en distintos grados de
transición.
Además, esta investigación reporta pocos riegos para los participantes ya que
no se llevó a cabo ningún procedimiento dañino.
También se han anonimizado las intervenciones en las entrevistas para evitar
una exposición al escarnio público o compromiso social que pueda resultar de
la divulgación de las identidades de los participantes.

En segundo lugar está la importancia del consentimiento informado.
El consentimiento informado es definido por González Ávila (2002) como
“provisión de información sobre la finalidad, los riesgos, los beneficios y
las alternativas a la investigación –y en la investigación–, la comprensión
del sujeto de esta información y de su propia situación, y la toma de una
decisión libre, no forzada sobre si es conveniente participar o no” (p. 101)

En el anexo \ref{ch:consentimiento} se puede observar el modelo de
consentimiento informado utilizado para asentar los acuerdos con los
participantes.
Fundamentado en el código de ética de la psicología en Venezuela y en los
principios éticos desarrollados en esta sección, busca asegurar que los
participantes entienden las implicaciones de su participación y las garantías
a las que tienen acceso.

En tercer lugar se encuentra el respeto a los participantes.
Este es operacionalizado por González Ávila (2002) en al menos cuatro
diferentes acciones de parte del investigador.

\begin{enumerate}
    \item Derecho a cambiar de opinion: de parte de los participantes.
    \item Confidencialidad.
    \item Devolución de hallazgos.
    \item Reconocimiento al aporte de los participantes.
\end{enumerate}

