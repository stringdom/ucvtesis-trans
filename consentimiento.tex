%%
%% Author: Stringdom
%% 1/5/2018
%%

\chapter{Consentimiento informado}

\small

Yo\_\_\_\_\_\_\_\_\_\_\_\_\_\_\_\_\_, mayor de edad, CI: \_\_\_\_\_\_\_\_\_ acepto
participar de manera voluntaria en la realización de una entrevista a
profundidad, así como a proporcionar la información necesaria para el trabajo de
investigación que realiza el Bachiller Leonardo Perez C.I. V-18.185.772 y
Emerson Yancul C.I. E-82.278.590, como parte de su trabajo de grado
obligatorio, el cual se llevará a cabo tomando en cuenta el Código de Ética del
Psicólogo y bajo la debida asesoría de la Lic. Luisana Gómez, profesora del
Departamento de Psicología Social de la Escuela de Psicología de la Universidad
Central de Venezuela.

Autorizo la categorización de mis respuestas, mientras sea garantizada la
confidencialidad entendiéndose por el derecho de mantener la información
recolectada bajo los estándares académicos, en cumplimiento con las normas de
ética que rigen el ejercicio en Psicología, en los artículos:

\begin{itemize}
    \item Artículo 55: la investigación en Psicología deberá ser realizada y
    supervisada por personas técnicamente entrenadas y científicamente
    calificadas.
    \item Artículo 57: Para proteger la integridad física y mental de la
    persona, se deben cumplir los siguientes requisitos: a) Toda persona debe
    expresar con absoluta libertad su voluntad de aceptar o rechazar su
    condición de sujeto de experimentación. b) Debe tener la facultad de
    suspender la experiencia en cualquier momento. c) Debe estar suficientemente
    informado acerca de la naturaleza, fines y consecuencias que pudieran
    esperarse de la evaluación, excepto en aquellos casos en que la información
    pudiera alterar los resultados de los mismos.
\end{itemize}

\centering

Caracas, \_\_ del mes \_\_\_\_\_\_\_\_ de \_\_\_\_

\vfill

\_\_\_\_\_\_\_\_\_\_\_\_\_\_

Firma del participantes