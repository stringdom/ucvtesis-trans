\section{Planteamiento del problema}

La transexualidad es una temática que nunca ha sido investigada en la escuela de
psicología. Esto incluye a la mención de psicología social, en donde la temática
apenas ha sido tocada tangencialmente en algunas elaboraciones sobre el género.

Actualmente la temática que parece dominar el discurso desde la psicología
social, en otras partes del mundo, es el asunto de la patologización de la
transexualidad. En particular la paradoja que se encuentra en conjunto con los
requerimientos legales para realizar la transición de sexo. La situación actual
en Venezuela se desconoce. Por ello decidimos realizar un estudio de las
percepciones, imaginarios y creencias que existen alrededor de la transexualidad
en relación con el acceso legal a los procedimientos de transición y la
patologización de la transexualidad. Para ello se parte de dos artículos
recientes. El primero trata de un planteamiento realizado por
\textcite{Coll-Planas2015} en el que visibilizan el uso estratégico que es dado
por parte de las personas transexuales de la patologización como medio para
acceder a los procedimientos de transición de sexo.

En su revisión del contexto médico y jurídico de España encuentran que, la
perspectiva de los servicios que ofrecen los procedimientos de transición de
sexo sobre la transexualidad, conlleva una carga negativa, binarista,
biologicista y patologizante que obliga, en el caso de los transexuales
masculinos, a una reproducción exagerada de los estereotipos de la masculinidad
hegemónica.

El segundo es la investigación realizada en Bogotá, Colombia, por
\textcite{LassoBaez2014}. En la misma se realizaron varias entrevistas a
profesionales psicólogos y psiquiatras, así como a personas trans, acerca del
tema de la transición y los servicios médicos utilizados para realizar la misma.
Uno de los aspectos resaltantes es que, desde su punto de vista, se hizo
necesario consultar tanto a las personas que acuden a los servicios de salud
como a los psicólogos y un psiquiatra que atiende a estos usuarios. Esto los
lleva a tener dos puntos de vista alrededor de una sola realidad que pueden ser
contrastados en función de los conceptos centrales. Aunque se menciona el papel
y la relación que tienen otros actores, como enfermeras y personal
administrativo, sólo se toma en cuenta la construcción que hacen de ellos los
entrevistados.

Allí encontraron, mediante un análisis crítico del discurso como es propuesto
por VanDijk (citado en \cite[][p. 111]{LassoBaez2014}), que una de las fuentes más
fuertes de discriminación, en la forma de micro agresiones y trato prejuicioso,
provienen del personal auxiliar de los servicios de salud.

En menor medida, entre los profesionales de atención, donde se presenta con más
frecuencia el desconocimiento de la transexualidad y falta de estrategias para
proveer atención de calidad a esta población. Se encontraron, además, con la
patologización de la transexualidad con una doble función paradójica. Pues
permite, simultáneamente, el acceso a los servicios de transición, así como un
elemento de discriminación y vulneración de la identidad y expresión de la
persona trans. Ambas investigaciones apuntan a un rol preponderante de la
patologización en algunos aspectos de la construcción de la identidad transexual
en la cultura occidental.

En particular cuando se desea transitar de un sexo a otro mediante
procedimientos seguros ya que existen muchos riesgos asociados a esta
transición. Por esto las personas transexuales se ven forzadas a lidiar con las
visiones hegemónicas de la medicina sobre el sexo y el género para poder acceder
a estos procedimientos. Esta situación es merecedora de atención en el contexto
venezolano donde parecen prevalecer con fuerza los estereotipos y no se ha
conducido investigación desde la psicología social para comprender el fenómeno.

\todo{Son las preguntas de investigación viejas}
Así que en función de orientar la investigación y tomando en cuenta los
elementos expuestos anteriormente, se plantean las siguientes preguntas de
investigación: ¿Cuál es el lugar de la patologización de la transexualidad en la
construcción de la identidad de género de los transexuales con edades
comprendidas entre 18 y 30 años que hacen vida en la ciudad de Caracas?, ¿cómo
acceden, utilizan y se insertan en los servicios médicos de transición de sexo
estas personas?, ¿qué lugar ocupa la transición de sexo en su vida cotidiana?