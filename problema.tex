\section{Planteamiento del problema}

La transexualidad y el transgenerísmo son temáticas que nunca ha sido
investigada en la escuela de psicología. Esto incluye a la mención de psicología
social, en donde la temática apenas ha sido tocada tangencialmente en algunas
elaboraciones sobre el género.

Actualmente la temática que parece dominar el discurso desde la psicología
social, en otras partes del mundo, es el asunto de la patologización de la
transexualidad\footnote{La transexualidad suele ser el único fenómeno estudiado
o mencionado debido a la prevalencia del discurso biologicista y médico en la
psiquiatría. Este considera innecesario conversar sobre el transgenerísmo bajo
la concepción de que en este no existe una ‘disforía de género’. Al no haber
patología no hay necesidad de acción. Sin embargo, se ignora que el
transgenerismo puede ser una etapa previa a la transexualidad si se dan los
procedimientos de transición necesarios para cumplir con la clasificación.
Revisar los conceptos relevantes en la página~\pageref{diferencia}}. En
particular la paradoja que se encuentra en conjunto con los requerimientos
legales para realizar la transición de sexo. La situación actual en Venezuela se
desconoce. Por ello decidimos realizar un estudio de las percepciones,
imaginarios y concepciones que existen alrededor del transgenerismo en
particular en relación a los procedimientos de transición de un sexo a otro.
Para ello partimos de dos artículos recientes. El primero trata de un
planteamiento realizado por \textcite{Coll-Planas2015} en el que visibilizan el
uso estratégico que es dado por parte de las personas transexuales de la
patologización como medio para acceder a los procedimientos de transición.

En su revisión del contexto médico y jurídico de España encuentran que, la
perspectiva de los servicios que ofrecen los procedimientos de transición de
sexo sobre la transexualidad, conlleva una carga negativa, binarista,
biologicista y patologizante que obliga, en el caso de los transexuales
masculinos, a una reproducción exagerada de los estereotipos de la masculinidad
hegemónica. Es decir, debido a que:

\begin{quote}
[…] se refuerza la idea de que lo natural y lo sano es que el sexo (nacer en un
cuerpo de macho o de hembra,) la identidad de género (sentirse hombre o
mujer) y el rol o la expresión de género (actuar de forma masculina o
femenina) estén articulados de un modo normativo \parencite[][p.
426]{Coll-Planas2015}.
\end{quote}

Se hace entonces necesario que la persona transgénero reproduzca los roles de
género de forma incluso más estricta y punitiva que lo que se esperaría de una
persona no trans.

Este punto nos hace pensar que existe una presión estructural social sobre la
persona trans para alterar su expresión de género y proyectar, no de la forma en
que ellos lo harían espontáneamente en otra circunstancia, sino de la forma que
se les exige para poder acceder a la transición. \textcite{Coll-Planas2015}
explican que:

\begin{quote}
[…] desde el modelo hegemónico[, binarismo masculino y femenino,] se pone
énfasis en normalizar a las personas trans para que reproduzcan un género
normativo, lo que incluye la presión para la transformación corporal ya que […]
la modificación de las características sexuales es crucial para fundar la
masculinidad y la feminidad en una base biológica \parencite[][p.
427]{Coll-Planas2015}.
\end{quote}

El segundo artículo consultado es una investigación realizada en Bogotá,
Colombia, por \textcite{LassoBaez2014}. En la misma se realizaron varias
entrevistas a profesionales psicólogos y psiquiatras, así como a personas trans,
acerca del tema de la transición y los servicios médicos utilizados para
realizar la misma. Uno de los aspectos resaltantes es que, desde su punto de
vista, se hizo necesario consultar tanto a las personas que acuden a los
servicios de salud como a los psicólogos y un psiquiatra que atiende a estos
usuarios. Esto los lleva a tener dos puntos de vista alrededor de una sola
realidad que pueden ser contrastados en función de los conceptos centrales.
Aunque se menciona el papel y la relación que tienen otros actores, como
enfermeras y personal administrativo, sólo se toma en cuenta la construcción que
hacen de ellos los entrevistados.

Allí encontraron, mediante un análisis crítico del discurso como es propuesto
por VanDijk (citado en~\cite[][p. 111]{LassoBaez2014}), que una de las fuentes
más fuertes de discriminación, en la forma de micro agresiones y trato
prejuicioso, provienen del personal auxiliar de los servicios de salud. En menor
medida, entre los profesionales de atención es donde se presenta con más
frecuencia el desconocimiento de la transexualidad y falta de estrategias para
proveer atención de calidad a esta población. Encontraron, además, con la
patologización de la transexualidad con una doble función paradójica. Pues
permite, simultáneamente, el acceso a los servicios de transición, así como un
elemento de discriminación y vulneración de la identidad y expresión de la
persona trans.

Esto ubica un poder sobrevalorado a la opinión clínica de los médicos.
\textcite{LassoBaez2014} señala: “en la práctica psiquiátrica es común que el
certificado de \emph{disforia de género} sea negado cuando la experiencia de la
persona no responde a los criterios de feminidad y masculinidad del profesional”
(p. 122). Por esta razón, lo que termina sucediendo es que se fuerzan los
estereotipos de género sobre la identidad de la persona trans que requiere
atención médica. Es por ello que Lasso Báez concluye que “Esta situación lleva a
muchas personas […] a mentir en las entrevistas psiquiátricas, exagerando
ciertos rasgos e hipervigilando su comportamiento para ajustarse a estereotipos
de género” (p. 122).

Ambas investigaciones apuntan a un rol preponderante de la
transición en la construcción de la identidad transexual y transgénero.
En particular cuando se desea transitar de un sexo a otro mediante
procedimientos seguros ya que existen muchos riesgos asociados a esta
transición. Por esto las personas transgénero se ven forzadas a lidiar con las
visiones hegemónicas de la medicina sobre el sexo y el género para poder acceder
a estos procedimientos. Siendo los principales métodos: a) la terapia de
reemplazo hormonal, que requiere la prescripción de una toma de hormonas de
parte de un especialista endocrinólogo; b) la mamoplastia o cirugía de senos, ya
sea mamoplastia de aumento o mastectomía, extirpación de la glandula mamaria y
reducción del volumen del seno; c) la cirugía de reasignación de sexo, en la
cual se transforman los genitales existentes para ajustarlos de un sexo u otro.

Esta situación es merecedora de atención en el contexto venezolano donde parecen
prevalecer con fuerza los estereotipos de género machistas y no se ha conducido
investigación desde la psicología social para comprender el fenómeno del
transgenerísmo y la transexualidad en nuestra cultura.

En función de abordar este vacío, realizamos la presente investigación que busca
dar respuesta a la pregunta planteada en la página~\pageref{preguntas}.

	
%¿Cuál es el lugar de la transición de sexo en la construcción de la identidad
%de género de los trans de Caracas?

\section{Objetivos}


\subsection{Objetivo General}
	\begin{itemize}
		\item Comprender la transición y la construcción de la identidad de las
		personas Transgénero que hacen vida en el Área Metropolitana de Caracas.
	\end{itemize}

\subsection{Objetivos específicos}
	\begin{itemize}
		\item Identificar los elementos centrales de la constitución de la identidad
	de género de un grupo de personas trans masculinos de Caracas.
		\item Identificar los elementos centrales de la constitución de la identidad
	de género de un grupo de personas trans femeninos de Caracas.
		\item Identificar las modalidades y estrategias que utilizan las personas
	transexuales de Caracas para acceder a los procedimientos de transición de sexo.
	\end{itemize}
