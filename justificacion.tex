\section{Justificación}

El interés y la preocupación social por el acoso, invisibilización y
prejuicio hacia una persona transgénero va en aumento.
Según Fernández et al.\ (2014) “este hecho se debe a actos graves tales como
depresión, ataques y suicidios de las partes afectadas a consecuencia de sufrir
acoso por parte de sus compañeros o incluso de su propia familia” (p. 32).

Para poder adentrarse dentro de lo que significa la transexualidad o el
transgenerísmo es necesario cimentar el interés de la presente investigación en
base a la noción de que la construcción de la identidad es un hecho influenciado
por factores psicosociales. Esto es planteado por autores como Paramo (2008) que
expresa que los individuos sufren una fuerte influencia del medio en el que se
desarrollan al momento de construir su identidad. Hecho que no es consciente,
sino un resultado de interacciones, es decir, un individuo que se vea rodeado o
se desenvuelva en una cultura en la que el hombre tiene unas características
particulares va a asumir estas como elementos constitutivos pero también su
propia interacción con otros miembros de su comunidad podrían afectar el tipo de
hombre o mujer que llegue a ser.

 Existe una progresiva concientización sobre la importancia de los derechos
 humanos y sobre la educación acerca de este tema en espacios tan vulnerables
 como el colegio, la universidad, transporte público, lugar de trabajo, entre
 otros (Fernández et al., 2014).

Investigaciones sobre las consecuencias, incidencias, factores, características
y todo tipo de variables del fenómeno \emph{bullying} han incrementado con el
fin de conocerlo mejor para la elaboración de instrumentos de evaluación y su
respectiva intervención. Estos tipos de acosos son generalizados, es decir,
afectan indiferentemente del país o clases sociales. “A menudo se asocia de
manera exclusiva el transgenerísmo con el mundo adulto como si la identidad de
género fuera únicamente fruto de un largo proceso de introspección personal
resuelto al cabo de muchos años” (Fernández et al., 2014 p. 32).

Éste mismo autor, hace referencia a que la realización de una intervención por
parte de los asistentes sanitarios correspondientes es de vital importancia,
pues “se debe preparar y enseñar a la sociedad conceptos éticos y morales,
enseñar a las personas a actuar delante de situaciones de acoso, burla o abuso y
a intervenir delante del primer síntoma” (Fernández et al., 2014, p. 34).

El transgénerismo según Mejía y Almanza (2010) “se manifiesta típicamente en la
medicina y la psiquiatría por una identificación intensa y persistente con el
otro sexo, con un sentimiento de inadecuación con el sexo asignado, y por un
deseo permanente de vestir, vivir y ser tratado como miembro del otro sexo” (p.
91). Las personas transgénero consideran que han nacido en un sexo equivocado.
La mayoría, refiere el inicio de los síntomas desde la primera infancia.

El concepto de género nació para designar todo aquello que es construido por las
sociedades para estructurar y ordenar las relaciones sociales. Al basarse estas
construcciones sociales y simbólicas en la diferencia sexual, se estructuran
relaciones de poder cuya característica esencial es el dominio masculino. La
dominación masculina se fundamenta en la diferencia sexual, la cual se explica
por el diferente lugar que ocupa cada sexo en el proceso de reproducción, idea
también del pensamiento judeocristiano. Se instaura así la lógica del género,
que parte de una oposición binaria: lo propio del hombre y lo propio de la mujer
(lo esencial en la feminidad y la masculinidad), y dicha lógica del género es
una lógica de poder, de la dominación del sistema patriarcal (Mejía y Almanza,
2010).

La persona transgénero es mal vista en la sociedad. Debido a la discriminación
dirigida hacia los transexuales y transgénero este grupo de personas
experimentan sentimientos de vergüenza de que se descubra su identidad de
género. Además deben enfrentar circunstancias en las que son calificados como
mentalmente enfermos. Dar pasos para expresarse de acuerdo a su identidad les
ocasiona sentimientos de culpabilidad en función del rechazo social. Deben
ocultar su verdadera identidad y en ocasiones incluso se les exija que esté de
acuerdo con los otros y pretenda condenar sus propios intereses y actividades,
como en los rituales de reafirmación del rol de género.

Esta doble vida de la persona transgénero, antes de decidir presentarse como su
identidad de género, es en ocasiones llevada con miedo, puede ser difícil de
mantener y también puede conducir a trastornos mentales como la depresión. El
transgénerismo es una condición fundamental en la construcción de la
personalidad de la persona trans, y su expresión permitirá el ingreso o no a
diversos campos dependiendo de los capitales con los que cuente. Pero las
actitudes de las demás personas hacia esta condición crean una situación de
tensión que también afecta del desarrollo de la personalidad y puede también
impedir la integración efectiva de las personas trans en la comunidad.

Una gran proporción de personas transgénero son incapaces de adaptarse a las
presiones y se convierten en bajas sociales. Estos son los Trans que se
encuentran con mayor frecuencia en las calles prostituyéndose. Ahora que se
estableció que los transgénero no son fisiológicamente diferentes, también
existe la posibilidad que poco a poco la sociedad sea más tolerable en cuanto a
su respeto y no discriminación (Sáez, 2006).
