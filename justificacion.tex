\section{Justificación}

El interés y la preocupación social por el acoso, invisibilización y
prejuicio hacia una persona transgénero va en aumento.
Según Fernández (2014)“este hecho se debe a actos graves tales como
depresión, ataques y suicidios de las partes afectadas a consecuencia de sufrir
acoso por parte de sus compañeros o incluso de su propia familia” (p. 32).

Para poder adentrarse dentro de lo que significa la transexualidad o el transgénerismo es necesario cimentar el interés de la presente investigación con base a un elemento particularmente importante y es el ver a la construcción de la identidad como un hecho influenciado por factores psicosociales. Esto es planteado por autores como Paramo (2008) que expresa que los individuos sufren una fuerte influencia del medio en el que se desarrollan al momento de construir su identidad (hecho que no es consciente, sino un resultado de interacciones), es decir, un individuo que se vea rodeado o se desenvuelva en una cultura en la que el hombre tiene unas características particulares va a asumir estas como elementos constitutivos (bien sea hombre o mujer) pero también su propia interacción con otros miembros de su comunidad podrían afectar el tipo de hombre o mujer que llegue a ser.

 Existe una progresiva concienciación sobre la importancia de los derechos humanos y sobre la educación acerca de este tema en espacios tan vulnerables como el colegio, la universidad, transporte público, lugar de trabajo, entre otros (Fernández, 2014).

Investigaciones sobre las consecuencias, incidencias, factores, características y todo tipo de variables del fenómeno bullying han incrementado con el fin de conocerlo mejor para la elaboración de instrumentos de evaluación y su respectiva intervención. Estos tipos de acosos son generalizados, es decir, afectan indiferentemente del país o clases sociales. “A menudo se asocia de manera exclusiva el transgénerismo con el mundo adulto como si la identidad de género fuera únicamente fruto de un largo proceso de introspección personal resuelto al cabo de muchos años” (Fernández, 2014 p. 32).

Éste mismo autor, hace referencia a que la realización de una intervención por parte de los asistentes sanitarios correspondientes es de vital importancia, pues “se debe preparar y enseñar a la sociedad conceptos éticos y morales, enseñar a las personas a actuar delante de situaciones de acoso, burla o abuso y a intervenir delante del primer síntoma” (Fernández, 2014, p. 34).

El transgénerismo según Mejía (2006) “se manifiesta típicamente en la medicina y la psiquiatría por una identificación intensa y persistente con el otro sexo, con un sentimiento de inadecuación con el sexo asignado, y por un deseo permanente de vestir, vivir y ser tratado como miembro del otro sexo” (p. 91). Las personas transgénero consideran que han nacido en un sexo equivocado. La mayoría, refiere el inicio de los síntomas desde la primera infancia.

El concepto de género nació para designar todo aquello que es construido por las sociedades para estructurar y ordenar las relaciones sociales. Al basarse estas construcciones sociales y simbólicas en la diferencia sexual, se estructuran relaciones de poder cuya característica esencial es el dominio masculino. La dominación masculina se fundamenta en la diferencia sexual, la cual se explica por el diferente lugar que ocupa cada sexo en el proceso de reproducción, idea también del pensamiento judeocristiano. Se instaura así la lógica del género, que parte de una oposición binaria: lo propio del hombre y lo propio de la mujer (lo esencial en la feminidad y la masculinidad), y dicha lógica del género es una lógica de poder, de la dominación del sistema patriarcal (Mejía, 2006).

La persona transgénero es mal vista en la sociedad, muchos problemas que crean una desigualdad a este tipo de población son creados  por el rechazo de la misma. Debido a su discriminación hacia los transexuales y transgénero, produce que este grupo de personas se sienta avergonzado de sus deseos sexuales y con frecuencia se mantienen en la posición de que se encuentran mentalmente enfermos. Si da pasos a sus deseos queda lleno de sentimientos de culpabilidad y estos sentimientos con frecuencia son más destructores que el acto sexual mismo. Debe ocultar su verdadera identidad y en ocasiones quizás se exija que esté de acuerdo con los otros y pretenda condenar sus propios intereses y actividades.

Esta doble vida de la persona transgénero, que en ocasiones es obligada a llevar llena de miedo, puede ser difícil de mantener y también puede conducir a la depresión. El transgénerismo es una condición que en sí misma marca a la personalidad del individuo permitiéndole el ingreso o no a diversos campos dependiendo de los capitales con los que cuente. Pero las actitudes de las demás personas hacia esta condición, crean una situación de tensión que puede tener un efecto profundo en el desarrollo de la personalidad y puede conducir a un deterioro del carácter de un género que impide la integración efectiva de la comunidad.

Una gran proporción de personas transgénero son incapaces de resistir las presiones y se convierten en bajas sociales. Estos son los Trans que se encuentran con mayor frecuencia en las calles prostituyéndose. Ahora que se estableció que los transgénero no son fisiológicamente diferentes, también existe la posibilidad que poco a poco la sociedad sea más tolerable en cuanto a su respeto y no discriminación. (Sáez, 2006).