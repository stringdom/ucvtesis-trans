% Context document test, Página de resumen ucvpsitesis
\language[es]
\setuppapersize[letter]
\setuplayout[
location=singlesided,
margin=3cm,
topspace=3cm,
backspace=3cm
]
\setupwhitespace[medium]
\setupinterlinespace[medium]
\setuppagenumber[state=stop]

\starttext
\startstandardmakeup
\vfill
\startalignment[center]
{\sca Gender identity and transition on transgender people from Caracas}
\vfill

\starttabulate[|pc|pc|]
\NC \bf Authors:\NC \bf Co-author:\NC \NR
\NC Leonardo R. Pérez & Emerson S. Yancul\NC Luisana Gómez\NC \NR
\NC leo.prez.k@gmail.com; eyanor@gmail.com \NC luisanago@gmail.com \NC \NR
\stoptabulate
\vfill

Universidad Central de Venezuela

Escuela de Psicología


\vfill
\bfa Abstract
\stopalignment

\vfill
\startnarrower

This research had the intention of exploring the construction of transgender's
identity that live in the city of Caracas. A qualitative methodology with a
hermeneutical perspective is used, by way of case study. Data collection was
done through in-depth interviews, which were conducted to adults over 18 who
identify themselves as transgenders. Interviews were codified using the software
ATLAS.ti. This coding resulted in 6 categories that constitute the essential
elements that allow the construction of identity in transgender people. We found
that their identity is largely mediated by the interaction with other
individuals. Transition is a tool used to reassure their own gender identity
both personally and socially. This transition is experienced as a continuum
whose ultimate goal is to become cisgender, while accepting that this is not
currently technically possible. For this reason, transgender people make use of
a set of secondary strategies to compensate and approximate their desired gender
expression according to their identity. Discrimination plays a major role as an
obstacle that encumbers the development of identity in transgender people.

\stopnarrower
 \vfill
{\bi Keywords:} Transition, transgender,
gender, identity,  development, gender expression.

\vfill
\startalignment[center]
Caracas, julio 2018.
\stopalignment
\stopstandardmakeup
\stoptext
