\chapter{Introducción}
% Jueves 26 de Abril de 2018, capítulo reescrito

El sexo, el género y la identidad sexual no carecen de investigación.
Todo un cuerpo investigativo se ha desenvuelto desde el inicio de los
movimientos de derechos para las mujeres.
Este discurso y forma de concepción de las diferencias de género se ha
constituido en un lenguaje de circulación común.
A su vez ha sido modelado por el activismo político a partir del surgimiento
del feminismo de segunda ola (Helfrich, 2001).
Alrededor del género circulan discursos que defienden ciertos ordenamientos
sociales y formas de interacción.
El acceso efectivo a campos sociales enteros, como son propuestos por
Bourdieu (1992), puede ser definido en función del género de una persona.
Y las formas de explicar, justificar y definir el género tienen el potencial
de hacer prevalecer hegemónicamente algunos tipos de ordenamientos sobre otros.

Esto ha generado cambios sociales en las formas de considerar y construir el
género.
Estos cambios son, sin embargo, dispaŁres entre distintos grupos culturales,
clases y campos sociales.
Estudiar estos cambios y las diferencias que existen permitirá complejizar
la comprensión del género y brindar herramientas de intervención.

Uno de los grupos sociales que han surgido a la mirada pública de la cultura
occidental son las personas transgéneúro y transexuales.
Estas personas se identifican y se autoreconocen como pertenecientes al
género opuesto al que les fue asignado socialmente.
Desde el punto de vista tradicionalmente patriarcal, machista y biologicista
no se concibe la posibilidad de que la identidad como hombre o mujer de un
individuo sea capaz de variar independientemente del sexo biológico.
