\chapter{Introducción}\label{ch:intro}
% Jueves 26 de Abril de 2018, capítulo reescrito

El sexo, el género y la identidad sexual no carecen de investigación.
Todo un cuerpo investigativo se ha desenvuelto desde el inicio de los
movimientos de derechos para las mujeres.
Este discurso y forma de concepción de las diferencias de género se ha
constituido en un lenguaje de circulación común.
A su vez ha sido modelado por el activismo político a partir del surgimiento
del feminismo de segunda ola (Helfrich, 2001).
Alrededor del género circulan discursos que defienden ciertos ordenamientos
sociales y formas de interacción.
El acceso efectivo a campos sociales enteros, como son propuestos por
Bourdieu (1992), puede ser definido en función del género de una persona.
Y las formas de explicar, justificar y definir el género tienen el potencial
de hacer prevalecer hegemónicamente algunos tipos de ordenamientos sobre otros.

Esto ha generado cambios sociales en las formas de considerar y construir el
género.
Estos cambios son, sin embargo, dispares entre distintos grupos culturales,
clases y campos sociales.
Estudiar estos cambios y las diferencias que existen permitirá complejizar
la comprensión del género y brindar herramientas de intervención.

Uno de los grupos sociales que han surgido a la mirada pública de la cultura
occidental son las personas transgénero y transexuales.
Estas personas se identifican y se auto-reconocen como pertenecientes al
género opuesto al que les fue asignado socialmente.
Desde el punto de vista tradicionalmente patriarcal, machista y biologicista
no se concibe la posibilidad de que la identidad como hombre o mujer de un
individuo sea capaz de variar independientemente del sexo biológico.

Esta población de personas transgénero y transexuales ha sido
tradicionalmente invisibilizada.
Usualmente se les agrupaba junto con los homosexuales y se les consideraba a
todos conjuntamente como enfermos mentales.
Igualmente existía en el imaginario colectivo la noción de que los hombres
homosexuales querían ser mujeres o que las mujeres homosexuales deseaban ser
hombres.
Con estas concepciones se confundían dos componentes de la identidad que hoy
en día se entienden de forma separada, la identidad sexual y la orientación
sexual (Mejı́a Turizo \& Almanza Iglesia, 2010).

Cuando inició el movimiento de reconocimiento de la diversidad sexual se
comenzó a reconocer la diferenciación entre algunas categorías, como por
ejemplo: la diferenciación entre sexo genético, sexo genital, identidad
sexual, identidad de género, expresión de género y orientación sexual.
Dicha diferenciación ha llevado al surgimiento de nuevas categorías y a
cambios en la forma en la cual se entiende la identidad sexual y de género
(Bleichmar, 2006; Figari, 2010).

En la sociedad en general prevalecen otros imaginarios y otras posturas.
El rechazo a la unión entre personas del mismo sexo, su posibilidad de adopción
son caras visibles del fenómeno.
Pero aún no se reconoce ni se conversa abiertamente sobre el cambio de género y
de sexo.
A su alrededor existen muchas complejidades que la misma academia científica no
es capaz de comprender en su totalidad aún (Bleichmar, 2006).

Como resultado de lo anterior existe aún un rechazo, a veces abierto y a veces
encubierto, ante aquellas personas que deciden cambiar de sexo o que buscan que
su expresión de género coincida con su identidad de género, a pesar de que esta
pueda ser distinta al género que les fue asignado al nacer.
Este problema no es visible entre aquellos que logran mostrarse exitosamente
como el género con el cual se identifican o entre aquellos que no inician aún
esfuerzos para mostrarse como ese género.
Sino que se hace frontal y visible para aquellos que apenas están iniciando una
transición de género o sexo.
En estos casos, desean mostrarse según su género deseado pero aún no lo logran
de manera exitosa.
Esto usualmente resulta en una estigmatización social de tanto de las personas
transgénero y transexuales.

La comprensión de la transexualidad así como del transgenerismo puede ser,
entonces, una herramienta fundamental para afrontar la discriminación, rechazo y
estigmatización que sufren las personas transexuales así como también los
transgénero.
Si bien existe una tradición de investigación experimental alrededor de los
fenómenos de exclusión social y estigmatización, también es cierto que otras
perspectivas menos psicologicistas, como las teorías performativas, sugieren que
existen un amplio componente de deseabilidad social e interacción que determinan
el rechazo activo o pasivo de algunas categorías y grupos sociales.
De ello se puede intuir que un análisis de las interacciones y prácticas que
estructuran al género hoy en día puede dar una mejor comprensión y brindar o
sugerir nuevas estrategias para la inclusión social.

Lo anterior vislumbra uno de los elementos centrales de la importancia del tema
del género, la transexualidad y el transgenerismo.
Muchas de las elaboraciones en psicología y sociología tienen su origen en
fenómenos y estudios del siglo pasado.
Sin embargo, el cambio social no se detiene y siempre el cambio de las teorías y
paradigmas en la academia será mucho más lento que los cambios sociales.

Es importante entonces mantener una vigilancia constante sobre aquellos temas
que se creen cerrados o terminados en cuanto a su comprensión teórica.
Igualmente, se han evidenciado cambios importantes y diferencias significativas
en las formas de comprensión de la masculinidad y la feminidad durante las
pasadas cuatro décadas.
Estos cambios no son obvios ni pueden darse por sentado pues el cambio social no
es uniforme ni homogéneo.
Es decir, no se da en todos los lugares simultáneamente, ni se da de igual
manera.

Venezuela tiene una inserción muy particular dentro de la dinámica de la cultura
occidental, con vastas variaciones y aproximaciones a las realidades sociales.
Las diferencias de clases y de campos en nuestro país son distintas a aquellas
de otros países.
No es entonces transferible directamente la interpretación de estos cambios
culturales, sin ser tampoco completamente ajeno a las influencias.
En particular desde el surgimiento de la comunicación instantánea mediante
Internet y el fenómeno de las redes sociales virtuales.

Es de valor entender estos cambios en Venezuela, permite ubicarnos dentro de un
ámbito que está cambiando rápidamente en el mundo, y que además tiene el
potencial de afectar nuestra dinámica social.
Los roles de género y las dinámicas de interacción entre ellos forman una parte
fundamental de la forma en la que se estructura una cultura y una sociedad.
Tiene impacto en prácticas, creencias, imaginarios y representaciones.
Por tanto es deber de la academia científica poder hacer seguimiento de este
fenómeno y dar cuenta de los cambios históricos y su interacción con otros
fenómenos.

Delgado y Madriz (2014) expresan que “El patriarcado y la heteronormatividad
vendrán de la mano de los conquistadores, misioneros y sacerdotes portadores de
un cristianismo fundamentalista, con una institución sumamente represiva, la
Inquisición” (p.102) en referencia a la visión que dentro de América Latina se ha construido
históricamente hacia las relaciones entre los géneros y la sexo diversidad desde
la época colonial.
Puede afirmarse que se ha presentado como una forma de dominación y prevalencia
del patriarcado y de la heteronormatividad.
Cualquier expresión que transgreda esta norma va a verse sujeta a
discriminación, segregación e invisibilización.

Por otra parte, resulta de particular importancia la inserción que tienen las
personas trans en los servicios de atención médica para la transición de un sexo
a otro.
Debido a que el cambio de sexo es el aspecto central del criterio para la
definición de la transexualidad, elemento que separa a esta condición de otras
expresiones alternativas tales como el ser transgénero—enfocado en la
alineación entre la expresión de género y el género sentido por un individuo—y
la expresión Queer.

El presente trabajo de investigación está compuesto por cinco capítulos.
Iniciando con esta introducción y posteriormente el planteamiento del problema.
Este señala que ante la incógnita que representa la construcción de la identidad
de las personas Trans desde una mirada psicológica no clínica, se plantea la
exploración de las vivencias de las personas Trans para poder dar respuesta a la
siguiente pregunta de investigación:

¿Cómo influye el proceso de transición de género en la construcción de la
identidad en un grupo de personas transgénero que residen en el Área
Metropolitana de Caracas? y ¿Qué lugar ocupa esa transición en su vida cotidiana?

Estos aspectos llevan a reflexionar acerca del transgénero y transexual como un
campo de estudio de suma amplitud y complejidad, por lo que resulta necesario
analizarlo desde una perspectiva psicosocial, y no conformarse con la
interpretación del sentido común que se tiene al respecto.
La perspectiva de una persona transgénero o transexual sobre las diferentes
dimensiones del mismo es muy relevante, desde el porqué de la elección de esta
forma de vida, la decisión de soportar la mirada a veces despectiva del otro,
como familiares, compañeros, trabajo, entorno, entre otros, y correr los
riesgos de los prejuicios.
Adicionalmente, se esbozan los objetivos, general y específicos, que rigen el
estudio para dar respuesta a la pregunta de investigación.

Luego continúa un marco referencial, ubicado en el
Capítulo~\ref{ch:marcoreferencial}, donde se realiza una definición
sobre género;
diferenciación entre sexo y género;
identidad y roles;
seguido de una breve reseña histórica del transgénero y transexual así como
apuntes históricos sobre la temática trans en Venezuela.
Luego se explica de manera concisa como es visto el transgénero y la
transexualidad desde las perspectivas: biologicista, sociológicas y
psicológicas.
Posteriormente, se describen conceptos que vienen a representar el eje central
de la investigación.

En el Capítulo~\ref{ch:metodologia} se plantea la perspectiva
metodológica de la
investigación.
Esta se define como de tipo cualitativa, con enfoque fenomenológico, mediante el
empleo de la entrevista a profundidad, con el propósito fundamental de conocer y
aprehender el fenómeno desde la perspectiva de cada participante, en
concordancia con la pregunta y objetivos planteados.

En el Capítulo IV se analizan los contenidos construidos en las entrevistas
generando, de acuerdo a los hallazgos, las categorías de análisis pertinentes
para su posterior discusión y redacción de las conclusiones del trabajo de
investigación.
El Capítulo V están destinados a la discusión de hallazgos y a las
conclusiones de la investigación respectivamente.
También abarca las limitaciones que encontró la presente investigación.
