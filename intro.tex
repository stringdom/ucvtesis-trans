\chapter{Introducción}
% Miércoles 19 de julio de 2017
El sexo, el género y la identidad sexual no carecen de investigación.
Todo un cuerpo investigativo se ha desenvuelto desde el inicio de los
movimientos de derechos para las mujeres.
Este discurso y forma de concepción de las diferencias de género se ha
constituido en un lenguaje de circulación común.
A su vez ha sido modelado por el activismo político a partir del surgimiento del
feminismo de segunda ola \parencite{Helfrich2001}.
Alrededor del género circulan discursos que defienden ciertos ordenamientos
sociales y formas de interacción.
El acceso efectivo a campos sociales enteros (como son propuestos por
\textcite{Bourdieu1992}) puede ser definido en función del género de una
persona.
Y las formas de explicar, justificar y definir el género tienen el potencial de
hacer prevalecer hegemónicamente algunos tipos de ordenamientos sobre otros.

Esto ha generado cambios sociales en las formas de considerar y construir el
género.
Estos cambios son, sin embargo, dispares entre distintos grupos culturales,
clases y campos sociales.
Estudiar estos cambios y las diferencias que existen permitirá complejizar la
comprensión del género y brindar herramientas de intervención.
La población de personas transgénero ha sido tradicionalmente invisibilizada.
Usualmente se les agrupaba junto con homosexuales y se les consideraba a todos
conjuntamente como enfermos mentales.
Igualmente existía en el imaginario la noción de que los hombres homosexuales
querían ser mujeres o que las mujeres homosexuales deseaban ser hombres.
Con estas concepciones se confundían dos posturas de identidad que hoy en día se
entienden de forma separada \parencite{MejiaTurizo2010-06}.

Cuando inició el movimiento de reconocimiento de la diversidad sexual se comenzó
a reconocer la diferenciación entre algunas categorías.
Estas se tratan de la diferenciación entre: sexo genético, sexo genital,
identidad sexual, identidad de género, expresión de género y orientación sexual.
Esto ha llevado al surgimiento de nuevas categorías y a cambios en la forma en
la cual se entiende la identidad sexual y de género
\parencite{Figari2010,Bleichmar2006}.

Sin embargo, esta se trata de la comprensión académica desde los estudios de
género.
En la sociedad en general prevalecen otros imaginarios y otras posturas.
El rechazo a la unión entre personas del mismo sexo, a la posibilidad de
adopción son caras visibles del problema.
Pero aún no se reconoce ni se conversa abiertamente.
Sobre el cambio de género y de sexo, existen muchísimos malentendidos y mitos
rodeando una condición que la misma academia científica no es capaz de
comprender en su totalidad aún \parencite{Bleichmar2006}.

Como resultado de lo anterior existe aún un rechazo, a veces abierto y a veces
encubierto, ante aquellas personas que deciden cambiar de sexo o que deciden
vivir de acuerdo a un rol de género distinto al que se le asignó al nacer.
Este problema no es visible entre aquellos que logran mostrarse exitosamente
como el género que desean o entre aquellos que no inician aún esfuerzos para
mostrarse como ese género.
Sino que se hace frontal y visible para aquellos que apenas están iniciando una
transición de sexo o, en algunos casos, cuando desean mostrarse según su género
deseado pero aún no lo logran de manera exitosa.
Esto usualmente resulta en una estigmatización social de la persona transgénero
o transexual.

La comprensión de la transexualidad puede ser, entonces, una herramienta
fundamental para afrontar la discriminación, rechazo y estigmatización que
sufren las personas transexuales o transgénero. Si bien existe una tradición de
investigación experimental alrededor de los fenómenos de exclusión social y
estigmatización, también es cierto que otras perspectivas menos psicologicistas,
como las teorías performativas, sugieren que existen un amplio componente de
deseabilidad social e interacción que determinan el rechazo activo o pasivo de
algunas categorías y grupos sociales. De ello se puede intuir que un análisis de
las interacciones y prácticas que estructuran al género hoy en día puede dar una
mejor comprensión y brindar o sugerir nuevas estrategias para la inclusión
social.

Lo anterior vislumbra una de los elementos centrales de la importancia del tema
del género y la transexualidad. Muchas de las elaboraciones en psicología y
sociología tienen su origen en fenómenos y estudios del siglo pasado. Sin
embargo, el cambio social no se detiene y siempre el cambio de las teorías y
paradigmas en la academia será mucho más lento que los cambios sociales. Es
importante entonces mantener una vigilancia constante sobre aquellos temas que
se creen cerrados o terminados en cuanto a su comprensión teórica. Igualmente,
se han evidenciado cambios importantes y diferencias significativas en las
formas de comprensión de la masculinidad y la feminidad durante las pasadas
cuatro décadas. Estos cambios no son obvios ni pueden darse por sentado pues el
cambio social no es uniforme ni homogéneo. Es decir, no se da en todos los
lugares simultáneamente, ni se da de igual manera. Venezuela tiene una inserción
muy particular dentro de la dinámica de la cultura occidental, con vastas
variaciones y aproximaciones a las realidades sociales. Las diferencias de
clases y de campos en nuestro país son distintas a aquellas de otros países. No
es entonces transferible directamente la interpretación de estos cambios
culturales, sin ser tampoco completamente ajeno a las influencias. En particular
desde el surgimiento de la comunicación instantánea mediante Internet y el
fenómeno de las redes sociales virtuales. Es de valor entender estos cambios en
Venezuela. Permite ubicarnos dentro de un ámbito que está cambiando rápidamente
en el mundo. Y además que tiene el potencial de afectar nuestra dinámica
social. Los roles de género y las dinámicas de interacción entre ellos forman una
parte fundamental de la forma en la que se estructura una cultura y una
sociedad. Tiene impacto en prácticas, creencias, imaginarios y representaciones.
Por tanto es deber de la academia científica poder hacer seguimiento de este
fenómeno y dar cuenta de los cambios históricos y su interacción con otros
fenómenos.

De particular importancia es la inserción que tienen las personas transexuales
en los servicios de atención médica para la transición de un sexo a otro. Debido
a que el cambio de sexo es el aspecto central del criterio para la definición de
la transexualidad separada de otras expresiones alternativas del género tales
como el ser transgénero y la expresión queer.

\todo{Verificar el orden de las descripciones a capítulos en este segmento}
Es por ello que el presente trabajo de investigación está compuesto de un
marco referencial ubicado en el Capítulo II, donde se realiza una definición
sobre género, seguido de una breve reseña histórica del transgénero o
transexualidad, seguido por apuntes históricos sobre el mismo en Venezuela.
Luego se explica de manera concisa como es visto el transgénero y la
transexualidad desde las perspectivas: biologicista, sociológicas, psicológicas
y patologización. Posteriormente, se describen conceptos que vienen a
representar el eje central de la investigación entre los que destacan identidad,
autopercepción y la mirada del otro.

Seguidamente en el Capítulo III  se describe el planteamiento del problema, este
en resumen, señala que ante la incógnita que representa la construcción de la
identidad trans desde una mirada psicológica no clínica, se plantea la
exploración de las vivencias de las personas trans para poder dar respuesta a las
siguientes preguntas de investigación: ¿Cuál es el lugar de la transición de
sexo en la construcción de la identidad de género de los transexuales de
Caracas?, ¿cómo acceden, utilizan y se insertan en los servicios médicos de
transición de sexo estas personas?, ¿qué lugar ocupa la transición de sexo en su
vida cotidiana?

Estos aspectos llevan a reflexionar acerca del transgénero o transexualidad como
un campo de estudio de suma amplitud y complejidad, por lo que resulta necesario
analizarlo desde una perspectiva psicosocial, y no conformarse con la
interpretación del sentido común que se tiene al respecto. La perspectiva de una
persona transgénero o transexual sobre las diferentes dimensiones del mismo es
muy relevante, desde el porqué de la elección de esta forma de vida, la decisión
de soportar la mirada a veces despectiva del otro y correr los riesgos de los
prejuicios.

En el capítulo IV, se esbozan los objetivos que rigen el estudio para dar
respuesta a la pregunta de investigación. En el Capítulo V se plantea una
investigación de tipo cualitativa, con enfoque fenomenológico, mediante el
empleo de entrevistas a profundidad, con el propósito fundamental de conocer y
aprehender el fenómeno desde la perspectiva de cada participante, en
concordancia con la pregunta y objetivos planteados.

En el capítulo VI se analizan los contenidos construidos en las entrevistas,
generando así, de acuerdo a los hallazgos, unas unidades de análisis.