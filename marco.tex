% Miércoles, 9 de mayo de 2018
\chapter{Marco referencial}\label{ch:marcoreferencial}
Para poder realizar una aproximación adecuada a la pregunta de investigación
el primer requisito es dar cuenta del estado actual de la comprensión
científica de la transexualidad y la transgeneridad.
Para ello haremos una exposición breve de los conceptos más importantes que
se ven involucrados en la construcción de la identidad desde el concepto
mismo del género hasta la autoimágen corporal.
Incluyendo varias exploraciones respecto al devenir histórico y científico
del entendimiento de lo trans, investigaciones previas y diversas teorías
explicativas que rodean a este tema.

\section{Género}

Para poder adentrarnos en el tema que es el foco de la presente investigación es
necesario primero realizar una revisión de lo que es el género.
Aunque a primera vista eso podría parecer un tema relativamente sencillo, no se
debe tomar a la ligera, pues como lo indica Lamas (1999) el emplear la palabra
género conlleva implicaciones históricas, en un primer momento el feminismo
académico anglosajón usó este término para diferenciar las construcciones
sociales y culturales, pero esto probó ser complicado pues dio pie a que
surgieran una variedad de formas de interpretación, simbolización y organización
de las diferencias relacionadas con la sexualidad y el sexo complejizando de
esta manera la concepción inicial y la aplicación que se le había dado al
concepto de género.

Sin embargo, Lamas rescata una propuesta realizada por Scott (1996 c.p. Lamas
1999) la cual plantea que el género posee dos
partes, analíticamente interrelacionadas aunque distintas, y cuatro elementos.
Lo central de la definición es la “conexión integral” entre dos ideas: el género
es un elemento constitutivo de las relaciones sociales basadas en las
diferencias que distinguen los sexos y el género es la forma primaria de
relaciones significantes de poder.
Con esto en mente es posible que establezcamos una primera característica para
poder adentrarnos en lo que significa el género, y esta sería que está
determinada por las relaciones sociales y las diferencias que se asignan según
el sexo (biológico) de las personas.

Se puede también rescatar lo propuesto por Bourdieu (2000) cuando se refiere
al hecho de que la socialización y la construcción de los géneros ha sido
históricamente demarcada y delimitada por la biologización de lo social, para
generar una división arbitraria de lo que implica haber nacido con un sexo y
tener impuesto un género y unos roles en base a este determinante biológico.
Este autor se adentra en esto planteando que las diferencias visibles entre los
cuerpos se convierten en el factor determinante para promover una visión
androcéntrica del mundo, dando así significados y valores particulares según sea
el caso.

Entonces, para comprender el significado de la palabra \emph{Género} es
preciso tener en cuenta que en los últimos quince años los trabajos
realizados han mostrado cuánto varían las categorías de género con el tiempo y
% TODO cuales trabajos?
con ellas los territorios sociales y culturales asignados a mujeres y hombres.
En muchos períodos históricos las percepciones populares acerca de los
% TODO cuales períodos?
temperamentos masculino y femenino han sufrido cambios sustantivos, que han
venido acompañados por un nuevo mapa de las fronteras sociales (Scott, 1996).

El concepto de género como se mencionó en párrafos anteriores, fue introducido
por feministas estadounidenses en la década de los sesenta para plantear la
reflexión sobre los sexos.
En la próxima sección exploraremos los orígenes precisos del término en los
experimentos de John Money.

Por su parte la autora Huggins (2005) señala que no sólo somos seres sexuados,
sino portadores de cultura y de necesidades que son atravesadas por nuestras
condiciones y experiencias de vida.
Por lo cual se debe ir más allá del dato biológico sexo cuando se analiza la
categoría del género, ya que esta “se desprende de manera tal que, en sus
contenidos, es cada vez más social” (p. 18), haciéndose necesaria la clara
distinción entre sexo y género.

\begin{quote}
    Que sea masculino o femenino no puede ser juzgado de la misma manera: los
    criterios serán culturales, por lo tanto, diferentes según el tiempo y el
    lugar.
    El sexo debe ser admitido como constante, pero también debe admitirse la
    variabilidad del género (Oakley, 1972; c.p. Muñoz, 2004, p. 17).
\end{quote}

Lo que esta perspectiva pretende explicar es que aquellas características y
actitudes que usualmente son tomadas como \emph{atributos naturales} de los
hombres y de las mujeres, no son necesariamente determinadas por la biología,
sino que son construidas socialmente.

En esos ámbitos de discusión, lo correspondiente al sexo se presenta como un
conjunto invariable de características biológicas y, en cierta medida, termina
suponiendo que ese sexo biológico es la base natural de una asignación cultural
del género, la cual determinaría las conductas y los roles que pueden ocupar
hombres y mujeres en cada sociedad.

Es importante destacar que los sistemas de género (sin importar el período
histórico) son sistemas binarios que oponen la hembra al macho, lo masculino a
lo femenino, rara vez sobre la base de la igualdad, sino, por lo general, en
términos jerárquicos.
Si bien las asociaciones simbólicas con cada uno de los géneros han variado
enormemente, han incluido el individualismo versus la crianza, la razón versus
la intuición, lo construido versus lo naturalmente procreativo, la ciencia
versus la naturaleza, la explotación versus la conservación, lo clásico versus
lo romántico, la universalidad de los rasgos humanos versus la especificidad
biológica, lo político versus lo doméstico, lo público versus lo privado.
Lo interesante en estos contrastes es que privan procesos sociales y culturales
mucho más complejos, en los que las diferencias entre mujeres y hombres no son
ni aparentes ni tajantes, y es en ello que reside su poder y relevancia (Scott,
1996).

\subsection{Diferencia entre Sexo y Género}
El origen de la distinción entre sexo y género surge en 1949, cuando Simone de
Beauvoir (1998 c.p. Butler, 2010) en su libro \emph{El Segundo Sexo}, afirma “no
se nace mujer: se llega a serlo”(p. 89)
Pero es John Money (c.p. Buttler, 2010), quien menciona la palabra género por
primera vez.
De esta manera se inicia la discusión sobre una diferencia que había sido
naturalizada, la distinción entre lo que es el sexo y el género, que ha sido
tema de discusión desde entonces en las ciencias sociales.
El primero se refiere al hecho biológico y a las características físicas de los
cuerpos, mientras que el segundo se refiere a los significados que cada sociedad
atribuye a esa diferenciación y que definen lo que es el género.
Este concepto describe el modo en que se organizan los sexos en su relación
social, por lo que implica siempre una relación, que es además una relación de
poder, donde la distinción por géneros vendría a ser parte de nuestra
\emph{naturaleza} tanto comportamental como de los espacios sociales que están
asignados a cada persona según su sexo, como también de acuerdo a su edad,
etnia, clase social, entre otras cualidades que puede tener una persona, y que
permite la restricción y clasificación de su acceso social (Beauvoir c.p. Butler
2010).
De esta forma, se encuentran distintas definiciones de sexo, como la de la
Organización Mundial de La Salud (en Villegas, Rodríguez y Ochoa, 2002) que lo
asume como:

\begin{quote}
    Una distinción por las características biológicas o fisiológicas asociadas
    típicamente con hembras o machos de una especie.
    Es la condición orgánica que distingue a las hembras (mujeres) de los machos
    (hombres).
    El sexo biológico es la anatomía sexual junto con los cromosomas de cada
    persona.
    Biológicamente o se nace hombre o se nace mujer (p. 84)
\end{quote}

Otra definición la plantea Almuneda (2007) quien menciona que, a pesar de que el
“sexo” parezca un concepto aséptico, biológico y sin posibilidad de matización,
se sabe desde los estudios de John Money (c.p. Almuneda, 2007), a mediados
de los años 50, que tiene una determinación multivariada, siendo necesario para
la determinación del sexo de un bebé, la combinación de cinco componentes
biológicos:

\begin{itemize}
    \item \emph{Sexo genético}: determinado por los cromosomas X e Y.
    \item \emph{Sexo hormonal}: el balance estrógenos-andrógenos.
    \item \emph{Sexo gonadal}: presencia de testículos u ovarios.
    \item Morfología de los órganos reproductivos internos.
    \item Morfología de los órganos reproductivos externos.
\end{itemize}

Sin embargo, la experiencia de John Hopkins (c.p. Almuneda, 2007) en su trabajo
% TODO determinar John hopkins o john money
con bebés, condujo a éste psiquiatra a utilizar el término “género” a partir de
1955 para plantear que el psiquismo de un nuevo ser, adquiere una orientación
distinta masculina o femenina dependiendo de si se trata de un hombre o una
mujer, esto se adquiere a través de las interrelaciones que establece con el
entorno social durante los primeros dos o tres años de vida.

Los estudios de género han seguido la línea del planteamiento de Hopkins,
pues, lo que entendemos por género no debe entenderse en el simple sentido de
contemplación desinteresada sino, como plantean Fonseca y Quintero (2009), que
es totalmente político el sentido de la distinción de géneros.
El género no es asumido sólo para la clasificación y distinción de las personas,
sino que más allá de ello se encuentra que en la clasificación existe un
elemento de control y restricción, y un posicionamiento de poder que genera
discriminación en cuanto a lo que no se encuentra bajo la heteronorma
\footnote{Cathy Cohen (2005) define la heteronormatividad como la práctica y
las instituciones “que legitiman y privilegian la heterosexualidad y las
relaciones heterosexuales como fundamentales y naturales dentro de la
sociedad”. Según Rich (1980) La heteronorma o heteronormatividad
es un régimen social, político y económico que impone las prácticas sexuales
heterosexuales mediante diversos mecanismos médicos, artísticos, educativos,
religiosos, jurídicos, etc. y mediante diversas instituciones que presentan
la heterosexualidad como necesaria para el funcionamiento de la sociedad y como
el único modelo válido de relación sexoafectiva y de parentesco.
El régimen se retroalimenta con mecanismos sociales como la marginalización,
invisibilización o persecución.
Tiene como base un sistema dicotómico y jerarquizado.
Esto incluye la idea de que todos los seres humanos se distribuyen en dos
categorías distintas y complementarias: varón y mujer;
que las relaciones sexuales y maritales son normales solamente entre personas de
sexos diferentes;
y que cada sexo tiene ciertos papeles naturales en la vida.
Así, el sexo físico, la identidad de género y el papel social del género
deberían encuadrar a cualquier persona dentro de normas íntegramente masculinas
o femeninas.}
y los patrones hegemónicos de lo que representa cada sexo en cada cultura.
Existen múltiples estudios de género desde la antropología
% TODO ejemplos de esos múltiples estudios
precisamente para distinguir entre el sexo biológico y el género como un
constructo social, precisando así, el papel de la construcción sociocultural en
la que se interviene en el mundo subjetivo de cada persona que al nacer, además
de un nombre adecuado a su género, luego debe ser instruido con patrones
distintivos para poder comprender el mundo social al cual va a ser inserto.

Este mundo social además de tener sus propios códigos de género, también en el
ejercicio del control social, asigna papeles asimétricos en el ejercicio de
poder de dichos roles.
Estas estructuras asimétricas no nacen de un vacío social, sino que surgen como
una identidad a partir de los procesos psicosociales.
De esta forma, algunos autores plantean que las interpretaciones de género
% TODO algunos quienes?
residen en lo biológico y lo corporal, es decir, sitúan la clasificación como un
asunto propio de la naturaleza biológica
Pero en la Teoría de los roles sexuales planteada por Fagot (1982) se sostiene
que la interpretación del rol se relacionaba con la estructura definida por la
diferencia biológica, la dicotomía entre lo masculino y lo femenino, y no con
una estructura definida por las relaciones sociales (Connell, 2003 c.p.
Botello, 2005).
Durante muchos años se consideró la masculinidad y la feminidad como una única
dimensión, con dos polos, que hacía posible clasificar a los individuos en un
determinado punto de ese continuo, es decir, éstos podían ser en mayor o menor
grado masculinos o femeninos, pero nunca las dos cosas a la vez.
Asimismo, los roles sexuales estaban rígidamente ligados al género, de manera
que el ser masculino o femenino dependía básicamente de ser hombre o mujer
(Kohlberg, 1996).

En la década de los setenta, “ha surgido una nueva concepción que considera
la masculinidad y feminidad como dos dimensiones independientes, de tal forma
que todos los individuos poseen en mayor o menor grado esos dos rasgos”
(Spence y cols, 1975, p. 29).
Así, ha nacido el concepto de androginia psicológica, para designar a aquellos
individuos que presentan en igual medida rasgos masculinos y femeninos, y se han
desarrollado una serie de cuestionarios y escalas específicas para medir la
masculinidad, feminidad y androginia.
En este nuevo enfoque de los roles sexuales, la masculinidad y feminidad
“representan dos conjuntos de habilidades conductuales y competencias
interpersonales que los individuos, independientemente de su sexo emplean para
interactuar con su medio” (Spence y cols, 1975, p. 35).

Esta perspectiva ha posibilitado el desarrollo de numerosas investigaciones, al
disminuir considerablemente la inevitabilidad y el determinismo ligado a los
rasgos masculino y femenino (Spence y cols, 1975).
Por su parte Lagarde (s.f. c.p. García, 2000), tiene planteamientos más
fundamentales sobre la conformación de los géneros, para ella:

\begin{quote}
    Son formaciones políticas que están estructuradas a partir de cargas y
    tensiones de poder que aseguran a los sujetos sociales cumplir sus deberes
    como mujeres y como hombres, y les impiden, al mismo tiempo, realizar
    las prohibiciones.
    Sus objetivos centrales son: a) Especializar a los sujetos definidos a partir de
    su sexo;
    b) convertirlos en expertas/os, en actividades y funciones particulares
    que los hagan ser mujeres y hombres;
    y c) lograr la continuidad del mundo así estructurado.
    Así, a través de variados mecanismos los sujetos quedan incluidos o excluidos de
    ámbitos y relaciones, y ocupan posiciones jerárquicas.
    Además, a las funciones y a las actividades asignadas se les confiere valor
    económico, social y cultural, que se convierte en poderío o en carencia de
    poderes, para mujeres y hombres, y para la sociedad.
    Así, la organización genérica asegura determinada distribución de poderes y de
    bienes que permite preservar el orden del mundo (p.77).
\end{quote}

Con una estructura social categorizada y clasificada, se facilita el control
social, por lo que otros autores buscan la distinción entre el sexo y el género
para generar un planteamiento ante lo naturalizado y así contribuir a la
comprensión de la complejidad de lo que se construye socialmente como
masculinidad.
Buenfil (1985 c.p. Botello 2005), en sus planteamientos feministas, similares
a los de Largade, asume que el género no es lo único que genera identidad en la
persona, sino que sus actividades sociales, laborales, la pertenencia a un
estrato social, el nivel de escolarización, entre muchas otras característica
van asignando un lugar social en el que se define al individuo y que dichas
características aseguran un lugar en la sociedad a la persona, a estos
planteamientos la autora los denomina “Polos de Identidad”, que describen a los
individuos dentro de continuos bipolares para definir su identidad.

Es importante rescatar, que tanto el sexo como el género son elementos que a lo
largo del desarrollo de la humanidad han marcado y direccionado las relaciones
entre los individuos y consecuentemente permiten dar sentido a los
cuestionamientos que se puedan presentar cuando se interpelan discursos que
naturalizan conductas o actos dentro del mundo social que se vean como propios
de la biología de un hombre o una mujer.
Pues, el discurso de género como había sido tratado, había sido un núcleo para
definir y clasificar a las personas, que se construye a través de un proceso de
socialización que distingue el sexo a través de lo biológico, y que también
ubica a éste discurso desde un puesto de poder de las ciencias básicas, lo que
implica que situar éste discurso desde la ciencia tiene también una carga
política.

De manera tal, que en el mundo la noción de género (antes de la intervención de
la teoría de género) se había reducido burdamente a su componente biológico,
naturalizando su carga social, que determina nociones, relacionales sexistas
promovidas socioculturalmente, que generan identificación, justificación de
conductas, encubrimiento de discriminaciones, entre otros, ahora tiene otras
posibles resignificaciones y reconstrucciones (Lizardo, 2008).

\subsection{Identidad de Género: Masculinidad, Feminidad y Trans}
El discurso social que interpretamos sobre las relaciones de género tiene sus
condiciones de existencia sobre la constitución de las identidades masculinas y
femeninas, con una modalidad excluyente que se construye en relación con una
división sexual del trabajo, fundada en la separación de la vida social entre
una esfera de lo público (producción) y otra esfera de lo privado
(reproducción), y la asignación de los hombres a la primera y de las mujeres a
la segunda (Cubillán, 2012).

La identidad puede entenderse como el autoreconocimiento y pertenencia que
experimenta un individuo particular.
Siendo estos elementos los que caracterizan al sujeto o a la colectividad frente
a los demás.
La identidad también puede considerarse como la conciencia que una persona tiene
respecto a sí misma y que la convierte en alguien diferente a los demás.
A pesar de que muchos de los rasgos que conforman la identidad son hereditarios
o innatos, el entorno ejerce gran influencia en la conformación de la
especificidad de cada individuo (Hothersall, 2004).

Por otra parte, la Identidad de Género hace referencia básicamente al
autoreconocimiento como niño o niña, hombre o mujer, mientras que la
identificación implica sentimientos de suplencia, deseos de ser como otro.
La tipificación sexual o asunción del rol que la sociedad asigna a cada sexo es
distinto de ambos conceptos, ya que se trata de asimilar las características
conductuales propias de cada sexo en una sociedad determinada (Hothersall,
2004).

Thompson (1975) hace una distinción fundamental que ha permitido clarificar este
campo de discusión, ya que definen la Identidad de Género como la
auto-clasificación como hombre o mujer, basada principalmente en la figura
corporal;
y el Rol Sexual, por otra parte, hace referencia a “los comportamientos,
sentimientos, actitudes que se consideran propios de un hombre
o una mujer, teniendo entonces una base más cultural” (p. 76).

El mismo autor refiere que la adquisición de la identidad y rol sexual ocurre
entre el año y medio a los tres o cuatro años.
Primero el niño aprende a reconocer que hay dos sexos, luego se incluye a sí
mismo en una u otra categoría, y a partir de aquí comienza a usar el rotulo del
sexo como guía de referencia relacionada con los roles sexuales.
La permanencia de género se adquiere a los seis o siete años.
Esta permanencia no tiene en un principio su origen en los genitales como podría
esperarse, sino que está más relacionado con las características del rol.
Existe un período posterior en el que el género está determinado por los
genitales en lugar de por las conductas dependientes del rol (llevar el cabello
de una u otra forma, usar vestidos, etc.).
A este periodo llegará el niño hasta los siete o nueve años (Thompson, 1975).
Es necesario aclarar que desde el momento en que este autor realizo sus
propuestas a la actualidad han surgido nuevas teorías que nutren la comprensión
de la identidad de género.

Vedrell (2009) afirma que la Identidad de Género no se trata exclusivamente de
la biología sino que incluye factores sociales, contextuales y psicológicos, por
lo que el entorno social y los valores relacionados a lo masculino y femenino en
una sociedad determina y en gran parte condiciona las conductas y actitudes
generando un modelo a seguir, el cual está basado en estereotipos e ideales.
Por esta razón, todo lo que no esté dentro de los estándares de normalidad
sexual pasa a ser anormal o patológico.

\subsubsection{Masculinidad}
Toda persona es validada según su hacer, una persona cuyo hacer no es propio
actúa en detrimento de sí misma y entonces se encuentra alienada, una
explicación para esto es lo que los escritores Hegel y Marx definen como la
alienación, un fenómeno que puede ser entendido como la extrañación o la
desapropiación.
Para estos teóricos “una persona está alienada cuando es de manera distinta,
ajena, extraña a cómo debería ser” (Catilla, 1986 c.p. Lizardo, 2008).
Entre estos fenómenos alienantes se encuentra la noción de que el género y el
sexo son elementos estáticos determinados por características biológicas como
por ejemplo la carga genética (noción que es errada pues el género se encuentra
determinado por factores socioculturales mientras que el sexo si podría
encontrarse influenciado por elementos genéticos), es en estas situaciones
cuando la incorporación de elementos que rompen con la norma llevan a la
alienación del individuo y en este caso específicamente podría plantearse que la
ruptura con la alienación a la hetero-norma, crea malestar en la forma
individual de vivirse.

Resulta importante reconocer la visión masculina para la comprensión de los
malestares que gestan la masculinidad hegemónica.
Los estudios de género empiezan a incorporar al hombre a partir de los años 90,
lo que permite ampliar la visión del concepto y no sólo el estudio de lo
femenino o la mujer (Cubillán, 2012), y en este sentido el mayor énfasis por el
estudio para la comprensión de lo masculino ha recaído principalmente sobre las
mujeres, sumando en su mayoría a mujeres estudiando lo femenino y mujeres
estudiando lo masculino, aportando mayor investigación desde lo femenino según
relata Botello (2005).

El mundo de la masculinidad se describe dentro de un mundo social más complejo,
donde existen otros roles y otras categorías definitorias de la persona, lo que
evidencia que el espectro de género es complejo.
Como hace referencia Butler (2010) “la categoría “mujer” no puede describirse
con características hegemónicas, tampoco la masculinidad puede definirse como
una única forma de ser, y aunque en su manifestación de poder sigue conservando
un lugar privilegiado en el mundo social” (p.106).
Es necesario reconocer que la conformación de la masculinidad está articulada
con otras estructuras sociales, donde conjuntamente participan en la elaboración
de contenidos simbólicos que orientan la organización social y que asignan un
lugar privilegiado al hombre, pero no a cualquier hombre, sino aquel que cumple
con unas características especiales en cuanto a etnia, posición económica y
orientación sexual (Botello, 2005).

De acuerdo a Burín (2003) en la actualidad factores como la clase, raza y
orientación sexual se han convertido en determinantes de la diferenciación
masculina es por esto que en palabras de la autora, existen distintas
masculinidades.
Esto no es algo descabellado pues como ya lo hemos expresado anteriormente la
construcción de la masculinidad es un proceso que viene determinado por las
interacciones del sujeto con otros individuos del mismo género así como con
individuos del género contrario.
Es entonces aquí donde se encuentran construcciones sobre lo que significa ser
hombre tan variadas como hombres hay, consecuentemente la construcción de un
hombre único con el que se pueda identificar un individuo es bastante difícil.

Autores como Téllez y Verdú (2011) afirman que:

\begin{quote}
    El “hacerse hombre” puede ser considerado un proceso de construcción social en
    el que se le asigna a lo masculino una serie de características definidas por la
    sociedad que tienen como finalidad mantener la experiencia exclusiva del poder a
    aquellos individuos masculinos (p. 80).
\end{quote}

Sumado a esto, los autores plantean que por medio del estudio de la masculinidad
se puede revelar un sistema en el que el género actúa como medio de control e
imposición de actividades sobre los individuos de una población.
Además es necesario remarcar que, en palabras de Téllez y Verdú (2011), la
concepción de lo que se considera que es masculino ha variado según factores
como el momento histórico, la etapa evolutiva, el nivel socioeconómico, entre
otros.

Las autoras también hacen énfasis en que a la masculinidad suele exigírsele una
expresión más activa.
Es decir, se debe demostrar ser hombre según tu etapa de desarrollo.
Tomando todo esto en cuenta se podría reafirmar la noción de que no existe una
forma única y correcta de lo que significa ser hombre, ya que la masculinidad
está construida en base a aspectos sociales y que los mismos no son estáticos e
inmutables.

\subsubsection{Feminidad}
Para el autor Martínez-Herrera (2007):

\begin{quote}
    La historia oficial es escrita por el hombre que asume la representación
    universal de la humanidad;
    otra muy diferente es la historia de las mujeres.
    Lo masculino y lo femenino constituyen producciones sociales en un momento dado,
    por lo cual no están exentas de tensiones y se encuentran siempre en movimiento
    (p. 88).
\end{quote}

El mismo autor considera que:

\begin{quote}
    Los hombres y las mujeres reproducen, aunque no inevitablemente a nivel onto y
    filogenético la perpetuación de estas condiciones.
    A pesar de los profundos cambios estructurales en la esfera de la producción, la
    ideología y las leyes, acaecidas a través de los tiempos, estas condiciones se
    mantienen relativamente invariables.
    La perduración de las condiciones vejatorias femeninas cuyo origen se pierde
    en los albores del tiempo humano, nos lleva a la pregunta acerca de cuáles son
    los procesos subyacentes a dicha constante histórica (p. 88).
\end{quote}

El género ha sido históricamente conceptuado sobre la base de parámetros
masculinos.
En la antigüedad lo femenino, era establecido por oposición a lo masculino, es
su negativo o el reverso.
Lo femenino es lo que no es, o lo que no se debe ser, un lugar proscrito que
convoca el horror o el rechazo.
De hecho, al rival se le deshonra asemejándolo a una mujer.
La feminidad se erige así, como un antivalor determinado por la exclusión y no
como un valor intrínseco a partir de sus propias características y naturaleza.
De lo anterior se concluye que el sexo femenino es un no-sexo o dicho en otras
palabras, es un sexo que no le pertenece a la mujer (Irigaray, 1977).

Para Martínez-Herrera (2007):

\begin{quote}
    La teoría feminista brinda como corpus teórico polisémico algunas claves para
    la comprensión del problema de la reproducción de las condiciones históricas de
    la discriminación femenina, derivándose diversas explicaciones en dos órdenes.
    Uno a nivel de las relaciones de poder omnipresentes en la teoría de género y
    otro que alude a la constitución y la construcción del género como atribución
    cultural, personal y psicológica (p. 89).
\end{quote}

Ambas dimensiones se entrecruzan y se multideterminan entre sí.
Concretamente Scott (1996) dice que:

\begin{quote}
    …tres han sido los enfoques teóricos privilegiados en los análisis del
    género, a saber: un esfuerzo específicamente feminista por explicar el
    patriarcado;
    un intento de compromiso de la tradición marxista con las críticas feministas;
    y la tradición psicoanalítica en dos de sus vertientes fundamentales, la
    denominada teoría de las relaciones objétales y el estructuralismo freudiano
    francés.
\end{quote}

En general, las distintas explicaciones y teorizaciones coinciden en la
existencia de dos constantes históricas como elementos determinantes en la
construcción social de la feminidad, una de ellas es lo que Bourdieu (2000 c.p
Martínez-Herrera, 2007) denomina:

\begin{quote}
    El cuerpo de la mujer como capital simbólico, en tanto objeto de apropiación
    y deseo, como cuerpo para el otro.
    Por otra parte, tenemos a la mujer/madre, siempre al servicio y cuidado de los
    demás.
    La mujer se debate así entre dos representaciones sociales disociadas entre sí,
    la maternidad a ella asignada y el erotismo que remite a la mujer a una
    condición primigeniamente sexual/genital (p. 89).
\end{quote}

Para Lagarde (1997 c.p. Martínez-Herrera, 2007), históricamente:

\begin{quote}
    La feminidad está atravesada por una dimensión óntica de ser para otros, que es
    donde adquiere sentido vital y reconocimiento de sí, por su contribución a la
    realización de los demás.
    Ésta condición remite a la mujer a una permanente incompletud y la
    ubica al servicio de una ética de cuidados, encargada de dar,
    preservar, proteger y reproducir la vida (p. 89).
\end{quote}

La ubicación de la mujer en una esfera no tradicional supone romper con el ideal
estereotipado de mujer-madre y la coloca en el sospechoso lugar de trasgresión,
lo cual funciona como una fuente de represión social y psicológica
(Martínez-Herrera, 2007).

Por su parte, Burín (2003) además de adentrarse en el tema de la subjetividad
masculina también toca principalmente la subjetividad femenina, la cual al igual
que el caso de la masculina viene determinada por aquellas interacciones que
tiene un individuo y que le permiten construir un ideal de feminidad, el cual no
puede estar ligado a una concepción fija de lo que es el ser.

Además de lo anteriormente mencionado, Martínez- Herrera (2007) reafirma que
históricamente el género se ha presentado en un continuo.
No es sólo una construcción social sino que también actúa como filtro cultural
que permite acercarse e interpretar de una manera particular al mundo.
Con esto en mente se podría entender entonces que la experiencia masculina y la
femenina suelen colocarse en extremos opuestos y con connotaciones particulares.

Es por esto que se asume al hombre como representante universal de la humanidad
a diferencia de las mujeres.
Para Lagarde (1997 c.p. Martínez-Herrera, 2007), quien hace referencia a la
feminidad, la propone como algo que históricamente ha sido concebida en función
de ser para otros, es decir, la imagen de mujer que ha sido construida se ha
enfocado en que su existencia adquiere sentido y reconocimiento en tanto se
dedique a la asistencia y el apoyo a otros.
De aquí nace la imagen de que una mujer no se encuentra completa o realizada
hasta que no complete una suerte de profesa al convertirse en madre.

Adicionalmente, Lagarde (1990, c.p. Martínez-Herrera, 2007) plantea que
debido a los cambios sociales, y al hecho de que en su mayoría los roles de
género se encuentran demarcados por situaciones específicas de un tiempo
histórico, se ha visto un cambio en la percepción de la feminidad, apoyado por
movimientos feministas.
Estos han resignificado a la mujer así como también se ha visto afectada la
concepción de la masculinidad.

Es necesario entonces demarcar que existe una construcción de subjetividades
masculinas y femeninas, y que las mismas pueden ser sumamente variadas.
Pero se debe resaltar que esta variación permite explicar el por qué se podría
establecer subjetividades que trascienden a una asignación binaria

\subsubsection{Trans}
No existen muchos estudios en base a la subjetividad de la identidad Trans como
para tener un consenso pero pareciera poder justificarse en base al hecho de que
la construcción de la identidad es un hecho interno y privado y que la misma
viene determinada por las relaciones que forma el sujeto.

Autoras como Jerez (2012) exponen que “Transgénero” es un término que incluye
las distintas maneras en que las identidades de género de las personas pueden
diferir del sexo que se les asignó al nacer.
Existen muchos términos diferentes que las personas transgénero utilizan para
describirse a sí mismas, por ejemplo: en ocasiones, la palabra “transgénero” se
acorta como “Trans” o incluso “mujer u  hombre Trans”, es necesario aclarar que
dentro de este término se pueden agrupar tanto las personas transgéneros como
transexuales e incluso travestis.
Siempre es mejor utilizar el lenguaje y las etiquetas que cada persona elige
para sí misma.

Las personas transgénero expresan su identidad de género de maneras diferentes,
algunas utilizan su vestimenta, comportamiento y gestos para vivir según el
género que sienten.
Algunas personas toman hormonas y pueden someterse a una cirugía para
transformar su cuerpo con el fin de que coincida con su identidad de género, en
cuyo caso se les denomina transexuales.
Otras rechazan el entendimiento tradicional de género dividido en “masculino” y
“femenino”, y se identifican solo como transgénero, intergénero, de género
fluido o de otras formas.

Las personas transgénero según Jerez (2012) son diversas en sus identidades de
género:

\begin{quote}
    …La manera en que llegan a sentirse emocionalmente, en las expresiones de
    género que muestran, por su forma de vestirse y también de actuar ante los
    otros y su entorno, y en las orientaciones sexuales o atracción particular
    hacia otras personas (p.35).
\end{quote}

Según la autora Dowshen-Atanda (2014) ser Trans no es lo mismo que ser
homosexual.
Ser transgénero tiene que ver con la identidad de género, la forma en que la
persona se ve a sí misma y el género con el que se identifica.
Ser homosexual, es decir, ser gay o lesbiana, tiene que ver con la orientación
sexual, el género por el que se siente atraído.
Muchos homosexuales están cómodos y se sienten bien con su género, no desean
tener un género diferente al que poseen, lo que ocurre es que se sienten
atraídos por personas de su mismo sexo.
Puesto que la orientación sexual es distinta de la identidad de género, una
persona transgénero puede ser heterosexual, homosexual (gay o lesbiana) o
bisexual.

Para Lothstein (1983), el término transexual fue empleado para referirse:

\begin{quote}
    A aquellos pacientes con un desorden de la identidad de género de toda la
    vida, quienes además de travestirse, se identifican completamente con el sexo
    opuesto, por lo que se sentían atrapados en el cuerpo equivocado y querían una
    cirugía para corregir dicho desorden (p. 67).
\end{quote}

La definición propuesta por Blanchard (1989) sugiere que es un tipo de
alteración pues las personas tienden a sentir que el cuerpo que poseen no es el
indicado y por ello recurren a vestirse de otras maneras y adoptar actitudes que
no son propias de su género biológico una posible alteración o marcada variación
psicológica del sentido de la identidad, tanto corporal (genital) como de la
identidad mental, es decir, de la idea del propio género (Graille, 2001).

De igual forma, Techeira, M. (2014) un defensor de la intervención quirúrgica,
aseveró que:

\begin{quote}
    La psicoterapia empleada para curar el transexualismo con los métodos
    actuales es inútil, ya que la orientación de género en estas personas no puede
    ser cambiada.
    Entonces ya que la mente del transexual no puede ser cambiada, es comprensible
    intentar lo opuesto, es decir, ajustar el cuerpo a la mente (p. 21).
\end{quote}

Por su parte, Stoller (1968), define el Transexualismo como “la convicción de
una persona, biológicamente normal, de pertenecer al otro sexo” (p. 16).
En la actualidad, el adulto acompaña esta creencia con la demanda de
intervención quirúrgica y endocrinológica para modificar la apariencia anatómica
en sentido del otro sexo.

Person y Ovesey (1974), hacen referencia al término Transexual, considerando que
diversos autores alegan que es contextual, por lo que toman en cuenta las
características históricas en los que se desarrolla la Transexualidad, razonando
que la necesidad de las intervenciones quirúrgicas o endocrinológicas no son
inherentes a esta definición, sino más bien, producto del momento actual en el
que el avance del conocimiento y de la ciencia hacen posible la implementación
de terapias hormonales y quirúrgicas para realizar el cambio morfo-anatómico,
que era imposible en otras etapas de la humanidad.

Entonces, ¿cómo se puede definir lo Trans? Pues la APA (American Psychiatric
Association and others, 2013) plantea que se puede hablar de transgénero, que es
un término que define a personas cuya identidad de género, identidad o conducta
no se ajusta a aquella con la que se le debería ver asociado por su sexo
biológico.
Por otra parte se puede hablar también de transexualidad, en este caso se
plantea una identidad de género que es diferente al sexo asignado y que está
ligada a un deseo de alterar el cuerpo para alcanzar ese sexo con el que sí
existe una identificación a pesar de la diferencia biológica.

Se podría entonces tomar en cuenta que existen deseos parecidos en ambos casos,
tanto en personas transgénero como en personas transexuales, el foco principal
se encuentra en la transición para llegar a ese sexo o género con el que se
identifican pues la apreciación que los individuos tienen tanto por su cuerpo
como por su genitalidad .
Es por esto que la American Psychological Association plantea que el uso de la
palabra Trans es adecuado pues engloba a ambas identidades.

\subsection{Roles de género}
Por su parte, el autor López (2015) expone los roles como un conjunto de papeles
y expectativas diferentes para mujeres y hombres que marcan la diferencia
respecto a cómo ser, cómo sentir y cómo actuar.
Estos roles son asignados por la sociedad en que vivimos y como consecuencia de
ello, las personas se desarrollan como mujeres o como hombres identificándose
con los roles que por su sexo le han sido asignados.

El mismo autor, expone que el concepto de “Roles de Género” es fundamental para
entender algunos procesos que se interrelacionan en la vida cotidiana.
Su transformación podría ser un paso importante para conseguir vivir en una
sociedad más equitativa.
La igualdad de oportunidades requiere la reformulación de los roles femeninos y
masculinos en función de sus necesidades actuales.
Estos roles se plasman, por ejemplo: en actitudes y planteamientos tradicionales
tales como:

\begin{itemize}
    \item Asociar el ser mujer u hombre a unas actividades, potencialidades,
    limitaciones y actitudes determinadas.
    \item Calificar algunas actividades como “de mujeres” o de “hombres”.
    \item Asignar tareas “propias” de las mujeres y otras de los hombres.
    \item Dar a una misma actividad una importancia diferente.
\end{itemize}

Con base a lo planteado anteriormente es necesario rescatar lo planteado por
Hernández (2016), según este autor esta diferenciación entre los sexos, asociada
a una diferencia en normas y valores que pueden tener como consecuencia que
hombres y mujeres se vean víctimas de expectativas sociales que finalmente
afectan a su comportamiento y desempeño, se le puede llamar rol de género.
Entendiendo el género como una categoría que abarca aspectos psicológicos,
sociales y culturales ligados a la feminidad y masculinidad que está a su vez
asociada a un proceso histórico de construcción social.
Autores como González y Cabrera (2013) expresan que esta construcción de roles
se puede presentar según tres aspectos, principalmente dentro del ambiente
escolar: el trato con otros miembros del grupo, la vestimenta y el juego.
Estos elementos moldean como se debe comportar una persona según su género y que
además son influenciados significativamente por los contextos en los que se
desenvuelve el individuo.

Con esto en mente se puede entonces plantear que, según la socialización del
individuo, se construyen roles de género diferenciados dependiendo del sexo
biológico que poseen, y que se busca mantener una consonancia entre lo que viene
asociado al mismo.
Es decir, un individuo con un pene que se ve como un hombre debe mantener una
conducta masculina y un individuo con una vagina que se vea como mujer deberá
mantener una conducta femenina, e ir en contra de esto significa romper con
roles que han sido ya históricamente establecidos.

\subsection{Socialización diferencial de género}
Como fue planteado en párrafos anteriores, el género se ve definido por las
relaciones sociales que se establecen entre los individuos.
Consecuentemente se puede pensar que existe un proceso histórico que, como lo
plantea Bourdieu (2000), es continuo en la diferenciación a la que los
hombres y mujeres se ven sometidos.
De esta manera se distinguen masculinizándose o feminizándose.
Es por esto que es necesario describir esta construcción social de lo que
significa ser hombre o mujer, pues se encuentra ligada con la identidad de las
personas Trans.

Bourdieu (2000) expresa que principalmente el trabajo de la reproducción de esta
construcción de la identidad ha venido estructurado por la familia, la iglesia y
la escuela.
Estas permiten a los jóvenes construir una identidad con base a las estructuras
subyacentes.
Adentrándose más en este aspecto se puede entender que el primer vínculo que
establece una persona es con la familia
Como lo plantean Berger y Luckmann (2003) la socialización primaria es
aquella que un individuo atraviesa durante la niñez y por medio de la misma
se convierte en un miembro de la sociedad.
Es por esto que la familia se convierte en la más importante referencia para el
individuo pues lo estructura y condiciona.
Por ello identificar el papel de la familia en la socialización diferencial de
género no resulta muy difícil, pues es esta la que impone una división, a veces
sin siquiera haber nacido el individuo.

Por otra parte la Iglesia, según Bourdieu (2000), inculca una visión
explícitamente pro-familiar enteramente dominada por los valores patriarcales.
Esto es particularmente importante en sociedades como las latinoamericanas en
las que la iglesia católica ha jugado un papel primordial en el control
social de los roles de género.

Finalmente la escuela, independientemente de si se encuentre o no bajo la
influencia directa de la iglesia, sigue transmitiendo una construcción
patriarcal de la sociedad, pues permite la reproducción de la división en base a
características biológicas e impone sobre los estudiantes una construcción de
género en base a la misma.

Autores como Mansilla (1996) reafirman lo planteado anteriormente al señalar que
el desarrollo psicosocial del niño o niña viene dado inicialmente por su
familia, que se convierte en su grupo de referencia afectiva, así como con el
mundo que lo rodea, para que de esta manera  pueda formarse y hacerse un ser
social.
Además de esto la autora plantea que existen modelos sociales, normas o reglas
que se usan para enseñar a los individuos formas aceptadas de interactuar,
pensar y ser.
Estas son implementadas en los individuos desde su nacimiento según su sexo
biológico, es así como a un individuo que nació con genitales masculinos se le
enseña a obrar como debería actuar alguien que pertenece al grupo masculino.
La autora plantea que es de esta manera en la que se generan estereotipos de
género, pues, cosas como que una mujer pueda ser igual o más fuerte que un
hombre entra en conflicto con la noción de que eso no sería algo femenino.

Según Antón (2001) se podría plantear, además de lo anterior, un nuevo factor en
la socialización de los niños y niñas y la forma en la que construyen su
identidad de género.
Este nuevo factor es la televisión y más allá de ella, los medios.
Según este autor se muestran en series televisivas infantiles identidades
patriarcalmente hegemónicas sobre lo que debería hacer un hombre y una mujer,
expresan a la masculinidad unida al recurso de la violencia y el riesgo,
mientras que a la feminidad unida a la debilidad, bondad y pasividad.

Entonces, parece ser sensato afirmar que la construcción del género y la forma
en la que socializan el género los individuos de una sociedad pueden verse
asignadas según el sexo del individuo.
Esto puede generar interacciones que que están ligadas a conceptos hegemónicos
que muy posiblemente no se adapten a la realidad de las relaciones sociales y
que permite únicamente una mayor separación entre hombres y mujeres, y puede
causar un mayor conflicto en alguien que no se sienta identificado con el
género que le asignaron por su sexo biológico.

\section{Transgénero o transexualidad a lo largo de la historia}
Desde la perspectiva religiosa San Gregorio en el siglo III (c.p.
Patai, 1967), basándose en Génesis versículo 27, aseveraba que:

\begin{quote}
    Dios creó al hombre según su imagen debió crearlo en principio hermafrodita,
    debido a que Eva se creó de una costilla de Adán, lo que significa que
    engendrada por él mismo, ya que cuando Dios creó a Adán lo hizo a su imagen y
    semejanza, como hombre y mujer (p.84).
\end{quote}

Existen pocas referencias de personajes transexuales en Génesis, entre ellos
se encuentra el mito de José (Patai, 1967) donde se describe a un joven:

\begin{quote}
    …muy vanidoso que se pintaba los ojos, caminaba con afectación, se peinaba como
    mujer, y se vestía con túnicas de mangas largas que en Egipto se consideraban
    vestidos meramente femeninos, incluso ciertos textos apócrifos indican que José
    rechazó a Zuleika, la esposa del eunuco Putifar por su rechazo al sexo femenino.
    No obstante, se casó con Asenat con quien tuvo dos hijos, sin embargo, en Egipto
    los matrimonios de homosexuales y eunucos con mujeres no eran raros entre
    miembros de la corte (p. 87).
\end{quote}

Otro de los mitos primitivos egipcios fue el del binomio divino Isis/Osiris
donde se subrayaba la oposición y la igual naturaleza de las deidades
masculina/femenina, las cuales se podían entender como deidades distintas que se
complementaban pues ambos no solo eran hermanos, también esposos, cuya unión
después de la muerte de Osiris da origen a Horus, quien logra traer balance a
Egipto, es resaltante pues el papel que tiene el balance entre lo masculino y lo
femenino.
Otro elemento importante es que en ciertas representaciones Isis tenía barba y
se le atribuían enigmáticas palabras: “aunque soy hembra, me he convertido en
macho y viceversa” (Luckert, 1991, p. 44).

Según Brissom (1973) en el mundo antiguo Griego se cree que los “mitos de cambio
de sexo no emanan únicamente de un deseo humano de placer, sino que también
representaban un tipo de castigo” (p. 29).
Por ejemplo el mito griego en el que viendo copular a dos serpientes y matar
a la hembra, “Tiresias es castigado convirtiéndose en mujer, pero una vez que
éste aceptaba de forma favorable su nueva forma femenina, es repentinamente
devuelto a su género original durante la intimidad con un hombre” (p. 31).

Por otra parte, Hipócrates describió un grupo de personas que vivió en Siria
y Palestina cerca del año 3000 a. C., a quienes nombró los no-hombres ya que
eran  similares a los eunucos pero sin estar castrados, con inclinaciones
femeninas y generalmente se dedicaban a los templos de la diosa del lugar (Lugo
2001).

Otro caso de transexualismo documentado fue el de William Sharp (1855-1905),
quien adoptó el seudónimo de Fiona Macleod durante la última década de su vida
con el propósito de expresar lo que él sentía como su alma femenina.
Se cree además que Sharp fue un caso de trastorno de identidad y personalidad
múltiple (Cox, 1966).

Actualmente en la India siguen existiendo las hijras, están tradicionalmente
socializadas y protegidas aunque sea en condiciones de marginalidad, viven en
pequeñas comunidades y se ganan la vida en el servicio doméstico o la
prostitución y con sus ingresos provenientes de rituales, debido a que se cree
que transmiten fortuna a los casados o a los recién nacidos, por lo que son
invitados a nacimientos o matrimonios.
Las hijras viven en comunidadeås de personas transgénero y personas en
condiciones de intersexualidad (generalmente hombres que se han castrado).
En los peldaños inferiores de esta escala social, las hijras viven una
existencia dura ya que se ganan la vida como bailarinas, prostitutas o mendigas
(Lugo, 2001).

El género ha sido tradicionalmente asociado con el sexo, basándose en la
anatomía genital para asignar un sexo y, consecuentemente, un género a un
individuo por lo que el estudio del género se encuentra ligado en alguna medida
a las ciencias que estudian diversos aspectos de la sexualidad.
Por lo tanto a la medicina y la biología les han sido asignados el rol de
evaluar las manifestaciones de los roles de género.
No sin trabas, pues el género, como se define actualmente, trasciende una
colección particular de características fisiológicas (Butler, 2001).

El principal discurso acerca del género que se constituye desde las ciencias
sociales está dominado por la teoría feminista y los estudios de género.
Luego del surgimiento de los movimientos de derechos de las mujeres, el
feminismo surge como postura ideológica que propone la igualdad de género como
meta (De Miguel, s.f.; Gamba, 2008; Guardia, 2013).
Psicólogos, sociólogos, filósofos, biólogos y la academia en general se volcó, a
partir de ese momento, a los estudios de la mujer.
Esto ha promovido un proceso efectivo de profesionalización de los movimientos
sociales, es decir, ha permitido que individuos forjen una carrera profesional
como lideres dentro de un determinado movimiento social con énfasis en la
reivindicación o visibilización de un hecho especial (Helfrich, 2001).

Así pues, el feminismo en primera instancia construye políticamente una nueva
forma de concepción del rol del género femenino.
Esto no llega automáticamente, ni carece tampoco de conflictos y
confrontaciones.
Pero comienza una visión del sexo y la sexualidad que interpela y cuestiona la
construcción tradicional.

La temática de conflicto interno más fuerte es quizás la definición del
trastorno mental (American Psychiatric Association and others, 2013).
Originalmente el discurso académico definía la desviación de las normas de
género como una enfermedad psiquiátrica.
Orientaciones sexuales distintas a la heterosexualidad eran concebidas como
aberraciones.
De hecho, el concepto mismo de homosexualidad se origina en su definición como
una afección clínica.

Existe detrás de esto una visión ontológica del ser humano principalmente
materialista, que dispone al cuerpo y su funcionamiento como los determinantes
de lo que debe ser.
Así, es inconcebible la alteración o modificación del sexo pues se presupone que
este está fijado en el cuerpo y manifestado a través de su expresión anatómica.

Hace falta el surgimiento del género como un rol socialmente construido y aparte
de la conformación del cuerpo para comenzar a considerar formas alternativas de
su expresión como no patológicas.
De la mano de los movimientos por la no discriminación de homosexuales y
bisexuales se da una alteración de algunas de las posturas académicas.
La noción de que la sexualidad puede ser una expresión independiente de la
función reproductiva y de la posesión de un sexo lleva a desligar ligeramente a
los roles de género de esta fijación corporal.

Sin embargo, la transexualidad y el transgénerismo siguen siendo considerados
como enfermedades mentales según la opinión de organizaciones que se encargan de
monitorear la salud como lo es la Organización Mundial de la Salud o la
Asociación Americana de Psicologia.
La Organización Mundial de la Salud, a pesar de encontrarse desde hace ya muchos
años en un proceso de reforma de sus diagnósticos psiquiátricos, continúa
sugiriendo el diagnóstico de la transexualidad como un trastorno.
A este se le considera el discurso bio-médico de la transexualidad (Helfrich,
2001).
Sumado a esto es necesario resaltar que para la edición mas reciente del Manual
de Diagnostico de la APA lo que se conocía como trastorno de identidad de género
ha sido adaptado como disforia de género, indicando así un cambio de visión
hacia la patologización que la condición trans ha vivido.

\subsection{Transgénero o transexualidad en Venezuela}
La transexualidad, como parece estar entendida en el imaginario colectivo del
venezolano, se asocia más comúnmente con las redes de prostitución de Trans
femeninos.
Se trata de hombres que, vestidos como mujer y con grados variables de
transición, ofrecen servicios sexuales.
Estas personas son concebidas de forma caricaturesca, pero constituyen una de
las poblaciones Trans más vulnerables.
Esto debido a que se exponen abiertamente para ejercer la prostitución, por lo
que son generalmente el blanco de las formas más frecuentes y gráficas de
discriminación.
Los testimonios recogidos por Lugo (2016) muestran esta realidad: Una de las
entrevistadas reporta: “Casi todos los días recibo amenazas de hombres”.

Esta cualidad caricaturesca y de perversidad extrema que se les atribuye en el
imaginario social también les ha hecho una población de estudio más común en las
investigaciones, generalmente, desde una perspectiva sanitaria-epidemiológica
(ONU Sida, 2012).
Esta particular expresión de la sexualidad Trans, posee muchas formas de origen.
Reproducción de ciclos de violencia y abuso, rechazo familiar, condición de
clase, vulnerabilidad financiera, entre otros factores que se mezclan para
generar este fenómeno.

En Venezuela, la Misión Negra Hipólita, cuyo objetivo es la erradicación de la
situación de calle mediante la inclusión de los ciudadanos que se encuentran en
esta situación, dedicó exclusivamente durante un tiempo uno de sus albergues a
la población de mujeres Trans sin hogar.
Este albergue, llamado Centro de inclusión social «Belinda Álvarez» funcionó
durante algún tiempo (Alianza Sexo-Género Diversa Revolucionaria, 2014).
Sin embargo la situación país, crisis económica y la falta de voluntad política,
terminó ocasionando el abandono por completo de esta iniciativa de asistencia
social (Asociación Civil Divas de Venezuela, s.f.; Orbita Gay, 2014).

Un elemento que es importante tener en cuenta al momento de observar tanto a la
transexualidad como al transgénerismo en Venezuela es la relación con
situaciones violentas y crímenes de odio que llegan a vivir transexuales y
transgéneros.
Medios de noticias como el diario Ultimas Noticias (en su versión virtual)
reporta en marzo del 2018 que un transexual fue asesinado de un disparo en su
cabeza mientras se presume esperaba clientes pues ejercía la prostitución
(Rojas, 2018).
Medios internacionales han hecho eco de este caso añadiendo que este es el
tercer asesinato a una persona transexual que sucede en Venezuela en menos de 9
meses (www.larepublica.pe, 2018).
Esto no solo visibiliza la situación de violencia a la que se ven expuestas las
personas trans, también muestra que una de las profesiones más frecuentemente
ejercidas por esta población es la prostitución, hecho que los expone a
infecciones de transmisión sexual.

Estas perspectivas llevan a inconvenientes a la hora de reconciliar nuevas
concepciones del género con formas tradicionales de atención médica y otros
aspectos relacionados.
Por ejemplo: el papel de las empresas aseguradoras en el mantenimiento del
bienestar del individuo ante situaciones inesperadas.
La población Trans se concibe en este sentido como vulnerable puesto que algunas
agencias internacionales, como la Organización Mundial de la Salud, aún definen
la transexualidad como un trastorno mental.
Esto tiene consecuencias directas en las personas Trans, como impedir su
afiliación a programas de seguro médico, limitar su acceso a la atención médica,
y además, en un aspecto pocas veces considerado, pone una limitante en las
habilidades de los profesionales de la salud para atender positivamente a esta
población.

\subsubsection{Perspectiva médica}
Algunos individuos transgénero optan por obtener ayuda médica con el fin de
cambiar la forma en que se ven.
Prato (2014) expone “Esto principalmente con aquellos que se consideran
transexuales dentro de la comunidad transgénero.
La terapia de reemplazo hormonal (HRT) está disponible para mujeres y hombres
Trans, dependiendo de si buscan una apariencia más femenina o masculina” (p.
37).

Pero en Venezuela, en una situación relacionada, los servicios de atención
psicológica y de salud sexual y reproductiva se encuentran generalmente sin
preparación para atender a la población Trans, o simplemente no existen
servicios de atención especializados.
Por ello se trata de una población vulnerable tanto al abuso, discriminación y
abandono, magnificado por su dificultad para el acceso a atención médica sin
discriminación.

\subsection{Diferencia entre transgénero y transexual}
Es de suma importancia para la presente investigación resaltar las diferencias
entre lo que es el transgénerismo y la transexualidad.
Para esto es necesario exponer claramente que es lo que condiciona a un
individuo para calificarlo como transgénero o transexual pues ambas identidades
tienen un elemento en común, el cual es un malestar asociado a la incongruencia
que existe entre su sentir, bien sea como identidad de género o sexo (Noseda,
2012).
Pero precisamente en este último punto yace la diferencia entre ambas
condiciones, para una persona transgénero el bienestar viene asociado con el
poder expresar una identidad de género congruente con la cual se sienten
identificados, mientras que para una persona transexual el cambio va no solo con
la expresión sino con el aspecto sexual (genital) de la misma.

Existe una diferencia marcada entre aquellas personas nacidas con genitales
masculinos que se identifican como mujeres (denominadas usualmente como MTF) y
quienes nacen con genitales femeninos y se identifican como hombres (FTM).
Estos son identificados usualmente con las iniciales correspondientes a las
expresiones en inglés: Male to female y Female to male.

Con base a lo planteado por Noseda (2012), se podría entonces considerar que una
persona transgénero no necesariamente pueda sentir malestar al identificarse con
sus genitalidad o sexo pero si con su género o expresión del mismo, sin embargo,
si esta persona decide cambiar de sexo se requiere de un tratamiento de por vida
con terapia de sustitución hormonal y cirugía de reasignación de sexo pasaría a
considerarse transexual.
Sin embargo, entre la expresión de vestimenta y meramente gestual, y las
distintas etapas de tratamiento se puede hablar plenamente de transexualidad
siempre y cuando exista un malestar relacionado con la genitalidad.
Indiferentemente del progreso o nivel de transición de sexo que tenga la
persona.
Es una referencia directa a que la persona se encuentra haciendo (o desea
iniciar) la transición de un sexo a otro.

Fundamentado en lo expuesto anteriormente la presente investigación se centra en
una población transgénero pues no ha existido cambio a nivel genital en los
participantes.
% TODO discutir. Se trata de transexuales porque iniciaron una transición,
% hay intención de cambiar

\section{Teoría Queer y performatividad de género}
Parece importante para poder adentrarse en el tema Trans tomar en cuenta
elementos teóricos que permitirían la apreciación de este fenómeno y esto no
puede lograrse sin mencionarse elementos propuestos por Judith Butler (2001).
Empezando por el concepto de que el sexo y el género son ambos constructos
sociales dependientes de un tiempo histórico y unas situaciones ambientales
particulares, tomándose entonces no como un elemento estático e inmutable sino
como maleable y flexible .
En obras posteriores de Butler (2010) la autora hace referencia explícita al
termino Queer, expresando que pasó de ser un término derogativo generado con
bases en la heteronormatividad a ser transformado en un término semánticamente
positivo, es decir al asumirse bajo esta bandera lo extraño, distinto y
subversivo se resignificó.
Otro autor que hace referencia a la resignificación de lo que significa queer es
Preciado en el portal www.paroledequeer.blogspot.com (2012) mencionando lo
siguiente:

\begin{quote}
    Desde el principio, “queer” es más bien la huella de un fallo en la
    representación lingüística que un simple adjetivo.
\end{quote}

Con esto preciado implica que inherentemente el término queer se encuentra
relacionado con lo que no encuentra calificación y que consecuentemente
transgrede lo que ha sido establecido como norma.

Además de esto es necesario mencionar que Preciado en su portal (2012),
propone lo siguiente sobre la teoría queer:

\begin{quote}
    …tiene por objetivo el análisis y la deconstrucción de los procesos históricos
    y culturales que nos han conducido a la invención del cuerpo blanco
    heterosexual como ficción dominante en Occidente y a la exclusión de las
    diferencias fuera del ámbito de la representación política.
\end{quote}

Con esto el autor remarca el papel transgresor de la teoría queer y la
importancia de la misma para resignificar elementos constitutivos de la
identidad que se asumen como inmutables e inalterables como lo son el sexo y el
género.
Desde esta visión se puede reforzar la idea de que el género es un constructo
social, pero además de esto asoma la posibilidad de considerar al sexo como un
elemento cuya construcción también puede estar fundamentada en elementos
sociales.
Más específicamente tomando en cuenta que aquella persona que se
entiende como autorizada para asignar un sexo u otro a un bebé depende de
criterios que han sido arbitrariamente elegidos.

Es por esto que se podría hablar de la Teoría Queer como aquella que afirma que
la orientación sexual y la identidad sexual (o de género) son el resultado de
una construcción social y consecuentemente esto significa que no existen
determinantes estrictos que asignen roles o papeles, sino formas variables de
desempeñar papeles sexuales en la sociedad (Fonseca y Quintero, 2009).

Por su parte autores como Solana (2013) plantea que desde la Teoría Queer se
puede reivindicar las identidades que bajo otros lentes podrían ser consideradas
incompletas, sean las de los sujetos Trans o de homosexuales que
performativamente rompan la norma.
Esta autora expresa que la teoría Queer aporta herramientas necesarias para
poder comprender la autenticidad de las prácticas de género.
Otro aspecto importante que rescata esta autora es que desde la teoría Queer se
puede reivindicar la pluralidad taxonómica con respecto a las categorías de
género.

Por su parte Fuss (1989 c.p. Fonseca y Quintero 2009, p. 42), plantea el
cuestionamiento: ¿existe realmente alguna identidad “natural”?, y menciona que
“la identidad no es más que un constructo político, histórico, psíquico o
lingüístico”;
y posturas como la de la Teoría Queer apoyan esta moción “rechazando toda
clasificación sexual y proponiendo destruir las identidades gay, lésbica,
transexual, travesti, e incluso la heterosexual, para englobarlas en un
“totalizador” que promueve un cambio social y colectivo desde muy diferentes
instancias en contra de toda condena” (Mérida c.p. Fonseca y Quintero, 2009,
p. 24).

La Teoría Queer permite replantearnos ideas sobre la concepción del género, las
identidades y las sexualidades en un marco de agudeza crítico con la finalidad
de desestabilizar no sólo al sistema, sino también a la academia.
El mayor aporte de ésta teoría radica en ofrecer nuevas explicaciones bajo un
marco conceptual en el que confluyen el género y la sexualidad así como los
significados y sus resistencias para dar origen a nuevas significaciones.
El término Queer ejemplifica este proceso, pues su significado es “homosexual”
desde una descripción peyorativa.

Fonseca y Quintero (2009) plantean desde una postura foucaltiana que al hablar
de “homosexualidad” se extiende el discurso homofóbico, y que dichas categorías
discursivas existen por la necesidad de representar a un sector político
oprimido;
esta clasificación lo que persigue es el control y la regulación de la práctica.
Pero aunque en inicios la teorización de lo Queer estaba para contribuir en la
destrucción de esta mirada peyorativa ante la homosexualidad, al ampliar su
margen ofrece una propuesta de romper con los patrones hegemónicos que nos
dominan a todas las personas desde distintos espacios.
Donde se describa al humano con distintas categorías sociales que buscan
definirnos según el color de la piel, la clase social, la edad, el peso y
claro está por el género, esta categorización que no es ingenua busca el
control.
Es por eso, que ésta propuesta teórica va más allá de la distinción sexo-género.

Para Mérida (c.p. Fonseca y Quintero, 2009), defender los postulados Queer no
significa combatir por un derecho a la intimidad, sino por la libertad pública
de ser quien eres, en contra de la opresión de la homofobia, el racismo, la
misoginia, la intolerancia que ha impuesto la religión como centro hegemónico
que regula el comportamiento social.
Es importante para esto distinguir que la propuesta teórica Queer no desestima
las otras categorizaciones que complejizan el mundo social, que necesariamente
están a la orden del control social hegemonizado y son los catalizadores para el
desempeño de un rol específico de acuerdo a categorías biológicas, sociales y
políticas;
sino más bien trata de problematizar la construcción del género y de
las otras estructuras clasificatorias y restrictivas.

La diferenciación de los géneros como formas antagónicas de estar en el mundo
posibilita entonces asimetrías de poder, con lo cual asumimos que el género
estructura la vida social, que se plantea a partir de la categoría del sexo para
dar origen a la construcción colectiva del género y su rol y participación
social, dentro de la cual no escapa indubitablemente la sexualidad, aunque en
apariencia ésta se encuentre en una esfera de lo privado.

Dar un nombre a un niño o niña es el comienzo del proceso por el cual se le
impone la “feminización” o “masculinización”, y a su vez, una forma de marcar
aquello que es masculino con una franca diferenciación de lo femenino.
La feminidad, por ende, “no es producto de una elección, sino de la llamada
forzosa de una regla cuya compleja historicidad es inherente a las relaciones de
disciplina, regulación y castigo” (Cubillán, 2012, p. 32).

Por tanto, de ninguna manera el género debe entenderse como una elección o un
artificio que podemos intercambiar.
Es precisamente esta elección lo que nos antecede como sujetos sociales, que
viene a definirnos dentro de un mundo colectivo y donde el género es
significativo pero no la única característica definitoria, de quiénes somos y
cómo debemos comportarnos para incorporarnos a la dinámica social (Cubillán,
2012).

Aunque no nos suscribimos a la Teoría Queer como un planteamiento para eliminar
las etiquetas sociales que nos identifican, “reconocemos su carácter político
para reivindicar la pluralidad, desde la posturas del feminismo postmoderno, nos
parece vital reconocer la multiplicidad de voces que se presenten dentro de
una misma categoría” (p. 91) y es desde este planteamiento que realiza Butler
(2010) que asumimos el carácter performativo de la identidad sexual.

\section{Antecedentes de la investigación}
En la actualidad tanto la transexualidad como el transgénerismo se encuentran
dentro de la mirada pública pues se ve expuesta por hechos registrados dentro de
los medios de comunicación como lo son la transición de Catlin Jenner o el caso
del Hijo de la cantante Karina.
A raíz de esta exposición nace un interés en comprender estos fenómenos.
La mayoría de los expertos como Lugo (2016) consideran que el hecho de ser
transgénero no está causado por un solo factor.
Creen que es la consecuencia de una compleja mezcla de factores biológicos,
psicológicos y ambientales, no solo una cuestión de gustos.
Actualmente se encuentran pocas investigaciones con una mirada Psico-Social en
esta área, lo cual motivó la elaboración de la presente investigación.

Desde la perspectiva médica, ha existido una visión dominada por el
biologicismo con respecto al tema de la identidad Trans.
Autores como Lugo (2016) plantean que se puede identificar sexualmente a los
individuos según su sexo cromosómico, gonadal, hormonal, embrionario, fenotípico
genital y sexo psicológico.
Con esta clasificación se puede hacer evidente que el papel que existe en la
construcción sexual con base a funciones biológicas pues, desde esta
perspectiva, existe una mayor cantidad de elementos biológicos que
permitirían determinar el sexo y el género de un individuo que psicológicos o
sociales.

Esto podría parecer reduccionista, pues limita la identidad, rol y demás hechos
asociados al género a una expresión biológica que puede no estar en concordancia
con la identidad del individuo.

Desde la perspectiva social, autores como Martínez-Guzmán (2012) plantean que
para poder romper o re-significar dualidades como sexo-género y hombre-mujer es
necesario emplear una mirada psicosocial a la construcción de identidades Trans,
llevando a un plano más tangible y real lo que es la expresión y vivencia del
género, pues la construcción que se tiene en la actualidad está marcada por
tintes biologicistas que pueden llegar a ser reduccionistas ante la complejidad
que significa ser humano y vivir experiencias que no son siempre comunes para
todos.

Desde la perspectiva psicológica, el principal enfoque que tiene la
ciencia psicológica al acercarse a la condición Trans es desde una mirada
clínica, principalmente enfocada en el uso de manuales de diagnóstico como el
DSM-V (2014).
Es por esto que la perspectiva psicológica enfocada a la condición trans abarca
investigaciones realizadas en años recientes como la de Pinto y Rocha (2012) en
la que exploraron las vivencias subjetivas de personas transexuales en la ciudad
de Caracas, o la de Maiz (2014) en la que la autora exploró en mujeres
transexuales cual es el concepto de sí mismas así como la imagen corporal que
tenían.
Ambas investigaciones (realizadas por estudiantes de la Escuela de Psicologia de
la Universidad Central de Venezuela) se encuentran enmarcadas dentro de una
visión clínica de lo que es la transexualidad.
Es decir, lo transexual como disforia de género.

En miradas distintas a la psicológica se pueden tomar en cuenta investigaciones
como las realizadas por Bolívar y Arrizure (2014) quienes desde una mirada
fundamentada en las ciencias sociales se han adentrado en el fenómeno trans,
tomando como foco principal la discriminación que vive la comunidad LGBT en la
parroquia Sucre del municipio Sucre en el estado Carabobo.
Los investigadores a lo largo de la investigación encontraron la prevalencia de
estereotipos ligados a la comunidad LGBT y consecuentemente a la población trans
ya que la interacción entre personas cisgénero
\footnote{Término utilizado para referirse a aquellas personas cuya identidad
de género corresponde con su sexo biológico.}
y transgénero de la parroquia
Sucre se ve permeada por elementos presentados por medios de comunicación así
como prejuicios basados en la falta de información y estereotipos.
Este es un elemento que se puede rescatar de otras investigaciones como la de
Gerdel (2012) cuyo foco de interés estuvo centrado en los estereotipos de género
influenciados por los medios de comunicación social, a pesar de que no se
encontraba centrada en la transexualidad o el transgénerismo se puede reforzar
la idea de los medios de comunicaron actúan como potenciadores de estereotipos
de género que contribuyen a la desinformación y prejuicios hacia estas
condiciones.

Sin embargo es necesario resaltar el hecho que la transexualidad validada desde
la visión médica, ha traído como consecuencia una patologización, con base a qué
instrumentos diagnósticos como lo son el DSM- IV y el CIE-10 considera a la
misma un trastorno de identidad sexual.
Mientras que el DSM- V (American Psychiatric Association and others, 2013) lo
considera como una disforia de género en adolescentes y adultos con
postrasnsición, es decir, aquellas personas que poseen una marcada
incongruencia entre el sexo que siente o expresa y el que se le asigna, y que
éste ha realizado la transición a una vida de tiempo completo con el sexo
deseado (con o sin legalización e intervenciones quirúrgicas del cambio de
sexo).
A pesar de esto, la OMS ha determinado que en la edición a publicarse en el año
2018 del CIE-10 deje de tratarse como trastorno y presentarse como incongruencia
de género (Borraz, 2017).

Sumado a esto, Vargas (2017) explica que se debería considerar a la identidad
sexual como un continuo, no como dos categorías estrictamente definidas con base
a características biológicas, principalmente porque pese a que existen las
mismas y pueden ser comunes entre los individuos no necesariamente todos las
comparten y las mismas pueden variar en mayor o menor medida dependiendo del
sujeto.

\subsection{Identidad Corporal e Identidad Sexual}
En el transexualismo existe mejoría en el malestar que vive el individuo a
través del tratamiento quirúrgico de reasignación de género, en lugar de que el
tratamiento psiquiátrico alivie o elimine los síntomas.
De hecho, existe gran cantidad de literatura científica que evidencia la gran
satisfacción y óptima adaptación social posterior a la cirugía sin presentar
dudas sobre su nuevo género y sexo en personas transexuales (Blanchard y
Fedoroff, 2000)

Esta característica del razonamiento humano, puede fallar cuando se aplica a los
conceptos de especie y género, siendo la transexualidad y otros aspectos del
sentimiento de identidad muestra de ello, ya que se sabe que la sexualidad
humana es muy amplia y por ende, es imposible catalogarlo todo dentro de las
categorías Hombre y Mujer sólo por la presencia de genitales determinados
(Blanchard y Fedoroff, 2000).

Siguiendo una línea de pensamiento similar Salin-Pascual (2007), plantea que
cuando se habla de dos sexos (masculino, femenino), se está abarcando en esta
dicotomía un disciplinamiento de aspectos muy complejos de la sexualidad humana.
Siendo tan fuerte el dogma de la dicotomía anatómica, que cuando no se la haya
se la reproduce, como en el caso de niños recién nacidos cuyos genitales son
ambiguos, no se revisa la idea de la naturaleza dual de los genitales, sino que
se disciplinan para que se ajusten al dogma.

Es por esto que resulta necesario hacer presente elementos que resultan
importantes en la constitución del individuo como lo es la identidad sexual.
López (1984) hace referencia a que es un proceso que se da desde edades
tempranas del desarrollo de un individuo y le permiten establecer una relación
con el sentirse como niño o como niña (y consecuentemente hombre o mujer
en un futuro) y que estas a su vez le permite generar y moldear su
comportamiento para cumplir con expectativas socialmente construidas hacia el
sexo con él que se identifica.
Esto es reforzado por lo expresado por Velandia (2016) cuando afirma que “El
\emph{querer ser} prima sobre el \emph{deber ser} en la medida en que
reafirma las diversas identidades sexuales” (p. 303).
Con esto el autor está dando un peso importante dentro de la constitución de la
identidad sexual al elemento volitivo de la conducta, esto permitiría afirmar
que la identidad sexual no es un elemento inmutable, fijo y determinado
únicamente por el sexo biológico.
Si es cierto que tiene un peso importante sobre la construcción de la identidad
pero también lo tiene el sentirse identificado con ese sexo, o con un sexo en
específico a pesar de que no coincida con la biología del individuo.

Tomando estos aspectos como punto de partida se hace necesario hablar de la
ruptura entre la identidad corporal y la identidad sexual y lo que esto puede
causar en un individuo.
Esto es un fenómeno llamado Disforia de Género, según lo planteado por autores
como Hurtado (2015) y Fernández, Guerra, Diaz y Grupo GIDSEEN (2013) se puede
entender como una condición en la que el individuo se ve afectado por la
identificación con un sexo y género contrario a su sexo biológico y género
socialmente esperado.
Esta condición ha sido recientemente incluida dentro del DSM-V como lo plantea
Lorenci (2013) siendo un cambio en terminología y en parte en diagnostico de lo
que se conocía como Transtorno de la Identidad de Género, pues como lo plantea
este autor, en la condición denominada Disforia de Género se encuentra un
elemento importante para el diagnostico y este es el malestar que puede sufrir
el individuo al verse en un cuerpo disfórico.

\subsection{El cuerpo como construcción social}
El cuerpo, cada vez más, se ha convertido en un espacio de interés para las
ciencias sociales.
El llamado proceso de “socialización de la naturaleza” evidencia que todo lo
humano es en definitiva una experiencia social y el cuerpo no escapa de esta
conclusión.

La noción de que el cuerpo es una realidad biológica queda cada vez más
invalidada y no solo con la comprensión de cómo los fenómenos sociales inciden
en su conocimiento sino con eventos tan tangibles como la modificación del sexo
biológico.
Crear el propio cuerpo a través del uso de tecnologías es una práctica muy
antigua pero no solo es un cambio de cómo se ve el cuerpo;
cada modificación responde a un contexto social en cual se intenta transmitir
algo a los otros y otras (Giddens, 2000).

En torno al uso del cuerpo encontramos una serie de normas que a lo largo del
desarrollo humano han sido delimitadas y jerarquizadas según el género las
cuales regulan desde el caminar y el aspecto externo hasta el manejo de las
sensaciones físicas;
la mayor parte de las veces estas normas no se enuncian sistemáticamente sino
que “se expresan verbalmente en forma negativa e indirecta a través del
llamado de atención, la burla, el desdén condescendiente, el desprecio o la
indignación moral” (Boltanski, 1975, p. 58), son además un código propio de
cada comunidad y naturalizado por esta, por lo que para su estudio y comprensión
es necesaria la observación sistemática de las prácticas y la comparación con
distintos grupos sociales.

Este conjunto de normas implícitas, los rituales y actividades realizadas en
torno al cuerpo pueden ser agrupadas bajo el rótulo de “cultura somática”
(Boltanski, 1975, p. 85).
Esta cultura es el producto de unas condiciones materiales de existencia que son
a su vez generadas y mantenidas a través de un orden cultural;
en otras palabras, la forma en que se usa y cuida el propio cuerpo es
resultado de las maneras en que se es capaz de consumir y acceder a los
bienes y servicios que a su vez está determinada por las formas en que
respondemos a las exigencias del entorno social.

Además del entendimiento del cuerpo como un elemento donde confluyen una serie
de formas de relación social, es necesario visibilizar que estas relaciones no
son neutrales sino son relaciones de poder.
En el cuerpo como objeto de interés psicológico se observan como constante los
rastros de la ubicación de su propietario en torno a otros sujetos sociales y
cómo las diferencias en el ejercicio del poder dejan marcas en este.

\subsection{Expresión de género e imagen corporal}
En psicología la apariencia se presenta constantemente como sinónimo de la
imagen corporal.
Hace referencia a todo lo que tiene que ver con cómo se percibe el cuerpo desde
afuera.
En otras palabras, como se ve el cuerpo en el espacio social: esos estereotipos
que nos dan un esbozo de las expectativas sociales (el modelo de belleza), las
formas que se supone podemos darle a los cuerpos para aproximarse al ideal, y
cómo ese modelo se relaciona con el consumo y lo “saludable”.
Así mismo la forma en la que un individuo se muestra o se \emph{expresa} puede
estar
cimentada en base a la construcción de género que tiene, así como su expresión
del mismo.

Cuando se habla de expresión de género se puede entender que es:

\begin{quote}
    La externalización que hace la persona, a través de la conducta, vestimenta,
    postura, interacción social, etcétera, de su identidad de género (Negro, 2010 p.
    157).
\end{quote}

Sumado a esto se puede rescatar lo propuesto por la APA (2011):

\begin{quote}
    La expresión de género se refiere al modo en que una persona comunica su
    identidad de género a otras a través de conductas, su manera de vestir,
    peinados, voz o características corporales (p. 1)
\end{quote}

Con estos referentes es posible establecer la noción de que la expresión de
género está ligada a la construcción de la identidad de género de un individuo y
que consecuentemente se ve afectada por la imagen corporal del mismo.
Pues desde esta perspectiva una persona trans (bien sea transgénero o
transexual) puede verse en situaciones aversivas por tener una identidad de
género distinta a la que se le puede permitir expresar por condiciones sociales.
Así mismo, la imagen corporal de una persona trans puede verse alterada al
establecer una relación inadecuada entre lo que puede expresar y lo que quiere
expresar.

En la cotidianidad encontramos que aspectos como ser delgado, vestirse a la
moda, haber tenido cirugías plásticas, tener la sonrisa y las proporciones
perfectas se vuelven más relevantes que otros aspectos de ser personas;
casi podría decirse que la personalidad, entendida como los rasgos y
comportamientos propios de cada persona, es menos relevante frente a la
apariencia que es capaz de procurarse cada quien (Díaz y Pérez, 2010).

Aunque la apariencia corporal pareciera ser un aspecto superficial es indudable
que ella media la forma en que nos relacionamos con otros (familiares, amigos,
compañeros de trabajo y la pareja) por ello se hace imperativo gestar acciones
destinadas agenciar este elemento tan relevante.

Los moralistas consideran que la apariencia carece de valor en la medida en que
es un producto de los buenos genes y el acceso a bienes y servicios;
pero actualmente se hace evidente que la superficie corporal es un lugar
donde quedan inscritos muchos mensajes, es en esta superficie donde:

\begin{quote}
Escribimos e inscribimos todos los días buena parte de nuestras
resistencias, subordinaciones y nuestros intentos de alterar los juegos de
poder en que somos poseídos y poseemos.
Si advertimos la inversión que en cuanto a tiempo, recursos y trabajo humano
destinamos al modelo y tratamiento de la apariencia corporal mediante la
manipulación de las superficies, podemos entender que es todo, menos trivial
(Gómez y González, 2007, p.53).
\end{quote}

Por otro lado, para autores como Exner y Sendín (1998) la \emph{autopercepción}
es un
conjunto de aspectos descriptivos y valorativos que la persona ha ido elaborando
para lograr un auto conocimiento y una auto-valoración, ajustada a la realidad,
acerca de sí mismo.
Entre los elementos que conforman la autopercepción los autores
anteriores mencionan: la autoimagen, la autoestima, estimación de la valía
personal, el autocentramiento, autoconcepto, identidad, entre otros.

Exner (2000), expone que el término alude a dos conceptos, la propia imagen y la
relación con uno mismo.
El primero (la autoimagen) se construye a partir de las impresiones que cada
individuo posee de sus características, presumiendo que parten de experiencias
basadas en la realidad.
Parte de estas impresiones se mantienen en un nivel consciente mientras que
otras están de forma parcial o en su totalidad inaccesibles a la consciencia,
quizás por su naturaleza disonante, estas son reprimidas.

Desde otra mirada se puede tomar en cuenta lo planteado por autores como
Salaberria, Rodriguez y Cruz (2007) quienes expresan que desde las bases
sociales y culturales se puede establecer patrones de apariencia física que
demarcan lo que se considera bueno y lo que se considera malo.
Con esto en cuenta se podría afirmar que la expresión de género de un individuo
se ve permeada por este elemento pues es aquel que modela como va a ese
individuo dejarse ver por otros así como este se va a sentir consigo mismo.

Por otro lado, la segunda, emana de la autoimagen y hace referencia al nivel en
que una persona tiene su atención orientada hacia los aspectos propios, en
contraposición a estar en contacto con el mundo exterior.
Esta relación consigo mismo puede configurarse en un marco tanto positivo como
negativo, y sea de una u otra manera termina siempre influyendo en los objetivos
que cada persona se propone.

Ahora bien, autores como Kottow y Bustos (2005) entienden por \emph{cuerpo}  un
complejo constituido por la interacción que se da entre lo físico (el cuerpo
como tal) y la mente (entendiéndose como conjunto de emociones, experiencias y
nociones del ser).
Desde esta mirada se podría presumir que en caso de existir algún elemento que
impida una adecuada interacción entre cuerpo y mente, como por ejemplo que el
sexo de un individuo no esté alineado con su identidad de género, causaría un
malestar muy intenso en el individuo.

Por su parte, Schilder (1999) emplea la expresión \emph{imagen corporal} para designar
“una representación a la vez consciente e inconsciente de la posición del cuerpo
en el espacio, encarado en sus tres aspectos de sostén fisiológico, estructura
libidinal y significación social”(p. 20).
Asimismo, la señala como la representación  mental que cada individuo genera de
su cuerpo, promoviendo así la conformación de su identidad.
Tal como señala el autor, es lo que uno imagina y percibe respecto su propio
cuerpo, asimismo involucra al nivel de satisfacción que tiene la persona sobre
su cuerpo.

Rosen (1995 c.p. Raich, 2004) define la \emph{imagen corporal} como “un concepto que
se refiere a la manera en que uno percibe, imagina, siente y actúa respecto a su
propio cuerpo” (p. 16).
Esta definición es complementada por la siguiente afirmación de Montaño (2004)
“la imagen corporal es una experiencia psicológica multidimensional relacionada
con la ‘figura’ que influyen profundamente en la calidad de vida de las
personas” (p.12).
En otras palabras la imagen corporal proviene de la experiencia con el propio
cuerpo para su percepción como aceptable o no, se relaciona con muchos elementos
más allá del mundo “privado” del individuo y afecta definitivamente su proceso
vital.

\subsection{Constitución de la subjetividad: Transición, afrontamiento y
aceptación}
Un factor que parece importante dentro de lo que determina no sólo la identidad
de un individuo sino también la forma en la que este se relaciona con otros es
la subjetividad.
Esta se podría asumir esencialmente como los procesos que significan
estructuras que un sujeto conoce y construye (Düsing, 2002).

Autores como González y Cabrera (2013) rescatan que la subjetividad y la
psicología poseen una estrecha relación sobretodo en el campo social, estos
autores expresan que la subjetividad social es la forma en la que se integran
sentidos subjetivos a diferentes configuraciones subjetivas de espacios
sociales.
Es decir, el individuo interactúa con un espacio así como con otros individuos
en base a una construcción previa, generada en base a factores como la
socialización primaria y secundaria.

Un punto importante a considerar, es la \emph{transición} experimentada por las
personas tanto transexuales como transgénero, donde autores como Franco (2011)
exponen que estas personas continuamente buscan por medio de su cuerpo, sus
expresiones corporales y su aspecto físico imponer, posicionar o expresar cosas
acerca de la identidad, roles, clase, prestigio e intencionalidad.

Es por esto que la modificación corporal ha tenido tradicionalmente un fuerte
impacto simbólico para la expresión de la \emph{identidad transexual}.
Pero la transexualidad trasciende estas expresiones.
En este caso es el deseo mismo de tener otro cuerpo o en palabras de las mismas
personas Trans, haber “nacido en el cuerpo equivocado” (Franco, 2011).
Esto es porque la condición llamada disforia de género surge cuando existe una
ruptura entre la identidad sexual y la expresión de género, causando malestar en
el individuo y siendo una de las formas de alcanzar esta alineación por medio de
operaciones correctivas y expresiones de género concordes.

Con base a lo planteado anteriormente se podría afirmar que el deseo de
modificar el cuerpo no es un acto expresivo únicamente, sino que es un acto
correctivo.
Es la alineación del cuerpo al sentir de la identidad del yo.
Esta expresión, esta posibilidad de transición, está marcada transversalmente
por la existencia de la misma como posibilidad material y simbólica.
Sumado a esto se puede afirmar que para muchas personas Trans (bien sean
transexuales o transgénero), antes de conocer de la existencia de esta
posibilidad identitaria, lo primero es una puesta en cuestionamiento de la
orientación sexual, esto es debido a que si se carece del acceso cultural a las
ideas y nociones de género que permiten una fluidez en la expresión o un cambio
por completo, se hace difícil engendrar la \emph{identidad Trans}.
Los mismos testimonios dan fe de una concepción de las etapas tempranas de la
constitución de la identidad Trans como una confusión.

Esto significa que la posibilidad real de acceder a cierto campo cultural altera
las formas en las que la identidad Trans se expresa.
No es equivalente la formación cultural del hombre que quiere ser mujer nacido
en el interior del país en un pueblo pobre.
Esta diferencia altera la capacidad de concepción de una trayectoria de vida
asociada con la transición de sexo.
Pero esta posibilidad ha ingresado en el imaginario colectivo de una forma que
puede ser fácilmente reconocido (Barrera, 2013).

Por otro lado, un aspecto importante a considerar respecto a la transición
vivida por una persona Trans, depende también del poder adquisitivo, el
conocimiento y la auto-confianza para ubicar un médico seguro y calificado,
donde en ocasiones estas personas deben confrontar a la transfobia y la
ignorancia alrededor del proceso de cambio que quieren llevar a cabo.
Dicho proceso la mayoría de las veces tiende a ser Complejizado por el rechazo y
dificultad que tienen las personas Trans para realizar una inserción laboral y
social exitosa pues, si los resultados no son los esperados, los deja en
posiciones vulnerables y expuestos a daños físicos o la muerte.
Cuando sin orientación y sin recursos intentan realizar la transición por su
cuenta se exponen a inyecciones de químicos dañinos.
Deterioro de nervios, operaciones con médicos no calificados y la
automedicación de terapias sustitutivas de hormonas son algunos de los
riesgos a los que se expone la persona Trans que, al carecer de la
información y de los medios económicos para una transición segura,
intenta desesperadamente hacer coincidir su cuerpo con su identidad (Barrera,
2013).

Según Rubio (2016) portales en línea como “Fundación Daniela”, hacen referencia
a lo engorroso que puede ser el proceso de la resignificación de identidad en
una persona Trans.
Principalmente expresan que gran parte de los sujetos se ven expuestos a malos
tratos a nivel familiar, profesional y entre amigos.
Dicha institución remarca la importancia del acompañamiento que se le debe hacer
a las personas Trans.

Salin-Pascual (2008) expresa que los conflictos más frecuentes con los que se
encuentran las personas Trans son el de temor, principalmente existente por el
hecho de transgredir la norma social y ser identificado como transgresor;
frustración por sentir que se vive una situación injusta al no tener una
relación normativa entre su sexo y su género.
Este autor expresa que dependiendo de las herramientas con las que cuente la
persona Trans podrá en mayor o menor medida sobrellevar estos conflictos.

Este mismo autor, inspirado en las fases creadas por Elizabeth Kubler-Ross en su
libro “Etapas de la muerte y de morir”, constituyó siete fases que suelen
recorrer las personas para lograr la aceptación de una persona Trans, donde se
parte de la fase uno, que hace referencia a cuando la persona se da cuenta y se
sorprende;
la segunda fase donde aparece la Negación y Culpa;
la tercera fase donde aflora el Miedo y Culpabilidad, seguido de la fase
cuarta donde la persona está Aprendiendo a Entender, para luego pasar a la
fase cinco donde se encuentra Buscando Aceptación, después la fase seis donde la
persona se encuentra Saliendo del closet para finalmente llegar a la fase
siete que va Más allá de la Aceptación hasta el Orgulloso Activismo.
