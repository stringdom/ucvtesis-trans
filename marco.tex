% Miércoles, 19 de julio de 2017
\chapter{Marco referencial}\label{marcoreferencial}

\section{Género}

Para poder adentrarnos en el tema que es el foco de la presente investigación es
necesario primero realizar una revisión de lo que es el género. Aunque a primera
vista eso podría parecer un tema relativamente sencillo, no se debe tomar a la
ligera, pues como lo indica \textcite{Lamas1999} el emplear la palabra
género conlleva implicaciones históricas, en un primer momento el feminismo
académico anglosajón usó este término para diferenciar las construcciones
sociales y culturales, pero esto probó ser complicado pues dio pie a que
surgieran una variedad de formas de interpretación, simbolización y organización
de las diferencias relacionadas con la sexualidad y el sexo complejizando de
esta manera la concepción inicial y la aplicación que se le había dado al
concepto de género. Sin embargo Lamas rescata una propuesta por Scott (1999 c.p.
Lamas) que propone que el género posee dos partes analíticamente
interrelacionadas, aunque distintas, y cuatro elementos. Lo central de la
definición es la "conexión integral" entre dos ideas: el género es un elemento
constitutivo de las relaciones sociales basadas en las diferencias que
distinguen los sexos y el género es la forma primaria de relaciones
significantes de poder. Con esto en mente es posible que establezcamos una
primera característica para poder adentrarnos en lo que significa el género, y
esta sería que está determinada por las relaciones sociales y las diferencias
que se asignan según el sexo (biológico) de las personas.

Se puede también rescatar lo propuesto por \textcite{Bourdieu2000} cuando se
refiere al hecho de que la socialización y la construcción de los géneros ha
sido históricamente demarcada y delimitada por la biologizacion de lo social,
para generar una división arbitraria de lo que implica haber nacido con un sexo
y tener impuesto un género y unos roles en base a este determinante biológico.
Este autor se adentra en esto planteando que las diferencias visibles entre los
cuerpos se convierten en el factor determinante para promover una visión
androcéntrica del mundo, dando así significados y valores particulares según sea
el caso. Entonces esto permite reafirmar el papel del género como una
construcción social, pues se encuentra ligado a la forma en la que las distintas
sociedades le asignan valores, y tal vez más allá de eso símbolos, a
características biológicas ligadas con elementos reproductivos que permiten a su
vez validar una visión de la vida en la cual los individuos son valorados según
su componente biológico.

\subsection{Masculinidad}

De acuerdo a \textcite{Burin2003} en la actualidad factores como la clase
raza y orientación sexual se han convertido en determinantes de la
diferenciación masculina es por esto que en palabras de la autora, existen
distinta masculinidades. Esto no es algo descabellado pues como ya lo hemos
expresado anteriormente la construcción de la masculinidad es un proceso que
viene determinado por las interacciones del sujeto con otros individuos del
mismo género así como con individuos del género contrario. Es entonces aquí
donde se encuentran construcciones sobre lo que significa ser hombre tan
variadas como hombres hay, consecuentemente la construcción de un hombre único
con el que se pueda identificar un individuo es bastante difícil.

Autores como \textcite{Tellez2011} afirman que el "hacerse hombre" puede ser
considerado un proceso de construcción social en el que se le asigna a lo
masculino una serie de características definidas por la sociedad que tienen como
finalidad mantener la experiencia exclusiva del poder a aquellos individuos
masculinos. Sumado a esto los autores plantean que por medio del estudio de la
masculinidad se puede revelar un sistema en el que el género actúa como medio de
control e imposición de actividades sobre los individuos de una población.
Además es necesario remarcar que, en palabras de \textcite{Tellez2011}, la
concepción de lo que se considera que es masculino ha variado según factores
como el momento histórico, la etapa evolutiva, el nivel socioeconómico, entre
otros.

Las autoras también hacen énfasis en que a la masculinidad suele exigírsele una
expresión mas activa. Es decir, se debe demostrar ser hombre según tu etapa de
desarrollo. Tomando todo esto en cuenta se podría reafirmar la noción de que no
existe una forma única y correcta de lo que significa ser hombre, ya que la
masculinidad está construida en base a aspectos sociales y que los mismos no son
estáticos e inmutables.

\subsection{Feminidad}

\textcite{Burin2003} además de adentrarse en el tema de la subjetividad
masculina también toca a la subjetividad femenina, la cual al igual que el caso
de la masculina viene determinada por aquellas interacciones que tiene un
individuo y que le permiten construir un ideal de feminidad, el cual no puede
estar ligado a una concepción fija de lo que es el ser.

Además de lo anteriormente mencionado, \textcite{Martinez-Herrera2007} reafirma
que históricamente el género se ha presentado en un continuo. No es sólo una
construcción social sino que también actúa como filtro cultural que permite
acercarse e interpretar de una manera particular al mundo. Con esto en mente se
podría entender entonces que la experiencia masculina y la femenina suelen
colocarse en extremos opuestos y con connotaciones particulares. Es por esto que
se asume al hombre como representante universal de la humanidad a diferencia de
las mujeres. Este autor citando a Lagarde, plantea que la feminidad
históricamente ha sido concebida en función de ser para otros, es decir, la
imagen de mujer que ha sido construida se ha enfocado en que su existencia
adquiere sentido y reconocimiento en tanto se dedique a la asistencia y el apoyo
a otros. De aquí nace la imagen de que una mujer no se encuentra completa o
realizada hasta que no complete una suerte de profesa al convertirse en madre.

\textcite{Lagarde1990} plantea que debido a los cambios sociales, y al hecho de
que en su mayoría los roles de género se encuentran demarcados por situaciones
especificas de un tiempo histórico, se ha visto un cambio en la percepción de la
feminidad el cual se ha sido en principio apoyado por movimientos feministas.
Estos han resignificado a la mujer así como también se ha visto afectada la
concepción de la masculinidad.

Es necesario entonces demarcar que existe una construcción de subjetividades
masculinas y femeninas, y que las mismas pueden ser sumamente variadas. Pero se
debe resaltar que esta variación permite explicar el por qué se podría
establecer subjetividades que trascienden a una asignación binaria.

\subsection{Trans}

No existen mucho estudios en base a la subjetividad de la identidad trans como
para tener un consenso pero pareciera poder justificarse en base al hecho de que
la construcción de la identidad es un hecho interno y privado y que la misma
viene determinada por las relaciones que forma el sujeto.

Según la autora \textcite{Dowshen-Atanda2014} ser trans no es lo mismo que ser
homosexual. Ser transgénero tiene que ver con la identidad de género, la forma
en que la persona se ve a sí misma y el género con el que se identifica. Ser
homosexual, es decir, ser gay o lesbiana, tiene que ver con la orientación
sexual, el género por el que se siente atraído. Muchos homosexuales están
cómodos y se sienten bien con su género, no desean tener un género diferente al
que poseen, lo que ocurre es que se sienten atraídos por personas de su mismo
sexo. Puesto que la orientación sexual es distinta de la identidad de género,
una persona transgénero puede ser heterosexual, homosexual (gay o lesbiana) o
bisexual.

Entonces, ¿cómo se puede definir lo trans? Pues la APA \parencite{APA} plantea
que se puede hablar de transgénero, que es un término que define a personas cuya
identidad de género, identidad o conducta no se ajusta a aquella con la que se
le debería ver asociado por su sexo biológico. Por otra parte se puede hablar
también de transexualidad, en este caso se plantea una identidad de género que
es diferente al sexo asignado y que está ligada a un deseo de alterar el cuerpo
para alcanzar ese sexo con el que sí existe una identificación a pesar de la
diferencia biológica.

Se podría entonces tomar en cuenta que existen deseos parecidos en ambos casos,
tanto en personas transgéneros como en personas transexuales, el foco principal
se encuentra en la transición para llegar a ese sexo o género con el que se
identifican. Es por esto que la American Psychological Association plantea que
el uso de la palabra trans es adecuado pues engloba a ambas identidades.

% jueves 20 de julio de 2017

\section{Roles de Género}

Con base a lo planteado anteriormente es necesario rescatar lo planteado por
Hernández (2000)\todo{Esta referencia Hernández 2000 no se encuentra}, según
este autor esta diferenciación entre los sexos, asociada a una diferencia
en normas y valores que pueden tener como consecuencia que
hombres y mujeres se vean víctimas de expectativas sociales que finalmente
afectan a su comportamiento y desempeño, se puede llamar rol de género.
Entendiendo el género como una categoría que abarca aspectos psicológicos,
sociales y culturales ligados a la feminidad y masculinidad que está ligada a
un proceso histórico de construcción social.  Autores como
\textcite{Gonzalez2013} expresan que esta construcción de roles se puede
presentar según tres aspectos principalmente dentro del ambiente escolar: el
trato con otros miembros del grupo, la vestimenta y el juego. Estos elementos
moldean como se debe comportar una persona según su género y que además son
influenciados significativamente por los contextos en los que se desenvuelve el
individuo.

Con esto en mente se puede entonces plantear que según la socialización del
individuo se construyen roles de género diferenciados dependiendo del sexo biológico que
poseen. Y, que se busca mantener una consonancia entre lo que viene asociado al
mismo, es decir, un individuo con un pene que se ve como un hombre debe mantener
una conducta masculina y un individuo con una vagina que se vea como mujer
deberá mantener una conducta femenina, e ir en contra de esto significa romper
con roles que han sido ya históricamente establecidos.

\subsection{Socialización diferencial de género}

Como fue planteado en párrafos anteriores, el género se ve definido por las
relaciones sociales que se establecen entre los individuos. Consecuentemente se
puede pensar que existe un proceso histórico que, como lo plantea \textcite{Bourdieu2000},
es continuo trabajo de diferenciación al que los hombres y mujeres se ven
sometidos. De esta manera se distinguen masculinizándose o feminizándose. Es
por esto que es necesario describir esta construcción social de lo que significa
ser hombre o mujer, pues se encuentra ligada con la identidad de las
personas trans.

\textcite{Bourdieu2000} expresa que principalmente el trabajo de la reproducción
de esta construcción de la identidad ha venido estructurado por la familia, la
iglesia y la escuela. Estas permiten a los jóvenes construir una identidad con
base a las estructuras subyacentes. Adentrándose más en este aspecto se puede
entender que el primer vínculo que establece una persona es con la familia, como
lo plantean \textcite{Berger1991} la socialización primaria es aquella que un
individuo atraviesa durante la niñez y por medio de la misma se convierte en un
miembro de la sociedad. Es por esto que la familia se convierte en la más
importante referencia para el individuo pues lo estructura y condiciona. Por
ello identificar el papel de la familia en la socialización diferencial de
género no resulta muy difícil, pues es esta la que impone una división, a veces
sin siquiera haber nacido el individuo. Por otra parte la Iglesia, según
\textcite{Bourdieu2000}, inculca  una visión explícitamente pro-familiar
enteramente dominada por los valores patriarcales. Finalmente la escuela,
independientemente de si se encuentre o no bajo la influencia directa de la
iglesia, sigue transmitiendo una construcción patriarcal de la sociedad, pues
permite la reproducción de la división en base a características biológicas e
impone sobre los estudiantes una construcción de género en base a la misma.

Autores como \textcite{Mansilla1996} reafirman lo planteado anteriormente al
señalar que el desarrollo psicosocial del niño o niña viene dado inicialmente
por su familia, que se convierte en su grupo de referencia afectiva, así como
con el mundo que lo rodea, para de esta manera formarse y hacerse un ser social.
Además de esto la autora plantea que existen modelos sociales, normas o reglas
que se usan para enseñar a los individuos formas aceptadas de interactuar,
pensar y ser. Estas son implementadas en los individuos desde su nacimiento según
su sexo biológico, es así como a un individuo que nació con genitales masculinos
se le enseña a obrar como debería actuar alguien que pertenece al grupo
masculino. La autora plantea que es de esta manera en la que se generan
estereotipos sexuales pues cosas como que una mujer pueda ser igual o más
fuerte que un hombre entra en conflicto con la noción de que eso no sería algo
femenino.

Según \textcite{Anton2001} se podría plantear, además de lo anterior, un nuevo
factor en la socialización de los niños y niñas y la forma en la que construyen
su identidad de género. Este nuevo factor es la televisan y, más allá de ella,
los medios. Según este autor se muestran en series televisivas infantiles
identidades patriarcalmente hegemónicas sobre lo que debería hacer un hombre y
una mujer, expresan a la masculinidad unida al recurso de la violencia y el
riesgo, mientras que a la feminidad unida a la debilidad, bondad y pasividad.

\todo[inline]{Verificar el siguiente párrafo con Emerson}Entonces parece ser
sensato afirmar que la construcción del género y la forma en la que socializan
el género los individuos de una sociedad puede venir determinada por el sexo con
el que nacen. Esto puede generar interacciones que pueden no ser las más
correctas o adecuadas pues están ligadas a conceptos hegemónicos que muy
posiblemente no se adapten a la realidad de las relaciones sociales y que
permite únicamente una mayor separación entre hombres y mujeres y puede causar
un mayor conflicto en alguien que no se sienta identificado con el género que le
asignaron por su sexo biológico.

\section{Transgénero o Transexualidad a lo largo de la historia}

La transexualidad está definida como una identidad desde la cual una persona
siente que pertenece a un género distinto al que socialmente se impone por su
sexo biológico \parencite{Helfrich2001}. Como resultado, la persona se encuentra
disconforme con su cuerpo y desea cambiarlo para que corresponda con su
identidad. Esta es además definida por la Organización Mundial de la Salud (OMS)
como una enfermedad mental por disforia de género \parencite{BergeroMiguel2008}.
La misma se caracteriza por el profundo sufrimiento psicológico, rechazo y
disconformidad que siente la persona por su propio cuerpo \parencite{APA}.

El género ha sido tradicionalmente asociado con la anatomía genital. Y por tanto
fuertemente relacionado al sexo biológico y a las ciencias que estudian estos
aspectos de la sexualidad. Por lo tanto la medicina y la biología han sido
asignados el rol de evaluar las manifestaciones de los roles de género. No sin
trabas pues el género, como se define actualmente, trasciende una colección
particular de características fisiológicas \parencite{Butler2001}.

El principal discurso acerca del género que se constituye desde las ciencias
sociales está dominado por la teoría feminista y los estudios de género. Luego
del surgimiento de los movimientos de derechos de las mujeres, el feminismo
surge como postura ideológica que propone la igualdad de género como meta
\parencite{DeMiguel,Gamb2008,Guardia2013}. Psicólogos, sociólogos, filósofos,
biólogos y la academia en general se volcó, a partir de ese momento, a los
estudios de la mujer. Se dan un proceso efectivo de profesionalización de los
movimientos sociales \parencite{Helfrich2001}.

Así pues, el feminismo construye politicamente una nueva forma de concepción del
rol de género. Esto no llega automáticamente ni carece tampoco de conflictos y
confrontaciones. Pero comienza una visión del sexo y la sexualidad que interpela
y cuestiona la construcción tradicional.

La temática de conflicto interno más fuerte es quizás la definición del
trastorno mental \parencite{APA}. Originalmente el discurso académico definía la
desviación de las normas de género como una enfermedad psiquiátrica.
Orientaciones sexuales distintas a la heterosexualidad eran concebidas como
aberraciones. De hecho, el concepto mismo de homosexualidad se origina en su
definición como una afección clínica.

Existe detrás de esto una visión ontológica del ser humano principalmente
materialista, que dispone al cuerpo y su funcionamiento como los determinantes
de lo que debe ser. Así, es inconcebible la alteración o modificación del sexo
pues se presupone que este está fijado en el cuerpo y manifestado a través de su
expresión anatómica.

Hace falta el surgimiento del género como un rol socialmente construido y aparte
de la conformación del cuerpo para comenzar a considerar formas alternativas de
su expresión como no patológicas. De la mano de los movimientos por la no
discriminación de homosexuales y bisexuales se da una alteración de algunas de
las posturas académicas. La noción de que la sexualidad puede ser una expresión
independiente de la función reproductiva y de la posesión de un sexo lleva a
desligar ligeramente a los roles de género de esta fijación corporal.

Sin embargo, la transexualidad y el ser transgénero siguen siendo considerados
como enfermedades mentales. La Organización Mundial de la Salud, a pesar de
encontrarse desde hace ya muchos años en un proceso de reforma de sus
diagnósticos psiquiátricos, continúa sugiriendo el diagnóstico de la
transexualidad como un trastorno. A este se le considera el discurso bio-médico
de la transexualidad \parencite{Helfrich2001}.

\subsection{Transgénero o Transexualidad en Venezuela}

La transexualidad, como parece estar entendida en el imaginario colectivo del
venezolano, se asocia más comúnmente con las redes de prostitución de trans
femeninos. Se trata de hombres que, vestidos como mujer y con grados variables
de transición, ofrecen servicios sexuales. Estas personas son concebidas de
forma caricaturesca, pero constituyen una de las poblaciones trans más
vulnerables. Esto debido a que se deben exponer abiertamente para ejercer la
prostitución, por lo que son generalmente el blanco de las formas más frecuentes
y gráficas de discriminación. Los testimonios recogidos por \textcite{Lugo2016}
muestran esta realidad. “Casi todos los días recibo amenazas de hombres y de las
actuales madrinas” (¶. 25) reporta una de las entrevistadas.

Esta cualidad caricaturesca y de perversidad extrema que se les atribuye en el
imaginario social también les ha hecho una población de estudio más común en las
investigaciones, generalmente, desde una perspectiva sanitaria-epidemiológica
\parencite{ONUSida2012}.

Tiene, esta particular expresión de la sexualidad trans, muchas formas de
origen. Reproducción de ciclos de violencia y abuso, rechazo familiar, condición
de clase, vulnerabilidad financiera, entre otros factores que se mezclan para
generar este fenómeno.

En Venezuela, la Misión Negra Hipólita, cuyo objetivo es la erradicación de la
situación de calle mediante la inclusión de los ciudadanos que se encuentran en
esta situación, dedicó exclusivamente durante un tiempo uno de sus albergues a
la población de mujeres trans sin hogar. Este albergue, llamado Centro de
inclusión social «Belinda Alvarez» funcionó durante algún tiempo
\parencite{ASGDR2014-05}. Sin embargo la situación país, crisis económica y la
falta de voluntad política, terminó ocasionando el abandono por completo de esta
iniciativa de asistencia social \parencite{ACDV,Gay2014-04-16}.

Estas perspectivas llevan a inconvenientes a la hora de reconciliar nuevas
concepciones del género con formas tradicionales de atención médica y otros
aspectos relacionados. Por ejemplo, el papel de las empresas aseguradoras en el
mantenimiento del bienestar del individuo ante situaciones inesperadas. La
población trans se concibe en este sentido como vulnerable puesto que algunas
agencias internacionales, como la Organización Mundial de la Salud, aún definen
la transexualidad como un trastorno mental.

Esto tiene consecuencias directas en las personas trans. Puede impedir su
afiliación a programas de seguro médico, limitar su acceso a la atención médica.
Y además, en un aspecto pocas veces considerado, pone una limitante en las
habilidades de los profesionales de la salud para atender positivamente a esta
población.

\subsection{Perspectiva médica}

En una situación relacionada, los servicios de atención psicológica y de salud
sexual y reproductiva se encuentran generalmente sin preparación para atender a
la población trans. O simplemente no existen servicios de atención
especializados. Por ello se trata de una población vulnerable tanto al abuso,
discriminación y abandono, magnificado por su dificultad para el acceso a
atención médica sin discriminación.

Existen, no obstante, otro aspecto de la atención médica de gran importancia
para el individuo trans. La transición de un género a otro requiere, en
condiciones ideales, de un tratamiento endocrinológico e intervenciones
quirúrgicas accesibles, usualmente de manera exclusiva, con un profesional de la
salud. Las hormonas y drogas de tratamiento químico para la transición son
accesibles sólo con orden médica por ser sustancias controladas. Y los
procedimientos quirúrgicos requieren ser ejecutados por cirujanos experimentados
y altamente entrenados.

Esto no ha limitado, sin embargo, el surgimiento de redes de tráfico de drogas
de transición que florecen mediante la explotación de la población trans que por
discriminación o limitada capacidad adquisitiva deciden acudir al mercado negro.
	
No todos los endocrinólogos comparten una visión positiva acerca de la población
trans. E incluso, entre aquellos que apoyan las decisiones de las personas trans
que deciden hacer la transición, se demandan altos estándares de evaluación
previa como: evaluaciones psiquiátricas, costosos exámenes hormonales y largos
periodos de observación previa. Todo esto redunda en altísimos costos en
facturas médicas. Complicado además por el hecho de que, al menos en Venezuela,
hay muy pocos endocrinólogos y psiquiatras dispuestos a realizar estas
evaluaciones sin discriminación. Tan pocos como menos de diez en cada profesión.

Como resultado de esta circunstancia, muchas personas trans deben enfrentar la
decisión de postergar o no realizar la transición; ó acudir a fuentes menos
confiables realizando tratamientos autoadministrados o copiando el tratamiento
de otras personas trans conocidas.

Existen, incluso, mafias de médicos no calificados que realizan intervenciones
quirúrgicas de cambio de sexo de manera clandestina.

\subsection{Diferencia entre transgénero y transexual}

La expresión de la transexualidad tiene algunas sutilezas. Estas dan visos de
como se relaciona con otros factores psicosociales.

El primero que puedo nombrar es la diferenciación entre transgénero y
transexual.  La referencia al primero es para aquellas personas que sienten que
su cuerpo no corresponde con su identidad. En este sentido se plantean que su
género es distinto, sin embargo, aún no han asumido o iniciado una transición
hacia el sexo con el cual se sienten identificados.

La experiencia típica, en mi entendimiento, es, en primer lugar, una
presentación únicamente estética de acuerdo al sexo deseado. Esto es, utilizar
vestimenta y presentarse conductualmente como el sexo deseado.

Hay una diferencia entre aquellas personas nacidas con genitales masculinos que
se identifican como mujeres (MTF) y quienes nacen con genitales femeninos y se
identifican como hombres (FTM). Estos son identificados usualmente con las
iniciales correspondientes a las expresiones en inglés: Male to female y Female
to male.

Cuando una persona trans decide cambiar de sexo se requiere de un tratamiento de
por vida con terapia de sustitución hormonal y cirugía de reasignación de sexo.

Sin embargo, entre la expresión de vestimenta y meramente gestual, y las
distintas etapas de tratamiento se puede hablar plenamente de transexualidad.
Indiferentemente del progreso o nivel de transición de sexo que tenga la
persona. Es una referencia directa a que la persona se encuentra haciendo (o
desea iniciar) la transición de un sexo a otro.

El punto anterior es de vital importancia debido a que la transexualidad
requiere una forma particular de materialidad corporal. La transformación del
cuerpo no es algo extraño ni poco común para el establecimiento de la posición
de la persona en el mundo social.

\subsection{Transición}

Continuamente utilizamos el cuerpo, la expresión corporal y el aspecto físico
para imponer, posicionar o expresar cosas acerca de nuestra identidad, roles,
clase, prestigio e intencionalidad.

La modificación corporal ha tenido tradicionalmente un fuerte impacto simbólico
para la expresión de la identidad en este sentido. Pero la transexualidad
trasciende estas expresiones. En este caso es el deseo mismo de tener otro
cuerpo. O, en palabras de las mismas personas trans, haber «nacido en el cuerpo
equivocado».

De esta manera el deseo de modificar el cuerpo no es una acto expresivo
únicamente, sino que es un acto correctivo. Es la alineación del cuerpo al
sentir de la identidad del yo.

Esta expresión, esta posibilidad de transición está marcada transversalmente por
la existencia de la misma como posibilidad material y simbólica. Para muchas
personas trans, antes de conocer de la existencia de esta posibilidad
identitaria, lo primero es una puesta en cuestionamiento de la orientación
sexual.

Si se carece del acceso cultural a las ideas y nociones de género que permiten
una fluidez en la expresión o un cambio por completo, se hace difícil engendrar
la identidad trans. Los mismos testimonios dan fe de una concepción de las
etapas tempranas de la constitución de la identidad trans como una confusión.

Esto significa que la posibilidad real de acceder a cierto campo cultural altera
las formas en las que la identidad trans se expresa. No es equivalente la
formación cultural del hombre que quiere ser mujer nacido en el interior del
país en un pueblo pobre al que puede disponer el hijo de Karina de quien hablaba
en un segmento anterior.

Esta diferencia altera la capacidad de concepción de una trayectoria de vida
asociada con la transición de sexo. Pero esta posibilidad ha ingresado en el
imaginario colectivo de una forma que puede ser fácilmente reconocido. La
problemática de orden superior que se le suma es la posibilidad de acceso
material al proceso de transición.

El poder adquisitivo y la auto-confianza para ubicar un médico de confianza se
deben confrontar a la transfobia y la ignorancia alrededor de este proceso.
Complejizado por el rechazo y dificultad que tienen las personas trans para
realizar una inserción laboral exitosa los deja en posiciones vulnerables y
expuestos a daños físicos o la muerte cuando, sin orientación y sin recursos,
intentan realizar la transición por su cuenta.

Inyecciones de químicos dañinos, daño a nervios, operaciones con médicos no
calificado y la automedicación de terapias sustitutivas de hormonas son algunos
de los riesgos a los que se expone la persona trans que, al carecer de la
información y de los  medios económicos para una transición segura, intenta
desesperadamente hacer coincidir  su cuerpo con su identidad.

%Es por esto pues, que para poder tener un entendimiento más efectivo de lo que
%significa la transexualidad y el transgenerismo (que se agrupan bajo el prefijo
%trans) y los factores asociados a los mismos, la presente investigación plantea
%explorarlos a continuación.

\section{Teoría Queer y performatividad de género}

Parece importante para poder adentrarse en el tema trans tomar en cuenta elementos teóricos que permitirían la apreciación de este fenómeno. Inicialmente se podría hablar de la teoría queer como aquella que afirma que la orientación sexual y la identidad sexual (o de género) son el resultado de una construcción social y consecuentemente esto significa que no existen determinantes estrictos que asignen roles o papeles, sino formas variables de desempeñar papeles sexuales en la sociedad (Fonseca y Quintero, 2009)\todo{no se de donde es esta cita, help}.

Por su parte autores como \textcite{Solana2013} plantea que desde la teoría
queer se puede reivindicar las identidades que bajo otros lentes podrían ser
consideradas incompletas, sean las de los sujetos trans o de homosexuales que
performativamente rompan la norma. Esta autora expresa que la teoría queer
aporta herramientas necesarias para poder comprender la autenticidad de las
practicas de género. Otro aspecto importante que rescata esta autora es que
desde la teoría queer se puede reivindicar la pluralidad taxonómica con respecto
a las categorías de género.

Se puede entonces hablar que desde la teoría queer la transexualidad podría
verse como una expresión de identidad sexual enmarcada en aspectos que asumen a
la misma como un continuo, no un hecho binario, lo cual permitiría repensar la
idea de que solo existen hombres y mujeres, yendo más allá de esto, entendiendo
que existen grados de masculinidad y feminidad y que cada individuo expresa y
vive los mismos según mejor le parezca.



\section{Antecedentes de la investigación}

La mayoría de los expertos consideran que el hecho de ser transgénero no está
causado por un solo factor. Creen que es la consecuencia de una compleja mezcla
de factores biológicos, psicológicos y ambientales, no solo una cuestión de
gustos. Actualmente se encuentran pocas investigaciones con una mirada
Psico-Social en esta área, lo cual motiva la elaboración de la presente
investigación.

\subsection{Perspectiva Biologicista}

Ha existido una visión dominada por el biologicismo con respecto al tema de la
identidad trans. Autores como \textcite{Lugo2016} plantean que se puede identificar
sexualmente a los individuos según su sexo cromosómico, gonadal, hormonal,
embrionario, fenotípico genital y sexo psicológico. Con esta clasificación se
puede hacer evidente que el papel que existe en la construcción sexual con base
a bases biológicas, pues desde esta perspectiva existe una mayor cantidad de
determinantes biológicos que permitirían determinar el sexo y el género de un
individuo que psicológicos o sociales.

Esto podría parecer reduccionista, pues limita la identidad, rol y demás hechos
asociados al género a una expresión biológica que puede no estar en concordancia
con la identidad del individuo.

\subsection{Perspectiva social}

\textcite{Martinez-Guzman2012} plantea que la perspectiva psicosocial aplicada a
la construcción de identidades trans permitiría romper o resignificar dualidades
como sexo-género y hombre-mujer,  llevando a un plano más tangible y real lo que
es la expresión y vivencia del género, pues la construcción que se tiene en la
actualidad está marcada por tintes biologicistas que pueden llegar a ser
reduccionistas ante la complejidad que significa ser humano y vivir experiencias
que no son siempre comunes para todos.

\subsection{Perspectiva psicológica}

El principal enfoque que tiene la psicología al acercarse a la condición Trans
es desde una mirada clínica, principalmente enfocada en el uso de manuales como
el DSM y el CIE (en sus versiones más actuales).

El Dr. Bockting (\cite*{Alonso2015} \citeauthor[c.p.][]{Alonso2015}) hace
referencia a que las disparidades de salud en el área de salud mental han sido
bien documentadas en esta población. Las personas transgénero son más
vulnerables a tener síntomas de depresión y ansiedad, que es, al menos en parte,
atribuible al estrés social que experimentan como miembros de una población
minoritaria con respecto al género. Las personas transgénero pueden también
enfrentar desafíos relacionados a la necesidad de afirmar su identidad de género
y a los cambios sociales y físicos que esto pueda requerir. Este proceso
requiere coraje, y entendimiento de uno mismo y de los seres amados y con
frecuencia también incluye una serie de pasos concretos y cambios para los
cuales el apoyo es todavía muy limitado.

Los psicólogos pueden jugar un rol invaluable. Esto incluye asistir a las
personas transgénero al explorar y afirmar su identidad de género, sea esto a
nivel individual, interpersonal o social/comunitario. El psicólogo también puede
asistir al individuo transgénero en la recuperación del impacto negativo del
estigma social en su salud y bienestar y facilitar experiencias positivas al
“salir del closet” con la familia, amigos y comunidad. Una y otra vez, las
personas transgénero han mostrado mucha resiliencia al afrontar los desafíos
psicosociales relacionados al estigma que enfrentan. Los proveedores de salud
mental comprensivos y compasivos han sido con frecuencia una parte importante de
eso.

\subsection{Patologización}

La transexualidad validada desde la visión médica, ha traído como consecuencia
una patologización, con base a qué instrumentos diagnósticos como lo son el DSM-
IV y el CIE-10 considera a la misma un trastorno de identidad sexual, mientras
que el DSM- V \parencite{APA} lo considera como una disforia de género en
adolescentes y adultos con postrasnsición, es decir, aquellas personas que
poseen una marcada incongruencia entre el sexo que siente o expresa y el que se
le asigna, y que éste ha realizado la transición a una vida de tiempo completo
con el sexo deseado (con o sin legalización e intervenciones quirúrgicas del
cambio de sexo). A pesar de esto, la OMS ha determinado que en la edición a
publicarse en el año 2018 del CIE-11 deje de tratarse como trastorno y
presentarse como incongruencia de género \parencite{Borraz2017}.

Sumado a esto, otros autores como \todo{No se encuentra esta cita}Hernández,
Rodríguez y Garcia-Valdecasas (2010) han expresado que esta clasificación puede
estar sesgada pues se encuentra ligada a factores exclusivamente culturales y
que se encuentra enmarcada en una ideología binaria en la definición de lo que
significa la identidad de género. \textcite{Vargas2017} explica
que se debería considerar a la identidad sexual como un continuo, no como dos
categorías estrictamente definidas con base a características biológicas,
principalmente porque pese a que existen las mismas y pueden ser comunes entre
los individuos no necesariamente todas las compartes y las mismas pueden variar
en mayor o menor medida dependiendo del sujeto.

\section{Constitución de la Subjetividad}

Un factor que parece importante dentro de lo que determina no sólo la identidad
de un individuo sino también la forma en la que este se relaciona con otros es
la subjetividad. \todo{Reescribir esta oración}Esta se podría asumir
esencialmente como los procesos que significan estructuras y procesos esenciales
que un sujeto conoce y construye \parencite{Duesing2002}.  Autores como
\todo{Verificar si esta es la cita que corresponde} \textcite{Gonzalez2013}
rescatan que la subjetividad y la psicología poseen una estrecha relación
sobretodo en el campo social, este autor expresa que la subjetividad social es
la forma en la que se integran sentidos subjetivos a diferentes configuraciones
subjetivas de espacios sociales. Es decir, el individuo interactúa con un
espacio así como con otros individuos en base a una construcción previa,
generada en base a factores como la socialización primaria y secundaria que
fueron mencionados anteriormente.

\subsection{Identidad}

\todo{Revisar redacción y relevancia de este párrafo}
Rescatando lo planteado por \textcite{Berger1991} se puede afirmar que en la
socialización se encuentra la base de la construcción de la identidad del
individuo. Como ya había sido planteado anteriormente la socialización primaria
permite al individuo una construcción inicial de su identidad en base a las
relaciones y vínculos que establece en su niñez, estos vínculos terminan por
moldear y estructurarlo. Es decir, crea una conciencia de los roles y actitudes
hacia específicos roles en la interacción social. Por otra parte la
Socialización secundaria está determinada por la primaria en el sentido de que
el acercamiento a otros actores a lo largo de la vida del sujeto viene
condicionada por la primera estructuración. Es por esto que se puede afirmar que
el sujeto se ve determinado en gran medida por aquellas interacciones que tiene
a temprana edad.

\subsection{La mirada del otro}

En la actualidad se puede apreciar cómo la corporalidad tomó gran valor,  la
mirada del otro adquiere preponderancia, y surgió la promoción del cuerpo,
pasando a ser éste el primer plano de las preocupaciones; lo que trajo como
consecuencia que la representación del cuerpo hecha por las personas pasara a
vincularse con los conceptos de posesión y pertenencia. Uno de los elementos que
influyó significativamente en esta sobrevaloración del cuerpo es la búsqueda de
la individualidad, otorgándole así el valor de fundamentación de la diversidad
\parencite{LeBreton1994}. Esto quiere decir que más allá de la necesidad básica
humana de ser mirado por el otro, existe en la actualidad  la necesidad de ser
reconocido pero como un individuo único.

\subsection{Autopercepción}

Según \textcite{Exner1998}, la autopercepción es un conjunto de aspectos
descriptivos y valorativos que la persona ha ido elaborando para lograr un
autoconocimiento y una auto-valoración, ajustada a la realidad, acerca de sí
mismo. Entre los elementos que conforman la autopercepción los autores
mencionan: la autoimagen, la autoestima, estimación de la valía personal, el
autocentramiento, autoconcepto, identidad, entre otros.

\textcite{Exner2000}, expone que el término alude a dos conceptos, la propia
imagen y la relación con uno mismo. El primero (la autoimagen) se construye a
partir de las impresiones que cada individuo posee de sus características,
presumiendo que parten de experiencias basadas en la realidad. Parte de estas
impresiones se mantienen en un nivel consciente mientras que otras están de
forma parcial o en su totalidad inaccesibles a la consciencia, quizás por su
naturaleza disonante, estas son reprimidas.

Por otro lado, la segunda, emana de la autoimagen y hace referencia al nivel en
que una persona tiene su atención orientada hacia los aspectos propios, en
contraposición a estar en contacto con el mundo exterior. Esta relación consigo
mismo puede configurarse en un marco tanto positivo como negativo, y sea de una
u otra manera termina siempre influyendo en los objetivos que cada persona se
propone.

\subsection{Imagen corporal}

Según los autores \textcite{Kottow2005} entendemos por cuerpo un complejo
constituido por el cuerpo propiamente dicho y la mente. El cuerpo es psique y
soma, es una unidad ordenada y se encuentra integrada por sistemas y estos por
diversos órganos, interconectados e interactivos en su funcionamiento.

Por su parte, \textcite{Schilder1999} emplea la expresión imagen corporal para
designar “una representación a la vez consciente e inconsciente de la posición
del cuerpo en el espacio, encarado en sus tres aspectos de sostén fisiológico,
estructura libidinal y significación social”. Asimismo, la señala como la
representación mental que cada individuo genera de su cuerpo, promoviendo así la
conformación de su identidad.

Tal como señala \textcite{Schilder1999}, es lo que uno imagina y percibe
respecto  su propio cuerpo, asimismo involucra al nivel de satisfacción que
tiene la persona sobre su cuerpo.

\todo{Revisar esta cita de rosenboum}
\textcite{Rosenbaum1979}, quien interpreta a la imagen corporal como la
sensación de corporeidad que va desde la infancia hasta el final de la vida,
sugiere también que no es una concepción estática sino que por el contrario se
modifica a lo largo de la vida, por factores evolutivos o por enfermedad, placer
y atención por parte de los otros.

\subsection{Afrontamiento}

Portales como el de la \todo{Agregar esto como una referencia} fundacion Daniela
(www.fundaciondaniela.org), hacen referencia a lo engorroso que puede ser el
proceso de la resignificación de su identidad. Principalmente expresan que gran
parte de los sujetos trans se ven expuestos a malos tratos a nivel familiar,
profesional y entre amigos. Esta institución remarca la importancia del
acompañamiento que se le debe hacer a las personas trans.

\textcite{Salin-Pascual2008} expresa además de que entre los conflictos que más
frecuentes que se encuentran en personas trans son el de temor, principalmente
existente por el hecho de transgredir la norma social y ser identificado como
transgresor;  frustración por sentir que se vive una situación injusta al no
tener una relación normativa entre su sexo y su género. Este autor expresa que
dependiendo de las herramientas con las que cuente la persona trans podrá en
mayor o menor medida sobrellevar estos conflictos.

\subsection{Aceptación}

Este mismo autor, inspirado en las fases creadas por Elizabeth Kubler-Ross's en
su libro “Etapas de la muerte y de morir”, constituyó siete fases que suelen
recorrer las personas para lograr la aceptación de una persona Trans, donde se
parte de la fase uno, que hace referencia a Cuando la persona se da cuenta y se
sorprende; la segunda fase donde aparece la Negación y Culpa; la tercera fase
donde aflora el Miedo y Culpabilidad, seguido de la fase 4 donde la persona está
Aprendiendo a Entender, para luego pasar a la fase cinco donde se encuentra
Buscando Aceptación, después la fase seis donde la persona se encuentra Saliendo
del closet para finalmente llegar a la fase siete que va Más allá de la
Aceptación hasta el Orgulloso Activismo.