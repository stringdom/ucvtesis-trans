\chapter{Conclusiones}\label{ch:conclusion}
% # Conclusiones
%
% “El objetivo de una persona trans es llegar a ser cisgénero.”
%
% “La verdad es que nunca voy a llegar a ser cisgénero, pero puedo intentar acercarme lo más que pueda.”
%
% “Los importante es lo que falta, no sobra lo que tengo, me falta lo que quiero.”
%
% * La transición es un continuo.
%
% * El propósito de la persona trans es poder ser cisgénero. Hay distintos niveles de alcance a ese objetivo.
%
% * La discriminación obstruye el desarrollo normal de la identidad. Experimentar discriminación y las dificultades que afronta la persona trans socialmente le diferencia del desarrollo identitario de las personas cissexuales.
%
% * Hay mayores dificultades para las mujeres trans que para los hombres trans debido a las diferencias de género y la influencia del patriarcado y el machismo en la sociedad.
%
% * La disforia inicia en etapas muy tempranas de la vida.
%
% * Hay 2 concepciones de la identidad: la identidad trans “Soy un hombre trans”, y la identidad cis “Soy un hombre”.
%

Partiendo desde del proceso de entrevistas, pasando por lo recopilado dentro del
proceso de codificación y categorización así como la discusión de estos se
hicieron  presente una serie de elementos que permiten dar pie a la elaboración
de las conclusiones para la presente investigación. Enumeraremos a continuación
y elaboraremos en el mayor detalle posible.

Un primer elemento que surge de la investigación es la propuesta de que la
transición comienza desde tempranas etapas del desarrollo del individuo y que se
mantienen como un continuo. En las entrevistas realizadas y los verbatims
analizados se pudo encontrar que la identidad transgénero se hacía presente
desde etapas tempranas de la vida de los individuos, cuando comentaban el
malestar que les causaba tener que usar ropa que sentían no resonaba con su
identidad de género, al ser moldeados bajo una mirada patriarcal que limitaba su
expresión así como sus interacciones.

Se plantea que la transición es un continuo porque la persona transgénero se
encuentra constantemente en un proceso de construcción identitaria. El tener que
buscar una validación por parte de otros miembros de la sociedad lleva al
individuo a enfrentarse a situaciones que los cuestiona y los motiva a
solucionar la discrepancia en su identidad de género.

Este proceso parece estar enfocado en la lucha del individuo transgénero por
llegar a ser cisgenero, un proceso largo y que no puede llegar a completarse a
cabalidad pues bajo los estándares médicos y científicos una persona transgénero
siempre va a permanecer como tal a pesar de que su identidad sea otra. Sin
embargo, se podría plantear que este objetivo, el ser cisgenero, se puede
alcanzar en distintos niveles, bien sea recurriendo a elementos que les permita
mostrarse de acuerdo al género con el que se identifican (como pueden ser el uso
de maquillaje, ropa especifica o incluso cirugías estéticas). Desde esta
perspectiva también parece ser posible afirmar que la condición transgénero se
puede enfocar mas en las faltas, que en aquello que sobra del individuo, esto
desde la propuesta de que se puede recurrir a realizarse operaciones de mamas en
el caso de una persona transgénero MaF o el uso de un dispositivo para orinar de
pie en el caso de una persona transgénero FaM. Dentro de la investigación los
participantes hablan más de la falta de caracteres sexuales secundarios que de
aquellos caracteres que pueden estar de mas por su identidad transgénero.

Este elemento resuena especialmente con que dentro de la investigación los
participantes parecen encontrar validez como hombre o mujer según la aceptación
que existe por otros miembros de la sociedad de su expresión de género.
Consecuentemente esta validación por el otro, lleva a la constitución
identitaria del individuo pues en mayor o menor medida lleva a solucionar
aquella incongruencia y permite la alineación del individuo con el género con el
que se identifica. Sin embargo partiendo de esto resulta resaltante el hecho de
que la rigidez de la dicotomía hombre-mujer impone modelos y estereotipos de lo
correcto o incorrecto según el sexo biológico, sin tener en cuenta que el género
es mucho más amplio y complejo, permitiendo entonces que todo lo que se salga de
los parámetros preestablecidos sea juzgado y discriminado.

Partiendo de esto se puede traer a colación otro elemento que surgió dentro de
la presente investigación y es que la vivencia de la identidad de las personas
transgénero se ve innegablemente influenciada por la visión patriarcal que
existe sobre lo que es un hombre y una mujer así como que elementos los definen.
Parece que para que una persona transgénero sea validada por otros tiene que
llegar a ser mas del género con el que se identifica que lo que podría llegar a
ser una persona cisgenero. Cabe plantear que esta visión hegemónica de la
identidad de género se va construyendo a lo largo de la vida del individuo,
basado en interacciones sociales con otros miembros de la sociedad.

Partiendo de este elemento se puede conectar con otro que también resalta dentro
de lo encontrado y es que parece que la discriminacion puede actuar en contra
del desarrollo identitaria de los individuos transgéneros. El experimentar actos
discriminantes así como las dificultades que puede llegar a afrontar una persona
transgénero (que pueden ir desde actos violentos, a violencia simbólica o
académica o incluso el atentar con la vida propia) es un elemento que diferencia
al individuo transgénero de aquellos que son cisgenero así como bisexuales. Es
necesario recordar que no toda masculinidad o feminidad se construye de igual
manera pero resulta resaltante que la incongruencia de género puede llegar a
generar un malestar que las personas cisgenero y bisexuales no van a
experimentar.

En algunos casos en los que un individuo no pueda asumir su identidad
transgénero eesta inhabilidad puede estar ligada a situaciones de interacción
con otros miembros de la sociedad e incluso falta de información. Se pudo
evidenciar que la identidad transgénero puede llegar a ser desconocida y por
esto resulta importante mejorar su visibilización así como trabajar procesos de
simbolización para que entrar en contacto con esta identidad no sea un hecho que
cause problemas.

Se evidenció a lo largo de las entrevistas cómo los participantes presentan
desconfianza ante el entorno, producto de las experiencias sufridas de
discriminación y rechazo a lo largo de su transición, estas vivencias
discriminativas llevaron a los individuos a generar una serie de mecanismos de
defensa que en definitiva guían sus interacciones sociales, pero que cabe
cuestionarse que peso pueden llegar a tener sobre la cotidianidad del individuo
así como en la forma que modelan sus interacción con otros miembros de la
sociedad.

Finalmente se puede señalar que los objetivos de la presente investigación
fueron cumplidos en su mayoría,. Las entrevistas permitieron una aproximación al
tema de estudio así como también aportaron conocimientos en un área muy
importante pero poco estudiada en Venezuela, sin embargo resultaría beneficioso
que se tomara como base esta investigación y se abordara desde otras
perspectivas para abarcar los aspectos cognitivos y dinámicos que surgieron en
este trabajo.

\section{Limitaciones y recomendaciones}

Resulta necesario expresar que pese a que se buscó un acercamiento completo
sobre como es la vivencia de la transición en personas transgénero se
presentaron condiciones que aunque no impidieron el normal desarrollo de la
investigación que si llegaron a limitar el alcance de la misma.

Una de estas limitaciones se presentó al momento de acceder a la muestra. Se
presentó resistencia por parte de candidatos a participantes al momento de
coordinar entrevistas, en algunos casos los candidatos dejaron de responder a
los intentos de contacto que eran realizados por los investigadores. Sumado a
esto se presenta una situación contingente basada en la escasez de medicamentos
para seguir el tratamiento hormonal que normalmente deben llevar las personas
transgénero, esta condición ha llevado a buena parte de la población transgénero
migrar a otros países para poder continuar con su tratamiento.

Sumado a esto existe un reducido número de estudios académicos desde una mirada
psicológica sobre personas Trans (tanto transgénero como transexuales) en
Venezuela y este es un factor que puede llegar a ser un limitante al momento de
plantear la investigación. Sin embargo, es necesario resaltar que en la presente
investigación se tomó en cuenta investigaciones previas realizadas dentro de
la mención de psicología clínica de la Universidad Central de Venezuela.

Tomando en cuenta estas limitantes así como también lo construido en base a la
información recopilada por la presente investigación se plantean las siguientes
recomendaciones para futuras investigaciones

\begin{itemize}
\item Se sugiere continuar indagando en este tema de
investigación dentro de la perspectiva de los estudios de género. Esta es un
área que requiere mayor profundización. Debido a múltiples razones, este tipo de
muestra estudiada pertenece a un población invisibilizada y discriminada, no
sólo en lo social, legal, teórico sino también en el plano de la investigación
psicológica.

\item Debido a que la concepción de género atravesó constantemente el discurso
sobre la corporalidad de estos participantes, es necesario ahondar en el género
como elemento regulador del cuerpo. Así como en el rol del cuerpo en la
construcción de la identidad del género.

\item Se recomienda emplear otras estrategias de recolección de información,
como por ejemplo grupos focales que permiten ver las interacciones entre
distintos puntos de vista de actores diversos sobre un mismo tema en común.

\item Se recomienda investigar desde la mirada psicosocial la percepción y los
imaginarios simbólicos que existen en la población en general de la sociedad
venezolana sobre las personas transgénero y transexuales. También explorar el
ámbito familiar, escolar y laboral alrededor de las personas transgénero y
transexuales.

\item Se recomienda sistematizar las estrategias de conocimiento y acceso de los
distintos métodos de transición que son accesibles o están disponibles en
Venezuela. Así también como las limitaciones y dificultades que son resultado de
la actual situación politica y económica venezolana.

\item Incentivar la realización de talleres y capacitaciones sobre identidad de
género y también sobre los derechos humanos, para así evitar la discriminación y
el acoso. Esto crea un conflicto tanto interno como externo en la persona que
está recibiendo este maltrato emocional ya sea en su lugar de trabajo, en
el hogar, lugares público, entre otros.

\item Se recomienda a los participantes realizar psicoterapia para adquirir
herramientas que les ayuden a afrontar de forma más efectiva el rechazo de la
sociedad y la discriminación a la que están sometidos con frecuencia. De igual
forma trabajar los niveles de ansiedad y depresión a los que tienen riesgo, así
como también para desarrollar habilidades sociales que les permita establecer
relaciones interpersonales profundas y significativas.
\end{itemize}
