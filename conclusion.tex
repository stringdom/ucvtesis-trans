\chapter{Conclusiones}\label{ch:conclusion}
% # Conclusiones
%
% “El objetivo de una persona trans es llegar a ser cisgénero.”
%
% “La verdad es que nunca voy a llegar a ser cisgénero, pero puedo intentar acercarme lo más que pueda.”
%
% “Los importante es lo que falta, no sobra lo que tengo, me falta lo que quiero.”
%
% * La transición es un continuo.
%
% * El propósito de la persona trans es poder ser cisgénero. Hay distintos niveles de alcance a ese objetivo.
%
% * La discriminación obstruye el desarrollo normal de la identidad. Experimentar discriminación y las dificultades que afronta la persona trans socialmente le diferencia del desarrollo identitario de las personas cissexuales.
%
% * Hay mayores dificultades para las mujeres trans que para los hombres trans debido a las diferencias de género y la influencia del patriarcado y el machismo en la sociedad.
%
% * La disforia inicia en etapas muy tempranas de la vida.
%
% * Hay 2 concepciones de la identidad: la identidad trans “Soy un hombre trans”, y la identidad cis “Soy un hombre”.
%


% FIXME Copiado sin modificaciones

Las conclusiones alcanzadas tras el análisis de la información suministrada por
la muestra a partir de las entrevistas, se orientan a dar cuenta del
cumplimiento del objetivo general y específicos de esta investigación. Por lo
que a partir de los resultados arrojados por los sujetos, se puede evidenciar:

La apariencia física determina en muchos casos lo que la sociedad percibe de los
individuos, y como esta retroalimentación con el otro es la que influye también
en el concepto que tiene cada uno de sí mismo e inclusive de cómo se ven a sí
mismos. Esto debido a que la rigidez de la dicotomía hombre-mujer impone modelos
y estereotipos de lo correcto o incorrecto según el sexo biológico, sin tener en
cuenta que el género es mucho más amplio y complejo, permitiendo entonces que
todo lo que se salga de los parámetros preestablecidos sea juzgado y
discriminado.

Se evidenció a lo largo de las entrevistas cómo los participantes presentan
desconfianza ante el entorno, producto de las experiencias sufridas de
discriminación y rechazo a lo largo de su transición de convertirse en hombres o
mujeres.

El proceso de identificación que marca la identidad de toda persona se produce
durante la infancia y la adolescencia. En el caso de estas personas al mismo
tiempo que adquieren características que se creen propias del género asumido,
deben renunciar a las que no se ajustan a su ideal de hombre o mujer con el que
se identifican para poder encajar y ser aceptados, evitando que el otro los vea
de forma ambigua.

Al existir estereotipos de lo que se considera femenino y masculino, se
evidencia una acentuación en estos rasgos de los cuales carecen biológicamente,
lo que se demuestra en el intento frecuente de verse femeninas con ropas,
maquillaje y operaciones estéticas o masculinos con ropas e instrumentos
improvisados para semejar un pene.

La percepción de estas personas se ratifica mediante el reconocimiento que el
otro (la familia y la sociedad) tiene de ellos como hombres o mujeres, y a su
vez, esta validación y aceptación por parte del ambiente está enormemente
influenciada por la imagen corporal que estas personas proyectan, lo que
significa que para ser reconocidos como hombre o mujer, no sólo deben sentirse
de este modo, sino que es necesario verse aparentemente de forma coherente con
este sentir para que el otro pueda validarlo como tal y esto le permita
aceptarlo.

Un aspecto resaltante, es que estas personas a pesar de tratar de verse en su
totalidad como hombre o mujer mantienen su característica biológica (pene o
vagina). Lo cual pudiese ser un intento de conservar el pene por la falta de
recursos económicos y/o temor a las posibles consecuencias negativas que una
operación tan invasiva pudiese causar (perdida de sensibilidad en los genitales)
o una posible fragmentación de la imagen corporal.

Finalmente se puede señalar que los objetivos fueron cumplidos, pero en la
búsqueda de respuestas nos encontramos con más interrogantes sobre éste
fascinante tema de investigación. Las entrevistas permitieron aproximarse al
tema estudiado y aportar algunos conocimientos en un área muy importante pero
poco estudiada en Venezuela, sin embargo resultaría beneficioso que se tomara
como base esta investigación y se abordara desde otras perspectivas para abarcar
los aspectos cognitivos y dinámicos que surgieron en este trabajo.


\section{Limitaciones y recomendaciones}

Como cierre de la investigación presentada anteriormente se desarrolla este
capítulo destinado a concretar  una serie de recomendaciones que surgen a partir
de la reflexión sobre la forma en que el estudio se llevó a cabo, donde los
investigadores deben evaluar el trabajo realizado considerando la base del
análisis anterior e indicar los puntos a mejorar para futuros trabajos de
investigación relacionados.

Ahora bien, la principal limitación que se presentó fue el acceso a la muestra
ya que debido al rechazo y discriminación que sufren constantemente, presentan
una resistencia muy fuerte y una actitud defensiva al momento de cooperar ante
una determinada investigación.

La falta de estudios académicos sobre personas Trans en Venezuela también fue
una gran limitación en el desarrollo de esta investigación, por lo que se
recurrido a las investigaciones realizadas, la mayor parte de las cuales fueron
apareciendo paralelamente a la elaboración de este trabajo de investigación.

Se considera pertinente realizar algunos ajustes en futuras investigaciones
tales como:

Se sugiere continuar indagando en este tema de investigación dentro de la
perspectiva de los estudios de género, ya que es un área que requiere mayor
profundización. Debido a múltiples razones, este tipo de muestra estudiada
pertenece a un población invisibilizada y discriminada, no sólo en lo social,
legal, teórico sino también en el plano de la investigación psicológica.

Debido a que la concepción de género atravesó constantemente el discurso sobre
la corporalidad de estos participantes, es necesario ahondar en el género como
elemento regulador del cuerpo.

Se recomienda emplear otras estrategias de recolección de información, como por
ejemplo grupos focales que permiten ver las interacciones entre distintos puntos
de vista de actores diversos sobre un mismo tema en común.

Incentivar la realización de talleres y capacitaciones sobre identidad de género
y también sobre los derechos humanos, para así evitar que se cometan delitos
psicológicos como la discriminación, la cual crea un conflicto tanto interno
como externo en la persona que está recibiendo este maltrato emocional
diariamente (en su lugar de trabajo, en el hogar, lugares público, entre otros).

Se recomienda a los participantes realizar psicoterapia para adquirir
herramientas que les ayuden a afrontar de forma más efectiva el rechazo de la
sociedad y la discriminación a la que están sometidos con frecuencia, de igual
forma trabajar los niveles de ansiedad y depresión que presentan, así como
también para desarrollar habilidades sociales que les permita establecer
relaciones interpersonales más profundas y significativas.
