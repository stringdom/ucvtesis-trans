\chapter{Conclusiones}\label{ch:conclusion}
% # Conclusiones
%
% “El objetivo de una persona trans es llegar a ser cisgénero.”
%
% “La verdad es que nunca voy a llegar a ser cisgénero, pero puedo intentar acercarme lo más que pueda.”
%
% “Los importante es lo que falta, no sobra lo que tengo, me falta lo que quiero.”
%
% * La transición es un continuo.
%
% * El propósito de la persona trans es poder ser cisgénero. Hay distintos niveles de alcance a ese objetivo.
%
% * La discriminación obstruye el desarrollo normal de la identidad. Experimentar discriminación y las dificultades que afronta la persona trans socialmente le diferencia del desarrollo identitario de las personas cissexuales.
%
% * Hay mayores dificultades para las mujeres trans que para los hombres trans debido a las diferencias de género y la influencia del patriarcado y el machismo en la sociedad.
%
% * La disforia inicia en etapas muy tempranas de la vida.
%
% * Hay 2 concepciones de la identidad: la identidad trans “Soy un hombre trans”, y la identidad cis “Soy un hombre”.
%

Partiendo desde del proceso de entrevistas, pasando por lo recopilado dentro del
proceso de codificación y categorización así como la discusión de los hallazgos,
podemos finalmente hacer presentación de las siguientes conclusiones para esta
investigación. Enumeramos a continuación estas conclusiones en resumen y
elaboraremos con mayor detalle posteriormente. Al final del capítulo presentamos
las limitaciones encontradas por la investigación y recomendaciones para
investigaciones futuras.

Las conclusiones son elaboradas en función de reportar lo más resaltante de los
contenidos elaborados en las entrevistas con los participantes y a la vez dar
respuesta a las preguntas de investigación expuestas en el
capítulo~\ref{ch:intro}, en la página~\pageref{preguntas}.

Las conclusiones encontradas se pueden resumir en los siguientes cinco (5)
puntos:

\begin{enumerate}

  \item La transición es un proceso continuo que se presenta como una gradiente.
  Esta inicia con elementos conductuales y llega hasta la modificación corporal
  quirúrgica.

  \item El principal deseo de la persona trans es llegar a ser cisgénero, o al
  menos lo más cerca que sea posible a esto dentro de las limitaciones y su
  acceso.

  \item La identidad de las personas transgénero entrevistadas es moldeada por los
  roles estereotípicos de género que prevalecen hegemónicamente en la sociedad
  venezolana.

  \item La discriminación es un obstáculo que afecta negativamente el desarrollo
  de las personas transgénero.

  \item Existen, entre las personas transgénero, dos concepciones distintas de
  la identidad de género. Una que concibe la existencia de una identidad
  transgénero que es mediada por la condición única de haber nacido en un
  cuerpo equivocado. Y otra binarista que afirma que la identidad de género de
  la persona transgénero es masculina o femenina, sin ser diferente a la de una
  persona cisgénero.

\end{enumerate}

\section{Respuestas a la pregunta de investigación}


A continuación expandimos en detalle el significado de cada una de las
conclusiones.

\subsection[La transición es un proceso continuo]{La transición es un proceso continuo que se presenta como un gradiente}

Un primer elemento que surge de la investigación es la propuesta de que el
inicio de la transición es una decisión a la que deben llegar las personas trans
y que se atraviesa como un proceso continuo. La identidad de género comienza su
construcción desde etapas tempranas de desarrollo del individuo. La disforia, o
incomodidad con el propio cuerpo, se presenta de formas diversas en la niñez. La
forma de afrontamiento aprendida y el acceso a la perspectiva de género modela
la conciencia del \emph{ser transgénero}. Sin embargo, la noción de ser
diferente, sentirse diferente, raro, extraño y distinto a los demás lleva al
individuo a la búsqueda de formas de denominación e identificación que den
respuesta a como se sienten. Esto se manifiesta en el malestar que les causa
tener que usar ropa, jugar con juguetes, o usar frases que sienten que no expresan
su identidad de género.

Planteamos la transición como un proceso continuo porque la persona transgénero
se encuentra constantemente en una construcción identitaria. La persona
transgénero se \emph{descubre} como tal en algún punto de su historia de vida.
Esto puede suceder tan temprano como la niñez o adolescencia o en las etapas
tardías de la adultez.

A partir de allí la persona transgénero puede, o no, decidir iniciar un proceso
de transición. Aunque el malestar por la incomodidad con el propio sexo y cuerpo
es involuntario, el inicio de la expresión de género y la transición es una
decisión voluntaria. Es posible por lo tanto afirmar la existencia de personas
transgénero que se identifican con el género opuesto a su sexo de nacimiento
pero que se siguen presentando como su género asignado al nacer por distintos
motivos. Esto puede ser por presión social, miedo al rechazo, violencia y
discriminación, etc.

Una vez iniciada la transición, esta es aproximada como un gradiente de
estrategias que van desde el cambio estético de ropa, cabello, forma de hablar,
gesticulación, conductas; Pasa por la terapia de reemplazo hormonal; Continúa
con la intervención quirúrgica de características sexuales secundarias; Y tiene
su expresión en el extremo opuesto en la cirugía de reasignación genital.

Este proceso de transición, una vez que la persona transgénero comienza a
presentarse como el género deseado, no tiene un final definitivo. El tener que
buscar una validación por parte de otros miembros de la sociedad lleva al
individuo a enfrentarse a situaciones que cuestionan su identidad y expresión de
género, y los motiva a solucionar la discrepancia en su identidad de género. La
búsqueda e implementación de estrategias nunca se detiene. Especialmente si el
entorno social pone presión sobre el individuo para actuar sobre la
incongruencia entre lo expresado y las expectativas de los estereotipos de
género.

\subsection{El principal deseo de la persona trans es llegar a ser cisgénero}

El proceso de transición está enfocado en la lucha del individuo
transgénero por llegar a ser cisgénero. Un deseo que no puede llegar a
completarse a cabalidad actualmente pues, bajo los estándares médicos y
científicos, una persona transgénero siempre va a permanecer como tal a pesar de
que su identidad sea otra\footnote{Dentro del contexto del reciente DSM-V se le
denomina ‘Disforia de género post-transición’ \parencite[p. 453]{APA2016}}.

Sin embargo, se podría plantear que este objetivo, el ser cisgénero, se puede
alcanzar en distintos niveles. Bien sea recurriendo a elementos que les permita
mostrarse de acuerdo al género con el que se identifican, como pueden ser el uso
de maquillaje, ropa específica o incluso cirugías estéticas. Cada persona
transgénero decide con cual nivel de la transición se siente cómodo. Pero los
tres casos de investigación coinciden en que el poder alinear el sexo genital
con su identidad de género es su principal deseo.

Desde esta perspectiva también es posible afirmar que la condición transgénero
se enfoca más en las faltas que en aquello que sobra de las características
corporales del individuo. Esto desde la propuesta de que tienen la posibilidad
de recurrir a realizarse operaciones de mamas en el caso de una persona
transgénero MaF o el uso de un dispositivo para orinar de pie en el caso de una
persona transgénero FaM. Dentro de la investigación los participantes hablan más
de la falta de caracteres sexuales secundarios que de aquellos caracteres que
pueden estar de más por su identidad transgénero.

Este elemento resuena especialmente con que los
participantes encuentran validez como hombre o mujer según la aceptación
que existe por otros miembros de la sociedad de su expresión de género.
Consecuentemente esta validación por el otro lleva a la constitución
identitaria del individuo pues, en mayor o menor medida, lleva a aliviar
aquella incongruencia y permite la alineación del individuo con el género con el
que se identifica.

Sin embargo, la rigidez de la dicotomía hombre-mujer impone modelos y
estereotipos de lo correcto o incorrecto según el sexo biológico, sin tener en
cuenta que el género es mucho más flexible, amplio y complejo. Esto lleva a que
todo lo que se salga de los parámetros preestablecidos sea juzgado y
discriminado.

\subsection[La identidad es moldeada por el patriarcado]{La identidad de la persona
transgénero es moldeada por los roles estereotípicos de género}

Partiendo de lo anterior ponemos de relieve que la vivencia de la identidad de
las personas transgénero se ve innegablemente influenciada por la visión
patriarcal que existe sobre el género. La concepción del género como binario y
los elementos de rol y expresión estereotípica de lo que es un hombre y una
mujer así como que elementos los definen.

Para que una persona transgénero sea validada por otros
tiene que llegar a ser más del género con el que se identifica que lo que podría
llegar a ser una persona cisgénero. Igualmente, la demanda social del entorno,
así como la percepción subjetiva de la persona transgénero, es tener que llegar
a demostrar esta pertenencia más allá de la demanda normal para la persona
cisgénero.

Cabe plantear que esta visión hegemónica de la identidad de género se va
construyendo a lo largo de la vida del individuo, basado en interacciones
sociales con otros miembros de la sociedad. Los roles y expresiones de género
son enseñados y reforzados por el contexto familiar, escolar y mediático desde
una temprana edad. La persona transgénero interioriza estos roles, estereotipos
y normas sociales. Durante su búsqueda de una identidad que alivie la
inconformidad con el género asignado, la persona transgénero explorará
acríticamente los roles, expresiones y normas del género opuesto al asignado. De
allí da inicio el complejo proceso de identificación que precede la
concientización de la condición transgénero.

Esta interiorización de los roles estereotípicos de género puede también tener
efectos negativos en la expresión de la persona transgénero. En ocasiones, como
el caso~3, habrán instancias o expresiones deseadas de parte de la persona
transgénero que serán inhibidas por no ser correspondientes con la identidad de
género sentida dentro la visión hegemónica machista. Por ejemplo, la
construcción de la masculinidad a través del deporte, la dureza física y la
relación promiscua y objetificante con las mujeres. Un hombre trans (MaH) que
sienta rechazo por este tipo de expresiones de género tiene problemas para pasar
por su género sentido. Ya que la sociedad demandará expresiones estereotípicas
para poder \emph{validar} la identidad de la persona transgénero como hombre.

Esto se puede componer en obstáculos difíciles para la persona transgénero que
una persona cisgénero no deberá navegar. Un hombre cisgénero puede tener una
expresión de género no estereotípica sin tener necesariamente su identidad de
género puesta en duda, más frecuentemente lo que se pondrá en duda es su
orientación sexual. Luego, tangencialmente, su cualidad masculina. Este hombre
cisgénero puede desafiar el estereotipo y el machismo hegemónico debido a su
condición sexual biológica. Una persona transgénero será descalificada como un
ente social válido si no se ajusta y conforma a la norma de género
correspondiente. Aunque se acepte su condición transgénero, su identidad se
pondrá en duda si no es expresada de forma estereotípica.

Otro factor que se suma a la influencia de los roles estereotípicos de género es
la dirección de la transición de la persona transgénero. El caso~2 expone en su
entrevista como el sentido de la transición, ya sea MaH u HaM, altera
significativamente la percepción y valoración social.

Lo que esto quiere decir es que para las mujeres trans existe una mayor
estigmatización y rechazo social. La perspectiva del patriarcado estipula la
superioridad del género masculino por encima del género femenino. Por ello, la
transición de ser hombre a ser mujer es concebida como una renuncia al
privilegio del macho dominante. Mientras que la transición de mujer a hombre es
percibido socialmente como una adquisición de privilegio, y las características
positivas asociadas con el ser hombre. De manera que una transición es vista
como una perdida y la otra como una ganancia.

\subsection{La discriminación obstaculiza el desarrollo de la persona transgénero}

Otra de los obstáculos que afecta el desarrollo de la persona transgénero es la
discriminacion que actua en contra del desarrollo identitario. El experimentar
actos discriminantes, así como otras dificultades que puede llegar a afrontar
una persona transgénero, diferencia al individuo transgénero de aquellos que son
cisgenero. Los actos discriminativos pueden ir desde violencia física, a
violencia simbólica o académica, e incluso el atentar contra la vida. Es
necesario recordar que no toda masculinidad o feminidad se construye de igual
manera pero resulta resaltante que la incongruencia de género puede llegar a
generar un malestar que las personas cisgenero no van a experimentar.

En algunos casos en los que un individuo no puede expresar su identidad
transgénero, la inhabilidad está ligada a situaciones de interacción con otros
miembros de la sociedad. También puede suceder debido a una falta de información
para concientizar el ser transgénero. Evidenciamos que la condición transgénero
puede llegar a ser desconocida por la propia persona y por esto resulta
importante mejorar su visibilización así como trabajar procesos de significación
social.

Observamos en las entrevistas cómo los participantes demuestran desconfianza
ante el entorno social, producto de experiencias sufridas de discriminación y
rechazo a lo largo de su transición. Estas vivencias discriminativas llevan a
los individuos a generar una serie de mecanismos de defensa que en definitiva
guían sus interacciones sociales. Pero cabe cuestionarse que peso pueden
llegar a tener sobre la cotidianidad del individuo así como en la forma que
modelan sus interacción con otros miembros de la sociedad.

\subsection[Dos concepciones de la identidad]{Existen dos concepciones de la
identidad de género de la personas trans}

Durante el trabajo de investigación se hicieron claras dos posturas sobre la
identidad de las personas transgénero. Siendo estas representadas por las
opiniones de los casos 2 y 3. Las dos posturas pueden ser explicadas de la
siguiente manera.

La primera, la cual defiende el caso~3, considera que existe una identidad
transgénero. Esta identidad es diferente de la identidad de género como hombre o
mujer, y también de la identidad género fluida. Esta concepción tiene relación
con las interpretaciones feministas sobre el género. Desde esta se interpreta
que, si bien la persona transgénero se identifica como hombre o mujer, su
condición de haber nacido con un sexo e identificarse con el género opuesto al
que la sociedad le corresponde le coloca en una situación especial y única. Esta
condición de ser transgénero es suficiente para hablar de una identidad distinta
a la de la persona cisgénero.

La segunda postura, defendida por el caso~2, plantea que no existe tal cosa como
la identidad transgénero. Que una persona transgénero masculina (MaH) posee una
identidad de género similar e indistinguible de la identdidad masculina de una
persona cisgénero. Esta mirada reivindica el género binario y reafirma los roles
tradicionales de género que predominan en la sociedad venezolana. Su mayor
defensa es aquella planteada por el caso~2 al decir:

\begin{verbatim}
Yo no soy un transgénero, soy un hombre con todas las de la ley. Sólo que nací
en un cuerpo equivocado.
\end{verbatim}

Estas dos posturas de la identidad conviven en distintos individuos de la
población transgénero y transexual. En ocasiones separando las opiniones en
cuanto a la forma de abordar la opinión pública en la lucha por derechos civiles
de las personas trans.

\section{¿Qué puede hacer la psicologia?}

Un elemento que se hace presente en las experiencias de todos los participantes
es la falta de orientación o guía no solo en la persona transgénero sino también
en sus familiares y personas con las que comparten cotidianamente. Este es un
elemento en el cual la psicología podría ayudar mucho para evitar conflictos o
malestares.

Tomando en cuenta verbatims como el del caso~1:

\begin{verbatim}
…en la escuela era algo terrible por el bullying, pero yo siempre imponía mi
carácter y jamás me deje amedrentar por nada ni nadie. De hecho me agarre a
golpes y me expulsaron por 10 días…
\end{verbatim}

Así como otros elementos encontrados que hacen referencia a la vulnerabilidad de
la población transgénero. Existe una preponderancia a vivir situaciones de
discriminación laboral que los puede llevar a recurrir a la prostitución como
medio para sobrevivir, o que les puede llevar a agotar los recursos psicológicos
para enfrentar situaciones adversas. Esto puede terminar en decisiones erráticas
y extremistas como puede ser el atentar contra la propia vida.

Se propone que tanto el transgenerismo como la transexualidad se
visibilicen aún más en la investigación académica. Esto podría
generar estrategias que permitan a las distintas ramas de la
psicología adentrarse en estos fenómenos y ofrecer respuestas a
los diversos malestares que puede experimentar una persona
transgénero o una transexual.

Podría enfocarse en generar un inventario de estrategias a ser utilizadas para
sensibilizar a la población en general. Esto puede llegar a un punto en el que
lo diverso de la condición transgénero deje de ser disruptor y pase a ser un
elemento más de la variedad de la identidad humana. Promover la participación de
psicólogos dentro de organizaciones que puedan hacer acompañamiento no solo a la
persona transgénero sino también a sus familiares más directos, para que se
pueda vivir el proceso de transición mas armoniosamente.

Estas son solo unas propuestas de las que se pueden presentar desde la
psicologia para atender las dificultades que pueden enfrentar
las personas transgénero.

\section{Limitaciones y recomendaciones}

Finalmente se puede señalar que los objetivos de la presente investigación
fueron cumplidos en su mayoría. Las entrevistas permitieron una aproximación a
la identidad de las personas transgénero así como también aportaron
conocimientos sobre una población muy poco estudiada en Venezuela. Sin embargo,
resultaría beneficioso que se tomara como base esta investigación y se abordara
desde otras perspectivas para abarcar los aspectos cognitivos y dinámicos que
surgieron en este trabajo.

En cuanto al objetivo de abordar las estrategias de acceso a las técnicas de
transición, aunque pudimos encontrar algunos elementos tangencialmente
relacionados con el acceso a la transición. No hemos alcanzado la totalidad del
objetivo planteado. Esto se debe a la larga extensión de las entrevistas sólo
para conversar sobre elementos identitarios. La dificultad de contactar,
adquirir la confianza y concretar momentos de entrevista con participantes
transgénero o transexuales. En la siguiente sección de recomendaciones
proponemos algunas estrategias que pueden ser utilizadas por otras
investigaciones para solventar estas dificultades.

Resulta necesario expresar que pese a que se buscó un acercamiento completo
sobre como es la vivencia de la transición en personas transgénero se
presentaron condiciones que, aunque no impidieron el normal desarrollo de la
investigación, si llegaron a limitar el alcance de la misma.

Una de estas limitaciones se presentó al momento de acceder a la muestra. Se
presentó resistencia por parte de candidatos a participantes al momento de
coordinar entrevistas. En algunos casos los candidatos dejaron de responder a
los intentos de contacto que eran realizados por los investigadores. Sumado a
esto se presenta una situación contingente basada en la escasez de medicamentos
para seguir el tratamiento hormonal que usualmente llevan las personas
transgénero, esta condición ha llevado a buena parte de la población transgénero
a migrar hacia otros países para poder continuar con su tratamiento.

Sumado a esto existe un reducido número de estudios académicos desde una mirada
psicológica sobre personas Trans (tanto transgénero como transexuales) en
Venezuela y este es un factor que puede llegar a ser un limitante al momento de
plantear la investigación. Sin embargo, es necesario resaltar que en la presente
investigación se tomó en cuenta investigaciones previas realizadas dentro de
la mención de psicología clínica de la Universidad Central de Venezuela.

Tomando en cuenta estas limitantes así como también lo construido en base a la
información recopilada por la presente investigación se plantean las siguientes
recomendaciones para futuras investigaciones

\begin{itemize}

\item Se sugiere continuar indagando en este tema de
investigación dentro de la perspectiva de los estudios de género. Esta es un
área que requiere mayor profundización. Debido a múltiples razones, este tipo de
muestra estudiada pertenece a una población invisibilizada y discriminada, no
sólo en lo social, legal, teórico sino también en el plano de la investigación
psicológica.

\item Debido a que la concepción de género atravesó constantemente el discurso
sobre la corporalidad de estos participantes, es necesario ahondar en el género
como elemento regulador del cuerpo. Así como en el rol del cuerpo en la
construcción de la identidad de género.

\item Se recomienda emplear otras estrategias de recolección de información,
como por ejemplo grupos focales que permiten ver las interacciones entre
distintos puntos de vista de actores diversos sobre un mismo tema en común.

\item Se recomienda investigar desde la mirada psicosocial la percepción y los
imaginarios simbólicos que existen en la población en general de la sociedad
venezolana sobre las personas transgénero y transexuales. También explorar el
ámbito familiar, escolar y laboral alrededor de las personas transgénero y
transexuales.

\item Se recomienda sistematizar las estrategias de conocimiento y acceso de los
distintos métodos de transición que son accesibles o están disponibles en
Venezuela. Así también como las limitaciones y dificultades que son resultado de
la actual situación política y económica venezolana. Una posibilidad es realizar
un inventario de estrategias de acceso en cooperación con servicios médicos que
ofrezcan atención a la población trans.

\item Igualmente se sugiere investigar las estrategias de afrontamiento
utilizadas por las personas transgénero para lidiar con el rechazo y la
discriminación. Así como evaluar modelos de intervención que permitan a los
psicólogos clínicos realizar intervenciones positivas en terapia que empoderen a
las personas transgénero sobre su propia identidad y el afrontamiento de
situaciones adversas.

\item Incentivar la realización de talleres y capacitaciones sobre identidad de
género y también sobre los derechos humanos, para así enfrentar la
discriminación y el acoso. Esto crea un conflicto tanto interno como externo en
la persona que está recibiendo este maltrato emocional ya sea en su lugar de
trabajo, en el hogar, lugares público, entre otros.

\item Se recomienda a los participantes, y la población transgénero en general,
realizar psicoterapia para adquirir herramientas que les ayuden a afrontar de
forma más efectiva el rechazo de la sociedad y la discriminación a la que están
sometidos con frecuencia. De igual forma trabajar los niveles de ansiedad y
depresión a los que tienen riesgo, así como también para desarrollar habilidades
sociales que les permita establecer relaciones interpersonales profundas y
significativas.

\item También es necesario desarrollar y entrenar a los estudiantes de
psicología en estrategias, métodos de apoyo y acompañamiento para personas con
identidades de género no binarias.

\end{itemize}
