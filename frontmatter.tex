%\chapter{Agradecimientos}

\chapter{Resumen}

\begin{center}
	\large \scshape \thetitle
\end{center}

\begin{quote}
Esta investigación tiene como fin explorar el lugar de la transición en
la construcción de la identidad de las personas transexuales que hacen vida en
la ciudad de Caracas empleando como participantes aquellas personas mayores de
18 años que se identifiquen a sí mismas como transgénero o transexuales.
Se seguirá el criterio de saturación para establecer el límite de participantes.
Se tratará de una investigación descriptiva fenomenológica con enfoque
cualitativo, que emplea el método propuesto por la teoría fundamentada para
poder explorar el fenómeno que es el foco de la presente investigación.
\end{quote}

\itshape \textbf{Palabras clave:} \normalfont Transexualidad, Género, Identidad, Transición, Patologización.

\chapter{Abstract}

\begin{center}
	\large \scshape \theengtitle
\end{center}

\begin{quote}
This research aims to explore the place of gender transition on the
construction of the identity of transexual people who live in Caracas city.
The participant population are persons aged above 18 who identify themselves as
transgender or transexual.
The saturation criteria will be used to establish the sample size.
It is a descriptive phenomenological research with a qualitative approach, which
will use the method proposed by the grounded theory to explore the phenomenon
that is the focus of the present investigation.
\end{quote}

\itshape \textbf{Keywords:} \normalfont Transexuality, Gender, Identity, Transition, Patholgization

\cleardoublepage