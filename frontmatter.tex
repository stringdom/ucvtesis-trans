\chapter{Agradecimientos}

\vspace{\stretch{1}}
\begin{quotation}
\flushright\textsl{
Queremos darle las gracias a los participantes por haber aportado su tiempo y
paciencia para hacer posible esta investigación. En especial a Ale por su
dedicación a responder en detalle todas nuestras dudas. Muchas gracias a
nuestra tutora por su guía en el proceso de investigación. También a nuestros
colegas cuya opinión crítica nos recuerda siempre mantener la vigilancia
epistemológica.
}
\end{quotation}

\vspace{\stretch{3}}

\chapter{Dedicatoria}

\chapter{Resumen}
% \begin{center}
% 	\large\scshape\thetitle\
% \end{center}

\begin{quote}
\small
La investigación tuvo como fin explorar la construcción de la identidad de un
grupo de personas transgénero que hacen vida en la ciudad de Caracas. Se empleó
una metodología cualitativa con perspectiva fenomenológica, basándose en el
estudio de casos. La recolección de datos se realizó a través de entrevistas a
profundidad, las cuales fueron realizadas a adultos mayores de 18 años de ambos
sexos que se identifican a sí mismos como transgénero y fueron codificadas
empleando el programa informático ATLAS.ti\@. Dicha codificación dio como
resultado 6 categorías que constituyen los elementos esenciales que permiten la
construcción de identidad en los participantes. Se pudo observar que la
identidad de estas personas se encuentra mediada en gran parte por la
interacción con otros individuos. La transición es una herramienta de
reafirmación tanto personal como social de la propia identidad de género. La
transición se vive como un continuo que tiene como fin último ser cissexual,
aunque se acepte que no existe actualmente la capacidad técnica viable para
lograr ese objetivo. Por ello la persona transgénero despliega un conjunto de
estrategias secundarias para compensar y lograr aproximarse lo más posible a la
expresión de género que consideran apropiada para su identidad. La
discriminación juega un papel de obstáculo que afecta el desarrollo de la
identidad de las personas transgénero.
\end{quote}

\itshape\textbf{Palabras clave:}\normalfont{}\ Transición, transgénero,
transexualidad, género, identidad, discriminación, desarrollo, cuerpo,
expresión de género.

\chapter{Abstract}

\begin{center}
	\large\scshape\theengtitle\
\end{center}

\begin{quote}
\small
This research had the intention of exploring the construction of
transgender's identity that live in the city of Caracas. A qualitative
methodology with a phenomenological perspective is used, by way of case
study. Data collection was done through in-depth interviews, which were
conducted to adults over 18 who identify themselves as
transgenders. Interviews were codified using the software ATLAS.ti\@. This
coding resulted in 6 categories that constitute the essential elements that
allow the construction of identity in transgender people. We found that their
identity is largely mediated by the interaction with other individuals.
Transition is a tool used to reassure their own gender identity both
personally and socially. This transition is experienced as a continuum whose
ultimate goal is to become cissexual, while accepting that this is not
currently technically possible. For this reason, transgender people make use
of a set of secondary strategies to compensate and approximate their desired
gender expression according to their identity. Discrimination plays a major
role as an obstacle that encumbers the development of identity in transgender
people.
\end{quote}

\itshape\textbf{Keywords:}\normalfont{}\ Transition, transgender, transsexual,
gender, identity, discrimination, development, body, gender expression.
