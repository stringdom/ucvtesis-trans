%\chapter{Agradecimientos}

\chapter{Resumen}
\begin{center}
	\large\scshape\thetitle\
\end{center}

\begin{quote}
La investigación tuvo como fin explorar la construcción de la identidad
de un grupo de personas transgénero que hacen vida en la ciudad de Caracas.
Se empleó una metodología cualitativa con perspectiva fenomenológica, basándose
en el estudio de casos para poder explorar el foco de dicha investigación.
La recolección de datos se realizó a través de entrevistas a profundidad, las
cuales fueron realizadas a adultos mayores de 18 años de ambos sexos que se
identifiquen a sí mismos como transgénero y fueron codificadas empleando el
programa ATLAS.TI\@.
Dicha codificación tuvo como resultado 6 categorías que constituyen elementos
que permiten la construcción de identidad en personas transgénero.
Se pudo observar una reafirmación respecto a esta, es decir, la identidad de
estas personas se encuentra mediada en gran parte por la interacción con otros
individuos.
\end{quote}

\itshape\textbf{Palabras clave:}\normalfont{}Transexualidad, Género, Identidad,
Transición.

\chapter{Abstract}

\begin{center}
	\large\scshape\theengtitle\
\end{center}

\begin{quote}
This research has the intention of the exploring the identity construction of
transsexual people that live in the city of Caracas.
It made use of a qualitative methodology with a phenomenological perspective, by
making use of a case study to explore the focus of the present investigation.
Data collection was done through in-depth interviews, which were conducted to
adults over 18 who identify themselves as transgenders or transsexuals and it
was codified using the program ATLAS.TI\@.
This coding resulted in 6 categories that constitute elements that allow the
construction of identity in trans people.
Trough this the research was able to reassure that identity it's largely
mediated by the interaction with other individuals.

\end{quote}

\itshape\textbf{Keywords:}\normalfont{}Transexuality, Gender, Identity,
Transition.

\cleardoublepage\
